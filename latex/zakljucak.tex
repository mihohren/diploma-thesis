\chapter{Zaključak}\label{cha:zakljucak}
Cilj ovog rada bio je ponuditi pregled visoke razine na alat za dokazivanje Coq te
ga primijeniti na formalizaciju logike prvog reda s induktivnim definicijama.
U tu svrhu smo u prvom poglavlju prikazali osnove Coqa i teorije na kojoj se on temelji,
dok smo ostala poglavlja posvetili logici.
Definirali smo osnovne pojmove poput formule, strukture, standardnog modela te dokaznog sustava
i dokazali smo neka njihova svojstva.
Kroz cijeli rad ilustrirali smo definicije i leme neformalnim i formaliziranim primjerima.
Iako je ponekad bilo teže razlučiti bitne dijelove neformalne definicije,
ipak se pokazalo da je Coq prikladan alat za formalizaciju logike te autor smatra
da je cilj rada uspješno postignut.

Ovim radom smo tek zagrebali površinu formalizacije logike prvog reda s induktivnim definicijama.
Prirodni nastavak rada jest dokaz potpunosti sustava \(\mathit{LKID}\) pomoću Henkinovih modela,
a nakon toga i formalizacija cikličkog dokaznog sustava \(\mathit{CLKID}^{\omega}\).
Nadalje, ima smisla formalizirati dokazivače teorema za sustave \(\mathit{LKID}\)
i \(\mathit{CLKID}^{\omega}\). Kako logika prvog reda nije odlučiva, takvi dokazivači teorema
trebali bi uvesti dodatne pretpostavke (kao što je pretpostavka zatvorenog svijeta u Prologu)
ili se usmjeriti prema interaktivnom dokazivanju.
Za našu formalizaciju sustava \(\mathit{LKID}\),
ima smisla iskoristiti dodatne Coqove mogućnosti,
posebno za automatizaciju dokaza jezikom taktika Ltac
te za prikladniju notaciju terma, formula i dokaza.

Po autorevu mišljenju, računalna formalizacija matematike je odličan način razmišljanja i razumijevanja,
posebno kada se koristi zajedno s tradicionalnim učenjem jer tada u potpunosti
postaje jasan pojam koji se proučava --- formalizacija nas tjera da budemo
posebice pažljivi. Nadalje, u današnje vrijeme umjetne inteligencije i automatski
generiranog koda, dokazivanje točnosti programa postaje bitnije no što je ikad bilo.
Za kraj ćemo reći da je formalizacija često naporan ali uvijek zadovoljavajuć proces.

%%% Local Variables:
%%% mode: LaTeX
%%% TeX-master: "master"
%%% End:
