\chapter{Logika prvog reda s induktivnim definicijama}\label{cha:logika-prvog-reda}
U ovom poglavlju predstavljamo glavne rezultate diplomskog rada koji uključuju formalizaciju
logike prvog reda s induktivnim definicijama \(\mathit{FOL_{ID}}\)
te dokaznog sustava \(\mathit{LKID}\), koje je prvi uveo Brotherston~\cite{brotherston2005}.
Definicije, leme i dokazi u ovom poglavlju preuzete su iz Brotherstonove disertacije~\cite{brotherstonphd}.

Prvo ćemo definirati sintaksu i semantiku logike \(\mathit{FOL_{ID}}\),
nakon čega ćemo definirati njene standardne modele.
Zatim ćemo prikazati dokazni sustav \(\mathit{LKID}\) te konačno dokazati adekvatnost
sustava \(LKID\) s obzirom na standardnu semantiku,
što je ujedno i glavni rezultat ovog diplomskog rada.

Svaka definicija i lema u ovom poglavlju bit će popraćena svojim pandanom u Coqu.
Jedan je od ciljeva diplomskog rada prikazati primjene Coqa u matematici,
zbog čega leme nećemo dokazivati \enquote{na papiru},
već se dokaz svake leme može pronaći na GitHub repozitoriju rada.\footnote{TODO: REPO LINK}
Zainteresiranom čitatelju predlažemo interaktivni prolazak kroz dokaze lema.

Prije no što krenemo na formalizaciju, valja prokomentirati odnos matematičkog i Coqovog vokabulara.
U matematici pojam \enquote{skup} može imati dva značenja;
prvo se odnosi na skupove kao \textit{domene diskursa},
dok se drugo odnosi na skupove kao \textit{predikate}, odnosno podskupove.
Primjerice, skup prirodnih brojeva \(\mathbb{N}\) je domena diskursa kada je riječ o svim prirodnim brojevima
te zbog toga pišemo \(n \in \mathbb{N}\) umjesto \(\mathbb{N}(n)\).
S druge strane, skup svih parnih brojeva \(E\) je podskup skupa \(\mathbb{N}\), a može se interpretirati
kao predikat na prirodnim brojevima te pišemo \(E(n)\) umjesto \(n \in E\).
U Coqu se skupovi kao domene diskursa formaliziraju tipovima sorte \coqset\footnote{Ili općenito kao tipovi sorte \texttt{Type}.},
dok se skupovi kao predikati formaliziraju funkcijama iz domene diskursa u sortu \coqprop.
Na primjer, tip prirodnih brojeva \texttt{nat} je sorte \texttt{Set}, a predikat \texttt{Nat.Even} je tipa \texttt{nat\(\rightarrow\)Prop}.

\section{Sintaksa}\label{sec:sintaksa}
Kao i u svakom izlaganju logike, na početku je potrebno definirati sintaksu.
\begin{definition}[Signatura]\label{def:signatura}
  \textit{Jezik prvog reda s induktivnim predikatima} (kratko signatura), u oznaci \(\Sigma\),
  je skup simbola od kojih razlikujemo \textit{funkcijske}, \textit{obične predikatne}
  i \textit{induktivne predikatne} simbole.
  Mjesnost simbola reprezentiramo funkcijom iz odgovarajućeg skupa simbola u skup \(\mathbb{N}\).
  Funkcijski simboli mjesnosti nula nazivaju se \textit{konstante},
  a predikatni simboli mjesnosti nula nazivaju se \textit{propozicije}.
  \begin{minted}{coq}
Structure signature := {
  FuncS : Set;
  fun_ar : FuncS -> nat;
  PredS : Set;
  pred_ar : PredS -> nat;
  IndPredS : Set;
  indpred_ar : IndPredS -> nat
}.
\end{minted}
\end{definition}

\begin{example}
  \misao{\(\Sigma_{PA}\)}
\end{example}

\noindent U ostatku poglavlja promatramo jednu proizvoljnu, ali fiksiranu signaturu \(\Sigma\).
Fiksiranje nekog proizvoljnog objekta je česta pojava u matematici,
prvenstveno zato što finksirane argumente ne trebamo spominjati eksplicitno.
Coq omogućuje fiksiranje naredbom \texttt{Context},
pod uvjetom da se korisnik nalazi u \texttt{Section} okolini.\footnote{\url{https://coq.inria.fr/doc/v8.18/refman/language/core/sections.html}}
Većina definicija i lema u ovom radu su napisane upravo unutar \texttt{Section} okoline.

\begin{definition}[Term]\label{def:term}
  \textit{Varijabla} je prirodan broj. Skup terma konstruiramo rekurzivno na način:
  \begin{enumerate}
  \item svaka varijabla je term;
  \item ako je \(f\) funkcijski simbol mjesnosti \(n\) te su \(t_{1}\), \ldots, \(t_{n}\) termi\footnote{Primijetimo, broj terma ovisi o mjesnosti funkcijskog simbola. U Coq implementaciji ovog \enquote{konstruktora} možemo vidjeti da je on zavisnog tipa.},
    onda je \(f(t_{1}, \ldots, t_{n})\) također term.
  \end{enumerate}
\begin{minted}{coq}
Inductive term : Set :=
| var_term : var -> term 
| TFunc : forall (f : FuncS Σ), vec term (fun_ar f) -> term.
\end{minted}
\end{definition}
Općenitije prezentacije logike prvog reda za skup varijabli uzimaju proizvoljan skup \(\mathcal{V}\),
no za formalizaciju je pogodniji skup prirodnih brojeva \(\mathbb{N}\).
Kasnije ćemo objasniti zašto.

\noindent Princip indukcije za \texttt{term} potrebno je ručno definirati.
Naime, induktivni tip \texttt{term} je \textit{ugniježđen} po konstruktoru \texttt{TFunc} što znači da se
javlja omotan oko drugog induktivnog tipa\footnote{Ovdje \texttt{vec}.} kao argument.
Za ugniježđene induktivne tipove, Coq generira \textit{neprikladne} principe indukcije.
\begin{minted}{coq}
Lemma term_ind
     : forall P : term Σ -> Prop,
         (forall v, P (var_term v)) ->
         (forall f args, (forall st, V.In st args -> P st) ->
                         P (TFunc f args)) ->
       forall t : term Σ, P t.
\end{minted}

\begin{definition}
  Skup svih varijabli koje se javljaju u termu \(t\), u oznaci \(\mathit{TV}(t)\), konstruiramo rekurzivno na način:
  \begin{enumerate}
  \item za varijablu \(v\) vrijedi \( \mathit{TV}(v) = \{ v \} \),
  \item za \(n\)-mjesni funkcijski simbol \(f\) i terme \(t_{1}, \ldots, t_{n}\) vrijedi \( \mathit{TV}(f(t_{1}, \ldots, t_{n})) = \bigcup_{1 \le i \le n} \mathit{TV}(t_{i}) \).
  \end{enumerate}
\begin{minted}{coq}
Inductive TV : term -> var -> Prop :=
| TVVar : forall v, TV (var_term v) v
| TVFunc : forall f args v st, V.In st args ->
                               TV st v -> TV (TFunc f args) v.
\end{minted}
\end{definition}

\begin{definition}[Formula]\label{def:formula}
  Skup formula konstruiramo rekurzivno na način:
  \begin{enumerate}
  \item ako je \(Q\) običan ili induktivan predikatni simbol mjesnosti \(n\) te su \(t_{1}\), \ldots, \(t_{n}\) termi,
    onda je \(Q(t_{1}, \ldots, t_{n})\) \textit{atomarna} formula;
  \item ako je \(\varphi\) formula, onda su \(\neg\varphi\) i \(\forall\varphi\) također formule;
  \item ako su \(\varphi\) i \(\psi\) formule, onda je \(\varphi \rightarrow \psi\) također formula.
  \end{enumerate}
\begin{minted}{coq}
Inductive formula : Set :=
| FPred (P : PredS Σ) : vec (term Σ) (pred_ar P) -> formula 
| FIndPred (P : IndPredS Σ) : vec (term Σ) (indpred_ar P) -> formula 
| FNeg : formula -> formula 
| FImp : formula -> formula -> formula 
| FAll : formula -> formula.
\end{minted}
  \noindent Ostale veznike definiramo kao sintaksne pokrate.
\begin{minted}{coq}
Definition FAnd (φ ψ : formula) : formula := FNeg (FImp φ (FNeg ψ)).
Definition FOr (φ ψ : formula) : formula := FImp (FNeg φ) ψ.
Definition FExist (φ : formula) : formula := FNeg (FAll (FNeg φ)).
\end{minted}
\end{definition}
Umjesto eksplicitne kvantifikacije po nekoj varijabli \(v\),
mi ćemo implicitno kvantificirati po varijabli \(0\).
Ovaj pristup kvantifikaciji\footnote{Ili općenitije, vezivanju varijabli.}, imena \enquote{de Bruijnovo indeksiranje},
bitno olakšava rad sa supstitucijama, a uveden je u članku~\cite{debruijnamelessdummies}.
O samoj implementaciji de Bruijnovog indeksiranja više se može pročitati u knjizi
\textit{Types and Programming Languages}~\cite{pierce2002tapl}.
Za potrebe ovog rada koristili smo program \textit{Autosubst2}\footnote{\url{https://github.com/uds-psl/autosubst2}}~\cite{stark-phd,autosubst2}
za automatsko generiranje tipova terma i formula te pripadajućih funkcija supstitucija i pomoćnih lema.
\misao{Ovaj odlomak preseliti nekamo na početak odjeljka.}

\begin{definition}
  Skup slobodnih varijabli formule \(\varphi\), u oznaci \(\mathit{FV}(\varphi)\), konstruiramo rekurzivno na način:
  \begin{enumerate}
  \item \( \mathit{FV}(P(u_{1}, \ldots, u_{n})) = \bigcup_{1 \le i \le n} \mathit{TV}(u_{i}) \),
  \item \( \mathit{FV}(\neg\varphi) = \mathit{FV}(\varphi)\),
  \item \( \mathit{FV}(\varphi \rightarrow \psi) = \mathit{FV}(\varphi) \cup \mathit{FV}(\psi) \),
  \item \( \mathit{FV(\forall \varphi)} = \{ v \mid 1 + v \in \mathit{FV}(\varphi) \} \).
  \end{enumerate}
\begin{minted}{coq}
Inductive FV : formula -> var -> Prop :=
| FV_Pred : forall R args v st,
    V.In st args -> TV st v -> FV (FPred R args) v
| FV_IndPred : forall R args v st,
    V.In st args -> TV st v -> FV (FIndPred R args) v
| FV_Imp_l : forall F G v, FV F v -> FV (FImp F G) v
| FV_Imp_r : forall F G v, FV G v -> FV (FImp F G) v
| FV_Neg : forall F v, FV F v -> FV (FNeg F) v
| FV_All : forall F v, FV F (S v) -> FV (FAll F) v.
\end{minted}
\end{definition}

\begin{definition}[Supstitucija]
  Supstitucija je svaka funkcija iz skupa \(\mathbb{N}\) u skup terma, a domena joj se može rekurzivno proširiti na skup terma i skup formula.
\begin{minted}{coq}
Fixpoint subst_term  (σ : var -> term) (t : term) : term :=
  match t with
  | var_term v => σ v
  | TFunc f args => TFunc f (V.map (subst_term σ) args)
  end.
\end{minted}
\begin{minted}{coq}
Fixpoint subst_formula
  (σ : var -> term Σ) (φ : formula )
  : formula  :=
  match φ return formula  with
  | FPred P args => FPred P (V.map (subst_term σ) args)
  | FIndPred P args => FIndPred P (V.map (subst_term σ) args)
  | FNeg ψ => FNeg (subst_formula σ ψ)
  | FImp ψ ξ => FImp (subst_formula σ ψ) (subst_formula σ ξ)
  | FAll ψ => FAll (subst_formula (up_term_term σ) ψ)
  end.
\end{minted}
  \noindent Ovdje funkcija \textit{up\_term\_term} brine da supstitucija \(\sigma\) mijenja samo one varijable koje nisu vezane.
\end{definition}

Konačno, potrebno je definirati sintaksu za indukciju.
U Coqu su definicije induktivnih propozicija proizvoljne do na ograničenje pozitivnosti,
no radi jednostavnosti u \(\mathit{FOL_{ID}}\) su moguće samo
induktivne definicije s atomarnim formulama, a pišemo ih u stilu prirodne dedukcije:
\begin{prooftree}
  \AxiomC{\(  Q_{1}\mathbf{u}_{1}  \ldots   Q_{n}\mathbf{u}_{n}  \)}
  \AxiomC{\(  P_{1}\mathbf{v}_{1}  \ldots   P_{m}\mathbf{v}_{m}  \)}
  \BinaryInfC{\(P\mathbf{t}\)}
\end{prooftree}
\noindent Ovdje su \(Q_{1}, \ldots, Q_{n}\) obični predikatni simboli, \(P_{1}, \ldots, P_{m}\) i \(P\) su induktivni predikatni simboli,
a podebljani znakovi predstavljaju vektore terma odgovarajućih duljina.

\begin{definition}[Produkcija]
  Uređene četvorke
  \begin{enumerate}
  \item listi parova običnih predikatnih simbola i vektora terma odgovarajućih duljina,
  \item listi parova induktivnih predikatnih simbola i vektora terma odgovarajućih duljina,
  \item induktivnog predikatnog simbola \(P\) mjesnosti \(n\) i
  \item vektora terma duljine \(n\)
  \end{enumerate}
  nazivamo produkcijama.
\begin{minted}{coq}
Record production := mkProd {
    preds : list ({ P : PredS Σ & vec (term Σ) (pred_ar P) });
    indpreds : list ({ P : IndPredS Σ & vec (term Σ) (indpred_ar P) });
    indcons : IndPredS Σ;
    indargs : vec (term Σ) (indpred_ar indcons);
  }.
\end{minted}
  \textit{Skup induktivnih definicija} \(\Phi\) je skup produkcija.
  U ostatku rada promatramu neki proizvoljan, ali fiksiran skup induktivnih definicija \(\Phi\).
\begin{minted}{coq}
Definition IndDefSet := production -> Prop.
\end{minted}
\end{definition}


\section{Semantika}\label{sec:semantika}
\begin{definition}[Struktura]
  Struktura prvog reda \(M\) je uređena četvorka skupa kojeg nazivamo \textit{nosačem}
  te interpretacija funkcijskih, običnih predikatnih i induktivnih predikatnih simbola.
  Funkcijski se simboli interpretiraju kao \(n\)-mjesne funkcije, a predikatni simboli kao \(n\)-mjesne relacije na nosaču.
  Koristit ćemo ime strukture kao sinonim za njen nosač, a interpretacije označavamo sa \(f^{M}\) odnosno \(P^{M}\).
\begin{minted}{coq}
Structure structure := {
    domain :> Set;
    interpF (f : FuncS Σ) : vec domain (fun_ar f) -> domain;
    interpP (P : PredS Σ) : vec domain (pred_ar P) -> Prop;
    interpIP (P : IndPredS Σ) : vec domain (indpred_ar P) -> Prop;
  }.
\end{minted}
\end{definition}

\begin{definition}
  Neka je \(M\) proizvoljna struktura. \textit{Okolina} \(\rho\) za \(M\) je proizvoljna funkcija iz skupa prirodnih brojeva u nosač strukture.
\begin{minted}{coq}
Definition env := var -> M.
\end{minted}
  \noindent Okolina se može interpretirati kao niz \(d_{0}, d_{1}, d_{2}, \ldots\). Tada je \textit{pomaknuta okolina}, u oznaci \(d \cdot \rho\),
  niz \(d, d_{0}, d_{1}, d_{2}, \ldots\) za neki \(d \in M\).
  \noindent Proširenje domene okoline \(\rho\) na skup terma zovemo \textit{evaluacijom}.
\begin{minted}{coq}
Fixpoint eval (ρ : env) (t : term Σ) : M :=
  match t with
  | var_term x => ρ x
  | TFunc f args => interpF f (V.map (eval ρ) args)
  end.
\end{minted}
\end{definition}

\begin{definition}[Relacija ispunjivosti]
  Neka je \(M\) proizvoljna struktura te \(\rho\) okolina za \(M\).
  Ispunjivost formule \(\varphi\) u okolini \(\rho\) pišemo \(\rho \vDash \varphi\), a definiramo rekurzivno na način:
  \begin{enumerate}
  \item ako je \(P\) običan ili induktivan predikatni simbol mjesnosti \(n\) te su \(u_{1}, \ldots, u_{n}\) termi,
    onda vrijedi \(\rho \vDash P(u_{1}, \ldots, u_{n})\) ako i samo ako vrijedi \(P^{M}(\rho(u_{1}), \ldots, \rho(u_{n}))\),
  \item vrijedi \(\rho \vDash \neg \varphi \) ako i samo ako ne vrijedi \(\rho \vDash \varphi\),
  \item vrijedi \(\rho \vDash \varphi \rightarrow \psi \) ako i samo ako \(\rho \vDash \varphi\) povlači \(\rho \vDash \psi\) i
  \item vrijedi \(\rho \vDash \forall \varphi\) ako i samo ako za sve \(d \in M\) vrijedi \(d \cdot \rho \vDash \varphi\)
  \end{enumerate}
\begin{minted}{coq}
Fixpoint Sat (ρ : env M) (F : formula Σ) : Prop :=
  match F with
    | FPred P args => interpP P (V.map (eval ρ) args)
    | FIndPred P args => interpIP P (V.map (eval ρ) args)
    | FNeg G => ~ Sat ρ G
    | FImp F G => Sat ρ F -> Sat ρ G
    | FAll G => forall d, Sat (d .: ρ) G
  end.
\end{minted}
\end{definition}

\begin{lemma}
  Sintaktička i semantička supstitucija komutiraju pod relacijom ispunjivosti.
\begin{minted}{coq}
Lemma strong_form_subst_sanity2 :
  forall (φ : formula Σ) (σ : var -> term Σ)
      (M : structure Σ) (ρ : env M),
      ρ ⊨ (subst_formula σ φ) <-> (σ >> eval ρ) ⊨ φ.
\end{minted}
\end{lemma}
\misao{Zvuči kul. Je li jasno što je to semantička supstitucija i zašto kažem da komutiraju?}

\section{Standardni modeli}\label{sec:standardni-modeli}
Želimo ograničiti semantička razmatranja na samo one strukture koje \enquote{imaju smisla} za induktivne predikate.
Prisjetimo se, predikatni simbol \(P\) mjesnosti \(n\) interpretira se na strukturi \(M\)  podskupom skupa \(M^{n}\).
Indukciju smatramo dokazivanjem \textup{u razinama} pa ima smisla promatrati \textit{razine interpretacije} induktivnog predikata,
gdje je nulta razina prazan skup,
a svaku iduću razinu konstruiramo pomoću produkcija induktivnog skupa definicija i prethodnih razina.
Tako je prva razina onaj podskup kojeg možemo dobiti samo jednom \enquote{primjenom produkcija},
druga je razina onaj podskup kojeg možemo dobiti pomoću dvije primjene produkcija, i tako dalje.
Na taj se način, korak po korak, gradi \textit{smislena} interpretacija induktivnih predikata.
Napominjemo da se zbog mogućih međuovisnosti induktivnih predikata razine interpretacije definiraju simultano.
Ovaj odjeljak posvećujemo formalizaciji ovih pojmova.

\begin{definition}[Operator skupa definicija]
  Neka je \(M\) proizvoljna struktura te neka je \(pr\) proizvoljna produkcija induktivnog skupa definicija \(\Phi\)
  u čijoj se konkluziji javlja induktivni predikatni simbol \(P\), primjerice:
  \begin{prooftree}
  \AxiomC{\(  Q_{1}\mathbf{u}_{1}  \ldots   Q_{n}\mathbf{u}_{n}  \)}
  \AxiomC{\(  P_{1}\mathbf{v}_{1}  \ldots   P_{m}\mathbf{v}_{m}  \)}
  \BinaryInfC{\(P\mathbf{t}\)}
  \end{prooftree}
\noindent Neka je \(f\) funkcija koja svakom induktivnom predikatnom simbolu \(R\) pridružuje podskup skupa \(M^{|R|}\).
Tada je \(\varphi_{pr}(f)\) skup svih \(|P|\)-torki \(\mathbf{d}\) elemenata nosača \(M\) za koje postoji okolina \(\rho\) za \(M\) takva da:
\begin{itemize}
\item za sve \( i \in \{ 1, \ldots, n \} \) vrijedi \( \mathbf{u}_{i}^{\rho} \in Q_{i}^{M} \), % TODO uvesti ovu notaciju za u
\item za sve \( j \in \{ 1, \ldots, m \} \) vrijedi \( \mathbf{v_{i}^{\rho}} \in f(P_{i}) \) i
\item \(\mathbf{d} = \mathbf{t}^{\rho}\).
\end{itemize}
\begin{minted}{coq}
Definition φ_pr
  (pr : production)
  (args : forall P : IndPredS Σ, vec D (indpred_ar P) -> Prop)
  (ds : vec D (indpred_ar (indcons pr)))
  : Prop :=
      exists (ρ : env M),
      (forall Q us, List.In (Q; us) (preds pr) ->
               interpP Q (V.map (eval ρ) us))
      /\
        ( forall P ts, List.In (P; ts) (indpreds pr) ->
                  args P (V.map (eval ρ) ts))
      /\
        ds = V.map (eval ρ) (indargs pr).
\end{minted}
\noindent Operator \(\varphi_{pr}\) je formalizacija ideje primjene produkcije.
Nadalje, potrebno je definirati operator koji će uzeti u obzir \textit{sve} produkcije koje se odnose na \(P\).
Definiramo \(\varphi_{P}(f)\) kao uniju svih \(\varphi_{pr^{\prime}}(f)\) gdje je \(pr^{\prime}\) produkcija u kojoj se \(P\) javlja u konkluziciji.
\begin{minted}{coq}
Definition φ_P
  (P : IndPredS Σ)
  (args : forall P : IndPredS Σ, vec D (indpred_ar P) -> Prop)
  : vec D (indpred_ar P) -> Prop.
  refine (fun ds => _).
  refine (@ex production (fun pr => _)).
  refine (@ex (P = indcons pr /\ Φ pr) (fun H => _)).
  destruct H as [Heq HΦ].
  rewrite Heq in ds.
  exact (φ_pr pr args ds).
Defined.
\end{minted}
\noindent Konačno, definiramo operator skupa definicija \(\varphi_{\Phi}\) kao preslikavanje
koje induktivnom predikatnom simbolu \(P\) pridružuje skup \(\varphi_{P}(f)\).
\begin{minted}{coq}
Definition φ_Φ
  (args : forall P : IndPredS Σ, vec D (indpred_ar P) -> Prop)
    : forall P : IndPredS Σ, vec D (indpred_ar P) -> Prop :=
    fun P => φ_P P args.
\end{minted}
\noindent Operator \(\varphi_{\Phi}\) omogućuje simultanu primjenu produkcija.
\end{definition}

\begin{remark}
  Kako je funkcija \(f\) bila uvedena na samom početku prethodne definicije,
  u stvari definicija operatora \(\varphi_{\Phi}\) glasi
  \(
    \varphi_{\Phi}(f)(P) = \varphi_{P}(f)
  \).
\end{remark}

\misao{Ova duga definicija vapi za primjerom. Meni je jasno, ali slutim da bi dragom čitatelju bilo incomprehensible.}

\begin{proposition}
  Operator \(\varphi_{\Phi}\) je monoton.
\begin{minted}{coq}
Proposition φ_Φ_monotone :
  forall (f g : forall P, vec D (indpred_ar P) -> Prop),
    (forall P v, f P v -> g P v) ->
    (forall P v, φ_Φ f P v -> φ_Φ g P v).
\end{minted}
\end{proposition}

\begin{definition}[Aproksimanti]
  Neka je \(M\) proizvoljna struktura.
  Definiramo aproksimaciju skupa induktivnih definicija \(\Phi\) razine \(\alpha\), u oznaci \(\varphi_{\Phi}^{\alpha}\), rekurzivno na način:
  \begin{enumerate}
  \item \( \varphi_{\Phi}^{0}(P) = \varnothing \) i
  \item \( \varphi_{\Phi}^{\alpha + 1} = \varphi_{\Phi}(\varphi_{\Phi}^{\alpha}) \).
  \end{enumerate}
\begin{minted}{coq}
Fixpoint φ_Φ_n P (α : nat) (v : vec M (indpred_ar P)) : Prop :=
  match α with
  | 0 => False
  | S α => @φ_Φ Σ M Φ (fun P => φ_Φ_n P α) P v
  end.
\end{minted}
  \noindent Tada je aproksimant induktivnog predikatnost simbola \(P\) razine \(\alpha\) upravo \(\varphi_{\Phi}^{\alpha}(P)\).
\begin{minted}{coq}
Definition approximant_of (P : IndPredS Σ)
  : nat -> vec M (indpred_ar P) -> Prop :=
  φ_Φ_n P.
\end{minted}
\end{definition}

\misao{Redoslijed argumenata u approximant\_of je zbunjujuć s obzirom na čitanje simbola: fi fi na alfa od pe.}
\misao{O redoslijedu smo na nekom sastanku raspravljali i zaključili da bi ipak trebao biti ovakav kakav jest.}

\begin{remark}
  Brotherston je definirao aproksimaciju razine \(\alpha\) kao uniju aproksimacija svih nižih razina.
  Takva je definicija ekvivalentna našoj.
\end{remark}
\misao{Malo je neprezicno reći ``unija'', ustvari se misli na uniju za svaki pojedini \(P\).}

\begin{lemma}
  Za sve prirodne brojeve \(\alpha\) i induktivne predikatne simbole \(P\) vrijedi
  \[
    \varphi_{\Phi}^{\alpha}(P) = \bigcup_{\beta < \alpha} \varphi_{\Phi}^{\beta}(P).
  \]
\begin{minted}{coq}
Lemma approximant_characterization : forall α P v,
      φ_Φ_n P α v <-> exists β,
                      β < α /\ @φ_Φ Σ M Φ (fun P => φ_Φ_n P β) P v.
\end{minted}
\end{lemma}

\begin{definition}
  Aproksimacija razine \(\omega\), u oznaci \(\varphi_{\Phi}^{\omega}\), je unija aproksimacija razina manjih od \(\omega\).
\begin{minted}{coq}
Definition φ_Φ_ω P v := exists α, φ_Φ_n P α v.
\end{minted}
\end{definition}
\misao{Opet, ``unija'' nije precizno.}

\begin{lemma}
  Aproksimacija razine \(\omega\) najmanji je skup sa svojstvom \(\varphi_{\Phi}(\varphi_{\Phi}^{\omega}) \subseteq \varphi_{\Phi}^{\omega}\).
\begin{minted}{coq}
Lemma ω_prefixed : forall P v, @φ_Φ Σ M Φ φ_Φ_ω P v -> φ_Φ_ω P v.
Lemma ω_least : forall args,
                (forall P v, @φ_Φ Σ M Φ args P v -> args P v) ->
                forall P v, φ_Φ_ω P v -> args P v.
\end{minted}
\end{lemma}

\begin{definition}[Standardni model]
  Kažemo da je struktura \(M\) standardni model za \(\Phi\) ako
  interpretira svaki induktivni predikatni simbol njegovim aproksimantom razine \(\omega\).
\begin{minted}{coq}
Definition standard_model
  (Σ : signature) (Φ: @IndDefSet Σ)
  : structure Σ -> Prop :=
  fun M =>
    forall (P : IndPredS Σ) ts, interpIP P ts <-> @φ_Φ_ω Σ M Φ P ts.
\end{minted}
\end{definition}
\noindent Standardni modeli su upravo one strukture koje imaju smisla za skup induktivnih definicija.


\section{Sistem sekvenata s induktivnim definicijama}\label{sec:sistem-sekvenata}
Kako to da je Brotherston definirao signaturu, ali onda nije uključio aksiome u LKID?
Primjerice, u \( \Sigma_{PA} \) imamo simbol jednakosti, ali ne možemo dokazati refleksivnost.
Valjda su aksiomi uključeni u \(\Gamma\), u smislu LKID je općenit dokazni sustav, pa bi neka
``aksiomatizirana teorija'' u stvari bili dokazi \(\mathit{akiomi} \vDash \varphi\). Možda.

LKID. Dopustiva pravila. Primjeri dokaza.

\section{Adekvatnost}\label{sec:adekvatnost}
Lokalne adekvatnosti za pravila izvoda. Glavni teorem.

%%% Local Variables:
%%% mode: LaTeX
%%% TeX-master: "master"
%%% End:
