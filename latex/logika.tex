\chapter{Logika prvog reda s induktivnim definicijama}\label{cha:logika-prvog-reda}
U ovom poglavlju predstavljamo glavne rezultate diplomskog rada koji uključuju formalizaciju
logike prvog reda s induktivnim definicijama \(\mathit{FOL_{ID}}\)
te dokaznog sustava \(\mathit{LKID}\), koje je prvi uveo Brotherston~\cite{brotherston2005}.
Definicije, leme i dokazi u ovom poglavlju preuzete su iz Brotherstonove disertacije~\cite{brotherstonphd}.

Prvo ćemo definirati sintaksu i semantiku logike \(\mathit{FOL_{ID}}\),
nakon čega ćemo definirati standardne modele te logike.
Zatim ćemo prikazati dokazni sustav \(\mathit{LKID}\) te konačno dokazati adekvatnost
sustava \(LKID\) s obzirom na standardnu semantiku,
što je ujedno i glavni rezultat ovog diplomskog rada.

Svaka definicija i lema u ovom poglavlju biti će popraćena svojim pandanom u Coqu.
Jedan je od ciljeva diplomskog rada prikazati primjene Coqa u matematici,
zbog čega leme nećemo dokazivati ``na papiru'',
već se dokaz svake leme može pronaći na GitHub repozitoriju rada.\footnote{TODO: REPO LINK}
Zainteresiranom čitatelju predlažemo interaktivni prolazak kroz dokaze lema.

\section{Sintaksa}\label{sec:sintaksa}
Kao i u svakom izlaganju logike, na početku je potrebno definirati sintaksu.
\begin{definition}
  \textit{Jezik prvog reda s induktivnim predikatima} (kratko: \textbf{signatura}), u oznaci \(\Sigma\),
  je skup simbola od kojih razlikujemo
  \begin{itemize}
  \item funkcijske simbole,
  \item \textit{obične} predikatne simbole i
  \item \textit{induktivne} predikatne simbole.
  \end{itemize}
\end{definition}
\begin{remark}
  Svaki simbol signature ima pripadajuću arnost.
  Funkcijski simboli arnosti nula nazivaju se \textit{konstante},
  a predikatni simboli arnosti nula nazivaju se \textit{propozicije}.
\end{remark}

Već smo vidjeli da se propozicije, odnosno skupovi, u Coqu mogu reprezentirati tipovima.
Tako i ovdje skupove funkcijskih, običnih predikatnih i induktivnih predikatnih
simbola prikazujemo tipovima.
Tada je arnost pojedinog simbola funkcija sa skupa odgovarajućih simbola u \(\mathbb{N}\).
\begin{minted}{coq}
Structure signature := {
    FuncS : Type;
    fun_ar : FuncS -> nat;
    PredS : Type;
    pred_ar : PredS -> nat;
    IndPredS : Type;
    indpred_ar : IndPredS -> nat
  }.
\end{minted}

\begin{definition}
  Term.
\end{definition}

\begin{definition}
  Formula.
\end{definition}

\begin{definition}
  Produkcija. Skup induktivnih definicija.
\end{definition}

\section{Semantika}\label{sec:semantika}
\begin{definition}
  Struktura.
\end{definition}

\begin{definition}
  Okolina, evaluacija.
\end{definition}

\begin{definition}
  Ispunjivost formule.
\end{definition}

\begin{lemma}
  Substitution sanity.
\end{lemma}

\section{Standardni modeli}\label{sec:standardni-modeli}
Operator \(\varphi_{\Phi}\). Aproksimanti. Standardni model.

\section{Sistem sekvenata s induktivnim definicijama}\label{sec:sistem-sekvenata}
LKID. Dopustiva pravila. Primjeri dokaza.

\section{Adekvatnost}\label{sec:adekvatnost}
Lokalne adekvatnosti za pravila izvoda. Glavni teorem.

%%% Local Variables:
%%% mode: LaTeX
%%% TeX-master: "master"
%%% End:
