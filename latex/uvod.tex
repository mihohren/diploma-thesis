\chapter{Uvod}\label{cha:uvod}
Coq je interaktivni dokazivač teorema u kojem
korisnik iskazuje matematičke tvrdnje tipovima te ih dokazuje programiranjem.
Teorija tipova koju Coq implementira temelji se na računu induktivnih konstrukcija
koji pak je proširenje \(\lambda\)-računa tipovima.
Iskazivanje tvrdnji moguće je zahvaljujući sorti \coqprop{} te zavisnim tipovima.
Zahvaljujući jeziku taktika moguće je i automatizirano dokazivanje nekih tvrdnji,
a osim za dokazivanje, Coq se može koristiti i za pisanje dokazano točnih programa.

Centralni pojam ovog rada je logika prvog reda s induktivnim definicijama
\(\mathit{FOL}_{\mathit{ID}}\), koju je prvi uveo James Brotherston, a u kojoj
se neki predikati definiraju na induktivan način produkcijama.
Zbog induktivnih predikata nužno je sa semantičkog stajališta promatrati
posebne vrste struktura u kojima interpretacije induktivnih predikata imaju smisla
s obzirom na njihove definicije .
Sa sintaksnog stajališta promatra se Gentzenov sistem sekvenata proširen
pravilima za induktivne predikate.

Ovaj rad podijeljen je na četiri poglavlja.
U prvom poglavlju prikazujemo osnovne Coqa kao programskog jezika te osnove njegove
teorijske pozadine, prvenstveno hijerarhije tipova i Curry--Howardove korespondencije.
U drugom poglavlju čitatelj se upoznaje s logikom \(\mathit{FOL}_{\mathit{ID}}\) kroz
pojmove formula, struktura te standardnih modela.
U trećem poglavlju definiramo dokazni sustav \(\mathit{LKID}\) za logiku \(\mathit{FOL}_{\mathit{ID}}\)
te dokazujemo neka njegova svojstva.
Sve definicije i tvrdnje u drugom i trećem poglavlju popraćene su svojom formalizacijom u Coqu.
Konačno, u četvrtom poglavlju ilustriramo pojam cikličkog dokaza primjerima u Coqu i u dokaznom
sustavu \(\mathit{CLKID}^{\omega}\).
Cijeli rad prožet je ilustrativnim primjerima, što neformalnih definicija i iskaza,
što njihovih formalnih reprezentacija u Coqu.



%%% Local Variables:
%%% mode: LaTeX
%%% TeX-master: "master"
%%% End:
