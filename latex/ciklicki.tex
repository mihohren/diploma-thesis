\chapter{Ciklički dokazi}\label{cha:cikliki-dokazi}
Svaki Coqov dokaz koji smo do sada prikazali je konačan term.
Svi formalni dokazi u sustavu \(\mathit{LKID}\) (uključujući primjere~\ref{ex:lkid-indr-proof} i~\ref{ex:lkid-indl-proof}) su konačni.
Ovakve dokaze jednim imenom nazivamo \textit{dobro utemeljenim dokazima} jer su sve \enquote{grane} njihovih dokaznih stabala konačne.
Međutim, možemo promatrati i \textit{neutemeljene dokaze} kod kojih postoje beskonačne grane.
\misao{\enquote{neutemeljen dokaz} zvuči kao \enquote{dokaz koji to nije}, treba neki bolji prijevod.}
Već smo u odjeljku~\ref{sec:programiranje-u-gallini} spomenuli neutemeljene, odnosno koinduktivne tipove.
Neutemeljeni dokaz je tada \enquote{beskonačan} dokaz neke koinduktivno definirane tvrdnje.
U ovom poglavlju ćemo u Coqu ilustrirati neutemeljene dokaze.

\section{Primjer: lijene liste}
Definirajmo za početak \textit{lijene liste}, to jest liste koje mogu biti konačne, a mogu biti i beskonačne.
Zbog potencijalne beskonačnosti koristimo koinduktivnu definiciju tipa.
\begin{minted}{coq}
CoInductive LList (A : Type) :=
| LNil : LList A
| LCons : A -> LList A -> LList A.
\end{minted}
\noindent Kažemo da je lijena lista \textit{beskonačna} ako je njen rep opet beskonačna lijena lista.
\begin{minted}{coq}
CoInductive Infinite {A} : LList A -> Prop :=
| Infinite_LCons : forall a l, Infinite l -> Infinite (LCons a l).
\end{minted}
\noindent Ovdje konstruktor \texttt{Infinite\_LCons} možemo shvatiti kao pravilo izvoda.
Pokušajmo dokazati da je intuitivno beskonačna lista i formalno beskonačna.
Definiramo funkciju \texttt{from} analogno istoimenoj funkciji iz odjeljka~\ref{sec:programiranje-u-gallini}.
\begin{minted}{coq}
CoFixpoint from (n : nat) : LList nat := LCons n (from (S n)).
\end{minted}
\noindent Intuitivno je jasno da je za svaki \texttt{n} lijena lista \texttt{from n}
beskonačna pa želimo dokazati tvrdnju \texttt{forall n, Infinite (from n)}.

Prije glavnog dokaza, pogledajmo konkretan primjer na tvrdnji \texttt{Infinite (from 0)}.
Po definiciji funkcije \texttt{from}, dovoljno je dokazati
tvrdnju \texttt{Infinite (LCons 0 (from 1))}, a primjenom pravila \texttt{Infinite\_LCons}
dolazimo do tvrdnje \texttt{Infinite (from 1)}.
Opet je po definiciji funkcije \texttt{from} dovoljno dokazati tvrdnju
\texttt{Infinite (LCons 1 (from 2))}, a ponovnom primjenom pravila \texttt{Infinite\_LCons}
dolazimo do tvrdnje \texttt{Infinite (from 2)}.
Ilustrirajmo ovaj postupak dokaznim stablom.
\begin{prooftree}
  \AxiomC{\vdots}
  \UnaryInfC{\texttt{Infinite (from 2)}}
  \RightLabel{(\texttt{Infinite\_LCons})}
  \UnaryInfC{\texttt{Infinite (LCons 1 (from 2))}}
  \UnaryInfC{\texttt{Infinite (from 1)}}
  \RightLabel{(\texttt{Infinite\_LCons})}
  \UnaryInfC{\texttt{Infinite (LCons 0 (from 1))}}
  \UnaryInfC{\texttt{Infinite (from 0)}}
\end{prooftree}
\noindent Sada je jasno da ćemo dokazati originalnu tvrdnju tek nakon beskonačno
mnogo primjena pravila \texttt{Infinite\_LCons}. Ipak, lakše je dokazati
tvrdnju \texttt{forall n, Infinite (from n)}.
Nakon introdukcije varijable \texttt{n}, odmatanja konstante \texttt{from n} i
primjene pravila \texttt{Infinite\_LCons}
dolazimo do tvrdnje \texttt{Infinite (from (S n))}.

\begin{prooftree}
  \AxiomC{\texttt{*}}
  \UnaryInfC{\texttt{Infinite (from (S n))}}
  \RightLabel{(\texttt{Infinite\_LCons})}
  \UnaryInfC{\texttt{Infinite (LCons n (from (S n)))}}
  \UnaryInfC{\texttt{Infinite (from n)}}
  \UnaryInfC{\texttt{forall n, Infinite (from n)}}
\end{prooftree}

\noindent Između tvrdnji \texttt{Infinite (from n)} i \texttt{Infinite (from (S n))}
\enquote{skinut} je prvi član beskonačne liste te je napravljen \textit{napredak}
u dokazu.
Bez napretka nema smisla ponavljati prethodni postupak beskonačno mnogo puta
jer se tvrdnja ne mijenja.
Uz napredak u dokazivanju ima smisla pozivati se na \textit{tvrdnju koja se dokazuje} te
takve dokaze nazivamo \textit{cikličkima}. Dokaz tvrdnje \texttt{forall n, Infinite (from n)}
jedan je primjer cikličkog dokaza, gdje je simbolom \(\mathtt{*}\) označeno cikličko \enquote{pozivanje} dokaza.
U Coqu, napredak je osiguran uvjetom produktivnosti kojeg smo spomenuli
u odjeljku~\ref{sec:ogranicenja}.

Pažljiv čitatelj će primijetiti da ista argumentacija vrijedi
i za predikat \texttt{Finite}, definiran na idući način.
\begin{minted}{coq}
Inductive Finite {A} : LList A -> Prop :=
| Finite_LNil : Finite LNil
| Finite_LCons : forall a l, Finite l -> Finite (LCons a l).
\end{minted}
\noindent Međutim, beskonačne liste po definiciji nisu konačne.
Zašto ipak gornji dokazi \textit{ne} prolaze za predikat \texttt{Finite}?
Riječ je o načinu definicije; predikat \texttt{Infinite} definiran je koinduktivno
te njegovi stanovnici \textit{nisu} dobro utemeljeni,
dok je predikat \texttt{Finite} definiran induktivno, a njegovi stanovnici \textit{jesu}
dobro utemeljeni. Štoviše, kada bismo predikat \texttt{Finite} definirali koinduktivno,
on bi postao trivijalan, to jest vrijedio bi za proizvoljnu lijenu listu.


Brotherston je formalizirao ideju cikličkih dokaza u dokaznom sustavu
\(\mathit{CLKID}^{\omega}\), podsustavu dokaznog sustava \(\mathit{LKID}^{\omega}\),
koji pak je proširenje sustava \(\mathit{LKID}\) potencijalno beskonačnim dokaznim stablima.
Dokazni sustav \(\mathit{CLKID}^{\omega}\) ograničen je u odnosu na sustav \(\mathit{LKID}^{\omega}\)
tako što svako (potencijalno beskonačno) dokazno stablo
smije imati \textit{najviše konačno mnogo} različitih podstabala te
svaka beskonačna grana tog stabla mora imati beskonačno mnogo \enquote{napredaka}.
% drugim riječima, beskonačno mnogo primjena induktivnih pravila




% bilo bi zanimljivo vidjeti vezu s LTL, prikazanu u Coq'Art
% možda dokazati neke teže "Exercises" od tamo

%%% Local Variables:
%%% mode: LaTeX
%%% TeX-master: "master"
%%% End:
