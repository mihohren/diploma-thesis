\chapter{Ciklički dokazi}\label{cha:cikliki-dokazi}
Svaki Coqov dokaz koji smo do sada prikazali je konačan term.
Svi formalni dokazi u sustavu \(\mathit{LKID}\) (uključujući primjere~\ref{ex:lkid-indr-proof} i~\ref{ex:lkid-indl-proof}) su konačni.
Ovakve dokaze jednim imenom nazivamo \textit{dobro utemeljenim dokazima} jer su sve \enquote{grane} njihovih dokaznih stabala konačne.
Međutim, možemo promatrati i \textit{neutemeljene dokaze} kod kojih postoje beskonačne grane.
\misao{\enquote{neutemeljen dokaz} zvuči kao \enquote{dokaz koji to nije}, treba neki bolji prijevod.}
Već smo u odjeljku~\ref{sec:programiranje-u-gallini} spomenuli neutemeljene, odnosno koinduktivne tipove.
Neutemeljeni dokaz je tada \enquote{beskonačan} dokaz neke koinduktivno definirane tvrdnje.
U ovom poglavlju ćemo u Coqu ilustrirati neutemeljene dokaze.

\section{Primjer: lijene liste}
Definirajmo za početak \textit{lijene liste}, to jest liste koje mogu biti konačne, a mogu biti i beskonačne.
Zbog potencijalne beskonačnosti koristimo koinduktivnu definiciju tipa.
\begin{minted}{coq}
CoInductive LList (A : Type) :=
| LNil : LList A
| LCons : A -> LList A -> LList A.
\end{minted}
\noindent Kažemo da je lijena lista \textit{beskonačna} ako je njen rep opet beskonačna lijena lista.
\begin{minted}{coq}
CoInductive Infinite {A} : LList A -> Prop :=
| Infinite_LCons : forall a l, Infinite l -> Infinite (LCons a l).
\end{minted}
\noindent Ovdje konstruktor \texttt{Infinite\_LCons} možemo shvatiti kao pravilo izvoda.
Pokušajmo dokazati da je intuitivno beskonačna lista nula formalno beskonačna.
Definiramo konstantu \texttt{zeros}.
\begin{minted}{coq}
CoFixpoint zeros : LList nat := LCons 0 zeros.
\end{minted}
\noindent Potrebno je dokazati tvrdnju \textit{Infinite zeros}.
Po definiciji konstante \texttt{zeros}, dovoljno je dokazati
tvrdnju \texttt{Inductive (LCons 0 zeros)}, no primjenom pravila \texttt{Infinite\_LCons}
opet dolazimo do tvrdnje \texttt{Infinite zeros}.
Međutim, uspjeli smo \enquote{skinuti} jednu nulu s početka niza.
Sada je dovoljno ponavljati ovaj postupak beskonačno mnogo puta i tvrdnja je dokazana.
Ilustrirajmo ovaj dokaz neformalnim stablom.
\begin{prooftree}
  \AxiomC{\texttt{Infinite zeros}}
  \RightLabel{(\texttt{Infinite\_LCons})}
  \UnaryInfC{\texttt{Infinite (LCons 0 zeros)}}
  \RightLabel{(Definicija konstante \texttt{zeros})}
  \UnaryInfC{\texttt{Infinite zeros}}
\end{prooftree}
\misao{Bilo bi super kada bih mogao kao Brotherston nacrtati strelicu od dna do vrha dokaza.}


% bilo bi zanimljivo vidjeti vezu s LTL ili CTL, mislim da takva postoji

%%% Local Variables:
%%% mode: LaTeX
%%% TeX-master: "master"
%%% End:
