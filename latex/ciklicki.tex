\chapter{Ciklički dokazi}\label{cha:cikliki-dokazi}
Svaki Coqov dokaz koji smo do sada prikazali je konačan term.
Svi formalni dokazi u sustavu \(\mathit{LKID}\) (uključujući primjere~\ref{ex:lkid-indr-proof} i~\ref{ex:lkid-indl-proof}) su konačni.
Ovakve dokaze jednim imenom nazivamo \textit{dobro utemeljenim dokazima} jer su sve \enquote{grane} njihovih dokaznih stabala konačne.
Međutim, možemo promatrati i \textit{neutemeljene dokaze} kod kojih postoje beskonačne grane.
Već smo u odjeljku~\ref{sec:programiranje-u-gallini} spomenuli neutemeljene, odnosno koinduktivne tipove.
Neutemeljeni dokaz je tada \enquote{beskonačan} dokaz neke koinduktivno definirane tvrdnje.
U ovom ćemo kratkom poglavlju ilustrirati neutemeljene dokaze te cikličke dokaze koji su posebna vrsta neutemeljenih.

\section{Primjer: lijene liste}
Definirajmo za početak \textit{lijene liste}, to jest liste koje mogu biti konačne, a mogu biti i beskonačne.
Zbog potencijalne beskonačnosti koristimo koinduktivnu definiciju tipa.
\begin{minted}{coq}
CoInductive LList (A : Type) :=
| LNil : LList A
| LCons : A -> LList A -> LList A.
\end{minted}
\noindent Kažemo da je lijena lista \textit{beskonačna} ako je njen rep opet beskonačna lijena lista.
\begin{minted}{coq}
CoInductive Infinite {A} : LList A -> Prop :=
| Infinite_LCons : forall a l, Infinite l -> Infinite (LCons a l).
\end{minted}
\noindent Ovdje konstruktor \texttt{Infinite\_LCons} možemo shvatiti kao pravilo izvoda.
Pokušajmo dokazati da je intuitivno beskonačna lista i formalno beskonačna.
Definiramo funkciju \texttt{from} analogno istoimenoj funkciji iz odjeljka~\ref{sec:programiranje-u-gallini}.
\begin{minted}{coq}
CoFixpoint from (n : nat) : LList nat := LCons n (from (S n)).
\end{minted}
\noindent Intuitivno je jasno da je za svaki \texttt{n} lijena lista \texttt{from n}
beskonačna pa želimo dokazati tvrdnju \texttt{forall n, Infinite (from n)}.

Prije glavnog dokaza, pogledajmo konkretan primjer na tvrdnji \texttt{Infinite (from 0)}.
Po definiciji funkcije \texttt{from}, dovoljno je dokazati
tvrdnju \texttt{Infinite (LCons 0 (from 1))}, a primjenom pravila \texttt{Infinite\_LCons}
dolazimo do tvrdnje \texttt{Infinite (from 1)}.
Opet je po definiciji funkcije \texttt{from} dovoljno dokazati tvrdnju
\texttt{Infinite (LCons 1 (from 2))}, a ponovnom primjenom pravila \texttt{Infinite\_LCons}
dolazimo do tvrdnje \texttt{Infinite (from 2)}.
Ilustrirajmo ovaj postupak dokaznim stablom.
\begin{prooftree}
  \AxiomC{\vdots}
  \UnaryInfC{\texttt{Infinite (from 2)}}
  \RightLabel{(\texttt{Infinite\_LCons})}
  \UnaryInfC{\texttt{Infinite (LCons 1 (from 2))}}
  \UnaryInfC{\texttt{Infinite (from 1)}}
  \RightLabel{(\texttt{Infinite\_LCons})}
  \UnaryInfC{\texttt{Infinite (LCons 0 (from 1))}}
  \UnaryInfC{\texttt{Infinite (from 0)}}
\end{prooftree}
\noindent Sada je jasno da ćemo dokazati originalnu tvrdnju tek nakon beskonačno
mnogo primjena pravila \texttt{Infinite\_LCons}. Ipak, lakše je dokazati
tvrdnju \texttt{forall n, Infinite (from n)}.
Nakon introdukcije varijable \texttt{n}, odmatanja konstante \texttt{from n} i
primjene pravila \texttt{Infinite\_LCons}
dolazimo do tvrdnje \texttt{Infinite (from (S n))}.

\begin{prooftree}
  \AxiomC{\texttt{*}}
  \UnaryInfC{\texttt{Infinite (from (S n))}}
  \RightLabel{(\texttt{Infinite\_LCons})}
  \UnaryInfC{\texttt{Infinite (LCons n (from (S n)))}}
  \UnaryInfC{\texttt{Infinite (from n)}}
  \UnaryInfC{\texttt{forall n, Infinite (from n)}}
\end{prooftree}

\noindent Između tvrdnji \texttt{Infinite (from n)} i \texttt{Infinite (from (S n))}
\enquote{skinut} je prvi član beskonačne liste te je napravljen \textit{napredak}
u dokazu.
Bez napretka nema smisla ponavljati prethodni postupak beskonačno mnogo puta
jer se tvrdnja ne mijenja.
Uz napredak u dokazivanju ima smisla pozivati se na \textit{tvrdnju koja se dokazuje} te
takve dokaze nazivamo \textit{cikličkima}. Dokaz tvrdnje \texttt{forall n, Infinite (from n)}
jedan je primjer cikličkog dokaza, gdje je simbolom \(\mathtt{*}\) označeno cikličko \enquote{pozivanje} dokaza.
U Coqu, napredak je osiguran uvjetom produktivnosti kojeg smo spomenuli
u odjeljku~\ref{sec:ogranicenja}.

Pažljiv čitatelj će primijetiti da ista argumentacija vrijedi
i za predikat \texttt{Finite}, definiran na idući način.
\begin{minted}{coq}
Inductive Finite {A} : LList A -> Prop :=
| Finite_LNil : Finite LNil
| Finite_LCons : forall a l, Finite l -> Finite (LCons a l).
\end{minted}
\noindent Međutim, beskonačne liste po definiciji nisu konačne.
Zašto ipak gornji dokazi \textit{ne} prolaze za predikat \texttt{Finite}?
Riječ je o načinu definicije; predikat \texttt{Infinite} definiran je koinduktivno
te njegovi stanovnici \textit{nisu} dobro utemeljeni,
dok je predikat \texttt{Finite} definiran induktivno, a njegovi stanovnici \textit{jesu}
dobro utemeljeni. Štoviše, kada bismo predikat \texttt{Finite} definirali koinduktivno,
on bi postao trivijalan, to jest vrijedio bi za proizvoljnu lijenu listu.

\section{Primjer: \(\mathit{CLKID}^{\omega}\)}
Brotherston je formalizirao ideju cikličkih dokaza u dokaznom sustavu
\(\mathit{CLKID}^{\omega}\), podsustavu dokaznog sustava \(\mathit{LKID}^{\omega}\),
koji pak je proširenje sustava \(\mathit{LKID}\) potencijalno beskonačnim dokaznim stablima.
Dokazni sustav \(\mathit{CLKID}^{\omega}\) ograničen je u odnosu na sustav \(\mathit{LKID}^{\omega}\)
tako što svako (potencijalno beskonačno) dokazno stablo
smije imati \textit{najviše konačno mnogo} različitih podstabala te
svaka beskonačna grana tog stabla mora imati beskonačno mnogo \enquote{napredaka}.
Dokazni sustav \(\mathit{CLKID}^{\omega}\) prikazati ćemo neformalno.

\begin{example}
  Dokazujemo tvrdnju \(\forall x, \mathit{Nat}(x) \rightarrow \mathit{Even}(x) \lor \mathit{Odd}(x)\)
  u sustavu \(\mathit{CLKID}^{\omega}\).
  Koristimo pokrate kao u primjeru~\ref{ex:lkid-indl-proof}..
  \begin{prooftree}\label{ex:clkidw1}
    \AxiomC{}
    \RightLabel{(\texttt{PA\_prod\_E\_zero}\textit{IndR})}
    \UnaryInfC{\(\vdash Eo, Oo\)}
    \AxiomC{\(Nx \vdash Ex, Ox \, (\dagger)\)}
    \RightLabel{\( \mathit{(Subst)} \)}
    \UnaryInfC{\(Ny \vdash Ey, Oy\)}
    \RightLabel{\(\mathit{(Perm)}\)}
    \UnaryInfC{\( Ny \vdash Oy, Ey \)}
    \RightLabel{(\texttt{PA\_prod\_O\_succ}\textit{IndR})}
    \UnaryInfC{\( Ny \vdash Oy, Osy \)}
    \RightLabel{(\texttt{PA\_prod\_E\_succ}\textit{IndR})}
    \UnaryInfC{\(Ny \vdash Esy, Osy\)}
    \RightLabel{\(\mathit{(EqL)}\)}
    \UnaryInfC{\(x = sy, Ny \vdash Ex, Ox\)}
    \RightLabel{\(\mathit{(Case \, N)}\)}    
    \BinaryInfC{\( Nx \vdash Ex, Ox \, (\dagger) \)}
    \RightLabel{\(\mathit{(OrR)}\)}
    \UnaryInfC{\( Nx \vdash Ex \lor Ox \)}
    \RightLabel{\(\mathit{(ImpR)}\)}
    \UnaryInfC{\(\vdash Nx \rightarrow Ex \lor Ox\)}
    \RightLabel{\(\mathit{(AllR)}\)}
    \UnaryInfC{\(\vdash \forall x, Nx \rightarrow Ex \lor Ox\)}
  \end{prooftree}
  Iako zapisan u konačnom stablu, ovaj dokaz je beskonačan.
  Čitajući dokaz od dna do vrha, primjećujemo nove oznake te nova pravila.
  Za početak, vidimo da su znakom \(\dagger\) označene gornja i donja sekventa \(Nx \vdash Ex, Ox\).
  Gornju sekventu nazivamo \textit{pupoljkom} (\textit{bud}), a donju nazivamo \textit{pratilcem} (\textit{companion}).
  Može se reći da su pupoljak i njegov pratilac povezani bridom, tvoreći ciklus u dokaznom
  stablu --- odavde dolazi naziv \textit{ciklički} dokaz.
  U stvari se isječak stabla dokaza između pratilca i pupoljka ponavlja \textit{ad infinitum}
  pa je dokaz beskonačan.
  Nadalje, vidimo novo pravilo izvoda \(\mathit{Case}\) koje odgovara rastavu po slučajevima,
  a koristi se umjesto pravila indukcije.
  Ako se između pupoljka i njegovog pratilca javlja primjena pravila \(\mathit{Case}\),
  kažemo da je napravljen \textit{napredak}.
  Konačno, vidimo znak jednakosti uz kojeg je vezano lijevo pravilo za jednakost \(\mathit{EqL}\).
  Znak jednakosti je za potrebe sustava \(\mathit{LKID}^{\omega}\) i \(\mathit{CLKID}^{\omega}\)
  (odnosno, \(\mathit{Case}\) pravila) ugrađen u definiciju formule.\footnote{Jednakost se u matematičkoj logici najčešće tretira kao predikatni simbol signature, a ne kao primitivni veznik u formulama.}
\end{example}

U primjeru~\ref{ex:clkidw1}, pupoljak i njegov pratilac javljaju se u istoj grani dokaza,
no to nije nužno. Ovu pojavu možemo vidjeti na dokazu obrata sekvente \(Nx \vdash Ex \lor Ox\).


\begin{example}
  Dokazujemo sekventu \(Ex \lor Ox \vdash Nx \) u sustavu \(\mathit{CLKID}^{\omega}\).
  \begin{prooftree}
    \AxiomC{}
    \RightLabel{\((1)\)}
    % \RightLabel{(\texttt{PA\_prod\_N\_zero}\textit{IndR})}
    \UnaryInfC{\(\vdash No\)}
    \RightLabel{\(\mathit{(EqL)}\)}
    \UnaryInfC{\(x = o \vdash Nx\)}
    \AxiomC{\(Ox \vdash Nx \, (\dagger)\)}
    \RightLabel{\(\mathit{(Subst)}\)}
    \UnaryInfC{\(Oy \vdash Ny\)}
    % \RightLabel{(\texttt{PA\_prod\_N\_succ}\textit{IndR})}
    \RightLabel{\((2)\)}
    \UnaryInfC{\(Oy \vdash Nsy\)}
    \RightLabel{\(\mathit{(EqL)}\)}
    \UnaryInfC{\(x = sy, Oy \vdash Nx \)}
    \RightLabel{\(\mathit{(Case \, E)}\)}
    \BinaryInfC{\(Ex \vdash Nx \, (\ast)\)}
    \AxiomC{\(Ex \vdash Nx \, (\ast)\)}
    \RightLabel{\(\mathit{(Subst)}\)}
    \UnaryInfC{\(Ey \vdash Ny\)}
    % \RightLabel{(\texttt{PA\_prod\_N\_succ}\textit{IndR})}
    \RightLabel{\((3)\)}
    \UnaryInfC{\(Ey \vdash Nsy\)}
    \RightLabel{\(\mathit{(EqL)}\)}
    \UnaryInfC{\(x = sy, Ey \vdash Nx\)}
    \RightLabel{\((\mathit{Case \, O})\)}
    \UnaryInfC{\( Ox \vdash Nx \, (\dagger) \)}
    \RightLabel{\(\mathit{(OrL)}\)}
    \BinaryInfC{\( Ex \lor Ox \vdash Nx \)}
  \end{prooftree}
  Ovdje je znakom \(\dagger\) označen par pupoljka i pratilca sekvente \(Ox \vdash Nx\),
  a znakom \(\ast\) je označen par pupoljka i pratilca sekvente \(Ex \vdash Nx\).
  Iako se odgovarajući parovi pupoljaka i pratilaca naizgled javljaju u različitim granama,
  uz dva \enquote{odmatanja} stabla dokaza vidimo da to u stvari nije slučaj.
  \begin{scriptsize}
    \begin{prooftree}
      \AxiomC{}
      \RightLabel{\((1)\)}
      % \RightLabel{(\texttt{PA\_prod\_N\_zero}\textit{IndR})}
      \UnaryInfC{\(\vdash No\)}
      \RightLabel{\(\mathit{(EqL)}\)}
      \UnaryInfC{\(x = o \vdash Nx\)}

      \AxiomC{\(Ex \vdash Nx \, (\ast)\)}
      \RightLabel{\(\mathit{(Subst)}\)}
      \UnaryInfC{\(Ey \vdash Ny\)}
      % \RightLabel{(\texttt{PA\_prod\_N\_succ}\textit{IndR})}
      \RightLabel{\((3)\)}
      \UnaryInfC{\(Ey \vdash Nsy\)}
      \RightLabel{\(\mathit{(EqL)}\)}
      \UnaryInfC{\(x = sy, Ey \vdash Nx\)}
      \RightLabel{\((\mathit{Case \, O})\)}
      \UnaryInfC{\( Ox \vdash Nx \, \xcancel{(\dagger)}\)}
      
      \RightLabel{\(\mathit{(Subst)}\)}
      \UnaryInfC{\(Oy \vdash Ny\)}
      % \RightLabel{(\texttt{PA\_prod\_N\_succ}\textit{IndR})}
      \RightLabel{\((2)\)}
      \UnaryInfC{\(Oy \vdash Nsy\)}
      \RightLabel{\(\mathit{(EqL)}\)}
      \UnaryInfC{\(x = sy, Oy \vdash Nx \)}
      \RightLabel{\(\mathit{(Case \, E)}\)}
      \BinaryInfC{\(Ex \vdash Nx \, (\ast)\)}

      \AxiomC{}
      \RightLabel{\((1)\)}
      % \RightLabel{(\texttt{PA\_prod\_N\_zero}\textit{IndR})}
      \UnaryInfC{\(\vdash No\)}
      \RightLabel{\(\mathit{(EqL)}\)}
      \UnaryInfC{\(x = o \vdash Nx\)}
      \AxiomC{\(Ox \vdash Nx \, (\dagger)\)}
      \RightLabel{\(\mathit{(Subst)}\)}
      \UnaryInfC{\(Oy \vdash Ny\)}
      % \RightLabel{(\texttt{PA\_prod\_N\_succ}\textit{IndR})}
      \RightLabel{\((2)\)}
      \UnaryInfC{\(Oy \vdash Nsy\)}
      \RightLabel{\(\mathit{(EqL)}\)}
      \UnaryInfC{\(x = sy, Oy \vdash Nx \)}
      \RightLabel{\(\mathit{(Case \, E)}\)}
      
      \BinaryInfC{\(Ex \vdash Nx \, \xcancel{(\ast)}\)}
      \RightLabel{\(\mathit{(Subst)}\)}
      \UnaryInfC{\(Ey \vdash Ny\)}
      % \RightLabel{(\texttt{PA\_prod\_N\_succ}\textit{IndR})}
      \RightLabel{\((3)\)}
      \UnaryInfC{\(Ey \vdash Nsy\)}
      \RightLabel{\(\mathit{(EqL)}\)}
      \UnaryInfC{\(x = sy, Ey \vdash Nx\)}
      \RightLabel{\((\mathit{Case \, O})\)}
      \UnaryInfC{\( Ox \vdash Nx \, (\dagger) \)}
      \RightLabel{\(\mathit{(OrL)}\)}
      \BinaryInfC{\( Ex \lor Ox \vdash Nx \)}
    \end{prooftree}
  \end{scriptsize}
  Pupoljak sekvente \(Ex \vdash Nx\) je zamijenjen
  isječkom stabla od pratilca sekvente \(Ex \vdash Nx\) do pupoljka sekvente \(Ox \vdash Nx\),
  Zatim smo odmotali stablo pri pupoljku sekvente \(Ox \vdash Nx\) zamijenivši ga
  isječkom stabla od pratilca sekvente \(Ox \vdash Nx\) do pupoljka sekvente \(Ex \vdash Nx\).
  \misao{Ako je moguće, tu bi bilo super koristiti boje teksta.}
\end{example}

%%% Local Variables:
%%% mode: LaTeX
%%% TeX-master: "master"
%%% End:
