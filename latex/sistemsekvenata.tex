\chapter{Sistem sekvenata s induktivnim definicijama}\label{sec:sistem-sekvenata}
Cilj je ovog odjeljka definirati dokazni sustav \(\mathit{LKID}\) za logiku \(\mathit{FOL_{ID}}\).
Ovaj dokazni sustav temelji se na Gentzenovu računu sekvenata, a proširen je lijevim i desnim pravilima za induktivne predikate.
Prije no što definiramo \(\mathit{LKID}\), potrebno je definirati pojam sekvente i međusobne zavisnosti induktivnih predikata.

\begin{definition}
  Neka su \(\Gamma\) i \(\Delta\) proizvoljne liste formula.
  Tada je \textit{sekventa} uređeni par \((\Gamma,\Delta)\), a označavamo ju s \(\Gamma \vdash \Delta\).
\begin{minted}{coq}
Inductive sequent : Set :=
| mkSeq (Γ Δ : list (formula Σ)).
\end{minted}
\end{definition}
\noindent Sekventom \(\Gamma \vdash \Delta\) konceptualno tvrdimo da istinitost \textit{svake}
tvrdnje u \(\Gamma\) povlači istinitost \textit{neke} tvrdnje u \(\Delta\).
Kažemo da je sekventa \(\Gamma \vdash \Delta\) \textit{dokaziva}, ili da \(\Gamma\) \textit{izvodi} \(\Delta\), ako u sustavu \(\mathit{LKID}\) postoji \textit{dokaz} za \(\Gamma \vdash \Delta\).
Neformalno možemo reći da je dokaz sekvente \(\Gamma \vdash \Delta\) konačno stablo u kojem je sekventa \(\Gamma \vdash \Delta\) korijen
te svaki čvor stabla slijedi iz svoje djece u skladu s pravilima izvoda sustava \(\mathit{LKID}\).
Kasnije ćemo precizirati pojam dokaza.

\begin{definition}
  Neka su \(P_{i}\) i \(P_{j}\) proizvoljni induktivni predikatni simboli.
  Kažemo da je simbol \(P_{i}\) u relaciji \(\mathit{Prem}\) sa simbolom \(P_{j}\)
  ako postoji produkcija u skupu induktivnih definicija \(\Phi\) takva da
  se simbol \(P_{i}\) javlja u njenoj konkluziji te
  se simbol \(P_{j}\) javlja u njenim premisama.
\begin{minted}{coq}
Definition Prem (Pi Pj : IndPredS Σ) :=
  exists pr, Φ pr /\
             indcons pr = Pi /\
             exists ts, List.In (Pj; ts) (indpreds pr).
\end{minted}
\end{definition}

\begin{definition}
  Definiramo relaciju \(\mathit{Prem}^{\ast}\) kao refleksivno i tranzitivno zatvorenje relacije \(\mathit{Prem}\) na skupu induktivnih predikatnih simbola.
\begin{minted}{coq}
Definition Prem_star := clos_refl_trans (IndPredS Σ) Prem.
\end{minted}
  \noindent Za induktivne predikatne simbole \(P\) i \(Q\) kažemo da su
  \textit{međusobno zavisni} ako vrijedi \(\mathit{Prem}^{\ast}(P, Q)\) i \(\mathit{Prem}^{\ast}(Q, P)\).
\begin{minted}{coq}
Definition mutually_dependent (P Q : IndredS Σ) :=
  Prem_star P Q /\ Prem_star Q P.
\end{minted}
  \noindent Za međusobno zavisne induktivne predikatne simbole kažemo još da su
  definirani simultanom rekurzijom.
  Simultana rekurzija se koristi za formalnu definiciju međusobno zavisnih induktivnih predikata
  i u mnogo složenijim sustavima od \(\mathit{FOL}_{\mathit{ID}}\); primjerice,
  Coqovi predikati \texttt{EVEN} i \texttt{ODD} iz primjera~\ref{ex:structure}
  definirani su simultanom rekurzijom.
\end{definition}

\begin{example}
  U skupu induktivnih definicija \(\Phi_{\mathit{PA}}\),
  induktivni predikatni simboli \(\mathit{Even}\) i \(\mathit{Odd}\)
  su međusobno zavisni.
\begin{minted}{coq}
Example mut_dep_E_O :
  @mutually_dependent Σ__PA Φ__PA PA_Even PA_Odd.
\end{minted}
  \noindent S druge strane, induktivni predikatni simboli \(\mathit{Nat}\) i \(\mathit{Even}\)
  \textit{nisu} međusobno zavisni.
\begin{minted}{coq}
Example not_mut_dep_N_E :
  ~ @mutually_dependent Σ__PA Φ__PA PA_Nat PA_Even.
\end{minted}
\end{example}

\begin{lemma}
  Međusobna zavisnost je relacija ekvivalencije na skupu induktivnih predikatnih simbola.
\begin{minted}{coq}
Lemma mutually_dependent_equiv : Equivalence mutually_dependent.
\end{minted}
\end{lemma}

Sustav \(\mathit{LKID}\) sastoji se od četiri vrste pravila izvoda:
strukturalna, propozicijska, pravila za kvantifikatore i pravila za induktivne predikate.
Navest ćemo ih tim redom, a definirati kroz više blokova Coqova koda.
Napominjemo da se sustav \(\mathit{LKID}\) razlikuje od Gentzenova sustava \(\mathit{LK}\)
na nekoliko mjesta, a zbog implementacijskih detalja te zbog induktivnih definicija.
Za neko pravilo izvoda kažemo da je \textit{dopustivo} ako nije u definiciji sustava \(\mathit{LKID}\),
ali je iz originalnih pravila izvedivo.
\begin{minted}{coq}
Inductive LKID : sequent -> Prop := (* nastavlja se *)
\end{minted}
\section{Strukturalna pravila}
Strukturalna su pravila prikazana na slici~\ref{fig:lkid-structural}, a služe baratanju strukturom sekvente.
Iako je sekventa implementirana kao par listi formula, te liste se ponašaju kao skupovi u strukturalnim pravilima
jer za sadržanost jedne liste u drugoj koristimo Coqov predikat \texttt{incl} koji se oslanja na predikat \texttt{In}.
Posljedično, pravilo slabljenja \(\mathit{Wk}\) povlači pravila permutacije i kontrakcije.
\begin{figure}[!htb]
  \centering

  \begin{prooftree}
    \AxiomC{\(\Gamma \cap \Delta \not = \varnothing \)}
    \RightLabel{\( \mathit{ (Ax) } \)}
    \UnaryInfC{\( \Gamma \vdash \Delta \)}
  \end{prooftree}

  \begin{prooftree}
    \AxiomC{\(\Gamma^{\prime} \vdash \Delta^{\prime}\)}
    \AxiomC{\(\Gamma^{\prime} \subseteq \Gamma\)}
    \AxiomC{\(\Delta^{\prime} \subseteq \Delta\)}
    \RightLabel{\( \mathit{ (Wk) } \)}
    \TrinaryInfC{\(\Gamma \subseteq \Delta\)}
  \end{prooftree}
  
  \begin{prooftree}
    \AxiomC{\( \Gamma \vdash \varphi, \Delta\)}
    \AxiomC{\( \varphi, \Gamma \vdash \Delta \)}
    \RightLabel{\( \mathit{ (Cut) } \)}
    \BinaryInfC{\( \Gamma \vdash \Delta \)}
  \end{prooftree}

  \begin{prooftree}
    \AxiomC{\( \Gamma \vdash \Delta \)}
    \RightLabel{\( \mathit{ (Subst) } \)}
    \UnaryInfC{\( \Gamma[\sigma] \vdash \Delta[\sigma] \)}
  \end{prooftree}
  
  \caption{Strukturalna pravila sustava \(\mathit{LKID}\).}\label{fig:lkid-structural}
\end{figure}
\begin{minted}[firstnumber=last]{coq}
(* Structural rules. *)
| Ax : forall Γ Δ φ, In φ Γ -> In φ Δ -> LKID (Γ ⊢ Δ)
| Wk : forall Γ' Δ' Γ Δ,
    Γ' ⊆ Γ ->
    Δ' ⊆ Δ ->
    LKID (Γ' ⊢ Δ') ->
    LKID (Γ ⊢ Δ)
| Cut : forall Γ Δ φ,
    LKID (Γ ⊢ φ :: Δ) ->
    LKID (φ :: Γ ⊢ Δ) ->
    LKID (Γ ⊢ Δ)
| Subst : forall Γ Δ,
    LKID (Γ ⊢ Δ) ->
    forall σ, LKID (map (subst_formula σ) Γ ⊢ map (subst_formula σ) Δ)
\end{minted}

\begin{example}
  Prošireno pravilo aksioma je dopustivo.
  \begin{prooftree}
    \AxiomC{}
    \RightLabel{\( \mathit{ (AxExtended) } \)}
    \UnaryInfC{\( \varphi, \Gamma \vdash \varphi, \Delta \)}
  \end{prooftree}
\begin{minted}[linenos=false]{coq}
Lemma AxExtended : forall Γ Δ φ, LKID (φ :: Γ ⊢ φ :: Δ).
\end{minted}
  \noindent Nadalje, pravilo permutacije je dopustivo.
  \begin{prooftree}
    \AxiomC{\( \Gamma^{\prime} \vdash \Delta^{\prime} \)}
    \AxiomC{\( \Gamma^{\prime} \sim \Gamma \)}
    \AxiomC{\( \Delta^{\prime} \sim \Delta \)}
    \RightLabel{\(\mathit{ (Perm) }\)}
    \TrinaryInfC{\( \Gamma \vdash \Delta \)}
  \end{prooftree}
  \noindent Ovdje znakom \(\sim\) označujemo relaciju permutacije na listama.
\begin{minted}[linenos=false]{coq}
Lemma Perm : forall Γ' Δ' Γ Δ,
    Permutation Γ' Γ ->
    Permutation Δ' Δ ->
    LKID (Γ' ⊢ Δ') ->
    LKID (Γ ⊢ Δ).
\end{minted}
  \noindent Konačno, lijeva i desna pravila kontrakcije su dopustiva.
  \begin{prooftree}
    \AxiomC{\( \varphi, \varphi, \Gamma \vdash \Delta \)}
    \RightLabel{\(\mathit{(ContrL)}\)}
    \UnaryInfC{\( \varphi, \Gamma \vdash \Delta \)}
  \end{prooftree}
  \begin{prooftree}
    \AxiomC{\( \Gamma \vdash \varphi, \varphi, \Delta \)}
    \RightLabel{\(\mathit{(ContrR)}\)}
    \UnaryInfC{\( \Gamma \vdash \varphi, \Delta \)}
  \end{prooftree}
\begin{minted}[linenos=false]{coq}
Lemma ContrL : forall Γ Δ φ,
    LKID (φ :: φ :: Γ ⊢ Δ) -> LKID (φ :: Γ ⊢ Δ).
Lemma ContrR : forall Γ Δ φ,
    LKID (Γ ⊢ φ :: φ :: Δ) -> LKID (Γ ⊢ φ :: Δ).
\end{minted}
\end{example}

\section{Propozicijska pravila}
Propozicijska su pravila prikazana na slici~\ref{fig:lkid-propositional}, a služe rasuđivanju s implikacijama i negacijama.
Ova su pravila identična u sustavima \(\mathit{LK}\) i \(\mathit{LKID}\).
\begin{figure}[!htb]
  \centering
  \begin{prooftree}
    \AxiomC{\( \Gamma \vdash \varphi, \Delta \)}
    \RightLabel{\( \mathit{ (NegL) } \)}
    \UnaryInfC{\( \neg \varphi, \Gamma \vdash \Delta \)}
  \end{prooftree}

  \begin{prooftree}
    \AxiomC{\( \varphi, \Gamma \vdash \Delta \)}
    \RightLabel{\( \mathit{ (NegR) } \)}
    \UnaryInfC{\( \Gamma \vdash \neg \varphi, \Delta \)}
  \end{prooftree}

  \begin{prooftree}
    \AxiomC{\( \Gamma \vdash \varphi, \Delta \)}
    \AxiomC{\( \psi, \Gamma \vdash \Delta \)}
    \RightLabel{\( \mathit{ (ImpL) } \)}
    \BinaryInfC{\( \varphi \rightarrow \psi, \Gamma \vdash \Delta \)}
  \end{prooftree}

  \begin{prooftree}
    \AxiomC{\( \varphi, \Gamma \vdash \psi, \Delta \)}
    \RightLabel{\( \mathit{ (ImpR) } \)}
    \UnaryInfC{\( \Gamma \vdash \varphi \rightarrow \psi, \Delta \)}
  \end{prooftree}
  
  \caption{Propozicijska pravila sustava \(\mathit{LKID}\).}\label{fig:lkid-propositional}
\end{figure}
\begin{minted}[firstnumber=16]{coq}
(* Propositional rules. *)
| NegL : forall Γ Δ φ, LKID (Γ ⊢ φ :: Δ) -> LKID (FNeg φ :: Γ ⊢ Δ)
| NegR : forall Γ Δ φ, LKID (φ :: Γ ⊢ Δ) -> LKID (Γ ⊢ FNeg φ :: Δ)
| ImpL : forall Γ Δ φ ψ,
    LKID (Γ ⊢ φ :: Δ) -> LKID (ψ :: Γ ⊢ Δ) ->
    LKID (FImp φ ψ :: Γ ⊢ Δ)
| ImpR : forall Γ Δ φ ψ,
    LKID (φ :: Γ ⊢ ψ :: Δ) -> LKID (Γ ⊢ (FImp φ ψ) :: Δ)
\end{minted}

\begin{example}
  Desno pravilo za konjunkciju je dopustivo.
  \begin{prooftree}
    \AxiomC{\( \Gamma \vdash \varphi, \Delta \)}
    \AxiomC{\( \Gamma \vdash \psi, \Delta \)}
    \RightLabel{\( \mathit{ (AndR) } \)}
    \BinaryInfC{\( \Gamma \vdash \varphi \land \psi, \Delta \)}
  \end{prooftree}
\begin{minted}[linenos=false]{coq}
Lemma AndR : forall Γ Δ φ ψ,
    LKID (Γ ⊢ φ :: Δ) -> LKID (Γ ⊢ ψ :: Δ) ->
    LKID (Γ ⊢ FAnd φ ψ :: Δ).
\end{minted}
  \noindent Lijevo pravilo za disjunkciju je dopustivo.
  \begin{prooftree}
    \AxiomC{\( \varphi, \Gamma \vdash \Delta \)}
    \AxiomC{\( \psi, \Gamma \vdash \Delta \)}
    \RightLabel{\( \mathit{ (OrL) } \)}
    \BinaryInfC{\( \varphi \lor \psi, \Gamma \vdash \Delta \)}
  \end{prooftree}
\begin{minted}[linenos=false]{coq}
Lemma OrL : forall Γ Δ φ ψ,
    LKID (φ :: Γ ⊢ Δ) -> LKID (ψ :: Γ ⊢ Δ) ->
    LKID (FOr φ ψ :: Γ ⊢ Δ).
\end{minted}
\end{example}

\section{Pravila za kvantifikatore}
Pravila za kvantifikatore prikazana su na slici~\ref{fig:lkid-quantificational}, a odnose se na univerzalnu kvantifikaciju.
Ova se pravila razlikuju od analognih u sustavu \(\mathit{LK}\) zbog korištenja de Bruijnovih indeksa kod kvantifikacije.
U lijevom pravilu koristimo supstituciju \(t \cdot \sigma_{\mathit{id}}\) u formuli \(\varphi\), gdje je \(\sigma_{\mathit{id}}\) supstitucija identiteta,
kako bismo iskazali da se term \(t\) javlja na mjestu neke varijable \(x\).
Za desno pravilo, primijetimo da se varijabla \(0\) ne javlja u listama formula \(\Gamma^{\uparrow}\) i \(\Delta^{\uparrow}\) zbog pomicanja.
Kako implicitno kvantificiramo po varijabli \(0\), a  ona se javlja
(eventualno) samo u formuli \(\varphi\), smijemo univerzalno kvantificirati formulu \(\varphi\).

\begin{figure}[!htb]
  \centering
  \begin{prooftree}
    \AxiomC{\( \varphi[t \cdot \sigma_{\mathit{id}}], \Gamma \vdash \Delta \)}
    \RightLabel{\( \mathit{ (AllL) } \)}
    \UnaryInfC{\( \forall\varphi, \Gamma \vdash \Delta \)}
  \end{prooftree}

  \begin{prooftree}
    \AxiomC{\( \Gamma^{\uparrow} \vdash \varphi, \Delta^{\uparrow}\)}
    \RightLabel{\( \mathit{ (AllR) } \)}
    \UnaryInfC{\( \Gamma \vdash \forall\varphi, \Delta \)}
  \end{prooftree}
  \caption{Pravila za kvantifikatore sustava \(\mathit{LKID}\).}\label{fig:lkid-quantificational}
\end{figure}
\begin{minted}[firstnumber=24]{coq}
(* Quantifier rules. *)
| AllL : forall Γ Δ φ t,
    LKID (subst_formula (t .: ids) φ :: Γ ⊢ Δ) -> 
    LKID (FAll φ :: Γ ⊢ Δ)
| AllR : forall Γ Δ φ,
    LKID (shift_formulas Γ ⊢ φ :: shift_formulas Δ) ->
    LKID (Γ ⊢ (FAll φ) :: Δ)
\end{minted}

\begin{example}
  Lijevo pravilo za egzistencijalnu kvantifikaciju je dopustivo.
  \begin{prooftree}
    \AxiomC{\( \varphi, \Gamma^{\uparrow} \vdash \Delta^{\uparrow} \)}
    \RightLabel{\( \mathit{ (ExistL) } \)}
    \UnaryInfC{\( \exists\varphi, \Gamma \vdash \Delta\)}
  \end{prooftree}
\begin{minted}[linenos=false]{coq}
Lemma ExistL : forall Γ Δ φ,
    LKID (φ :: shift_formulas Γ ⊢ shift_formulas Δ) ->
    LKID (FExist φ :: Γ ⊢ Δ).
\end{minted}
\end{example}

Preostaje nam definicija pravila za induktivne predikatne simbole,
gdje ćemo prvo definirati lijevo pravilo koje nazivamo \textit{pravilo indukcije},
a zatim ćemo definirati desno pravilo koje nazivamo \textit{produkcijsko pravilo}.

\section{Pravilo indukcije}
Neka je \(P_{j}\) proizvoljni induktivan predikatni simbol za koji definiramo pravilo indukcije.
Svakom induktivnom predikatnom simbolu \(P_{i}\) dodjeljujemo
proizvoljnu \(|P_{i}|\)-torku \textit{induktivnih varijabli} \(\mathbf{z}_{i}\) te
proizvoljnu \textit{induktivnu hipotezu} \(G_{i}\) takvu da
vrijedi \(G_{i} = P_{i}(\mathbf{z}_{i})\) ako \(P_{i}\) i \(P_{j}\) nisu međusobno zavisni.
Sada pravilo indukcije za induktivan predikatni simbol \(P_{j}\) glasi:
\begin{prooftree}
  \AxiomC{\textit{male premise}}
  \AxiomC{\( G_{j} [\mathbf{u} / \mathbf{z}_{i}], \Gamma  \vdash \Delta\)}
  \RightLabel{\( \mathit{ (Ind P_{j}) } \)}
  \BinaryInfC{\( P_{j} \mathbf{u}, \Gamma  \vdash \Delta \)}
\end{prooftree}
\noindent gdje sekventu \( G_{j} [\mathbf{u} / \mathbf{z}_{i}] , \Gamma  \vdash \Delta\) zovemo \textit{velikom premisom},
a za svaku produkciju u skupu induktivnih definicija \(\Phi\) u čijoj se konkluziji javlja
induktivni predikatni simbol \(P_{i}\) koji je međusobno zavisan sa simbolom \(P_{j}\), primjerice:
\begin{prooftree}
  \AxiomC{\(  Q_{1}\mathbf{u}_{1}  \ldots   Q_{n}\mathbf{u}_{n}  \)}
  \AxiomC{\(  P_{i_{1}}\mathbf{v}_{i_{1}}  \ldots   P_{i_{m}}\mathbf{v}_{i_{m}}  \)}
  \RightLabel{,}
  \BinaryInfC{\(P_{i}\mathbf{t}\)}
\end{prooftree}
\noindent postoji \textit{mala premisa} oblika:
\[
  Q_{1} \mathbf{u}_{1}^{\uparrow k}, \ldots, Q_{n} \mathbf{u}_{n}^{\uparrow k},
  G_{i_{1}}[ \mathbf{v}_{i_{1}}^{\uparrow k} / \mathbf{z}_{i_{1}} ], \ldots, G_{i_{m}}[ \mathbf{v}_{i_{m}}^{\uparrow k} / \mathbf{z}_{i_{m}} ],
  \Gamma \vdash G_{i}[\mathbf{t}^{\uparrow k} / \mathbf{z}_{i}], \Delta
\]
\noindent gdje je \(k\) neki prirodan broj koji se ne javlja u listama formula \(\Gamma\) i \(\Delta\) te formuli \(P_{j}\mathbf{u}\),
a s \(^{\uparrow k}\) označavamo supstituciju koja svaku varijablu \(n\) mijenja varijablom \(n + k\).
\begin{minted}[firstnumber=31]{coq}
(* Induction rules. *)
| IndL : forall Γ Δ
           (Pj : IndPredS Σ) (u : vec _ (indpred_ar Pj))
           (z_i : forall P, vec var (indpred_ar P))
           (z_i_nodup : forall P, VecNoDup (z_i P))
           (G_i : IndPredS Σ -> formula Σ)
           (HG2 : forall Pi, ~mutually_dependent Pi Pj ->
                        G_i Pi = FIndPred
                                   Pi
                                   (V.map var_term (z_i Pi))),
    let maxΓ := max_fold (map some_var_not_in_formula Γ) in
    let maxΔ := max_fold (map some_var_not_in_formula Δ) in
    let maxP := some_var_not_in_formula (FIndPred Pj u) in
    let shift_factor := max maxP (max maxΓ maxΔ) in
    let Fj := subst_formula
                (finite_subst (z_i Pj) u)
                (G_i Pj) in
    let minor_premises :=
      (forall pr (HΦ : Φ pr) (Hdep : mutually_dependent (indcons pr) Pj),
          let Qs := shift_formulas_by
                      shift_factor
                      (FPreds_from_preds (preds pr)) in
          let Gs := map (fun '(P; args) =>
                           let shifted_args :=
                             V.map
                               (shift_term_by shift_factor)
                               args in
                           let σ :=
                             finite_subst
                               (z_i P)
                               (shifted_args) in
                           let G := G_i P in
                           subst_formula σ G)
                      (indpreds pr) in
          let Pi := indcons pr in
          let ty := V.map
                      (shift_term_by shift_factor)
                      (indargs pr) in
          let Fi := subst_formula
                      (finite_subst (z_i Pi) ty)
                      (G_i Pi) in
          LKID (Qs ++ Gs ++ Γ ⊢ Fi :: Δ))
    in
    minor_premises ->
    LKID (Fj :: Γ ⊢ Δ) ->
    LKID (FIndPred Pj u :: Γ ⊢ Δ)
\end{minted}

Ilustrirajmo pravilo indukcije primjerima u Peanovoj aritmetici.
Kako se pravila za induktivne predikate sustava \(\mathit{LKID}\) odnose na proizvoljan skup induktivnih definicija \(\Phi\),
označimo s \(\mathit{LKID}_{\mathit{PA}}\)  sustav koji se odnosi na skup induktivnih defincija \(\Phi_{\mathit{PA}}\).
\begin{example}
  Pravilo indukcije induktivnog predikatnog simbola \(\mathit{Nat}\)
  u sustavu \(\mathit{LKID}_{\mathit{PA}}\) glasi:
  \begin{prooftree}
    \AxiomC{\( \Gamma \vdash G(o), \Delta \)}
    \AxiomC{\( G(x), \Gamma \vdash G(s(x)), \Delta \)}
    \AxiomC{\( G(t), \Gamma \vdash \Delta \)}
    \RightLabel{\( \mathit{(NatIndL)} \)}
    \TrinaryInfC{\( \mathit{Nat}(t), \Gamma \vdash \Delta \)}
  \end{prooftree}
  \noindent gdje je G induktivna hipoteza vezana uz induktivni predikatni simbol \(\mathit{Nat}\),
  a \(x\) je varijabla koja se ne javlja u \(\Gamma\), \(\Delta\) ili \(\mathit{Nat}(t)\).
\end{example}

\begin{example}
  Pravilo indukcije induktivnog predikatnog simbola  \(\mathit{Even}\)
  u sustavu \(\mathit{LKID}_{\mathit{PA}}\) glasi:
  \begin{small}
    \begin{prooftree}
      \AxiomC{\( \Gamma \vdash G_{E}(o), \Delta \quad \Gamma, G_{E}(x) \vdash G_{O}(s(x)), \Delta \quad \Gamma, G_{O}(x) \vdash G_{E}(s(x)), \Delta \)}
      \AxiomC{\( \Gamma, G_{E}(t) \vdash \Delta \)}
      \RightLabel{\(\mathit{EvenIndL}\)}
      \BinaryInfC{\( \Gamma, \mathit{Even}(t) \vdash \Delta \)}
    \end{prooftree}
  \end{small}
  \noindent gdje su \(G_{E}\) i \(G_{O}\) induktivne hipoteze vezane uz
  induktivne predikatne simbole \(\mathit{Even}\) i \(\mathit{Odd}\),
  a \(x\) je varijabla koja se ne javlja u \(\Gamma\), \(\Delta\) ili \(\mathit{Even}(t)\).
\end{example}

\section{Produkcijsko pravilo}
Neka je \(pr\) proizvoljna produkcija za koju definiramo produkcijsko pravilo.
Možemo pretpostaviti da je \(pr\) idućeg oblika:
\begin{prooftree}
  \AxiomC{\(  Q_{1}\mathbf{u}_{1}  \ldots   Q_{n}\mathbf{u}_{n}  \)}
  \AxiomC{\(  P_{1}\mathbf{v}_{1}  \ldots   P_{m}\mathbf{v}_{m}  \)}
  \RightLabel{.}
  \BinaryInfC{\(P\mathbf{t}\)}
\end{prooftree}
\noindent Produkcijsko pravilo za produkciju \(pr\) glasi:
\begin{prooftree}
  \AxiomC{\( \Gamma \vdash Q_{1} \mathbf{u}_{1}[\sigma], \Delta \quad \ldots \quad \Gamma \vdash Q_{n}\mathbf{u}_{n}[\sigma]\)}
  \AxiomC{\( \Gamma \vdash P_{1} \mathbf{v}_{1}[\sigma], \Delta \quad \ldots \quad \Gamma \vdash P_{m} \mathbf{v}_{m}[\sigma] \)}
  \RightLabel{\( \mathit{ (prIndR) }\)}
  \BinaryInfC{\( \Gamma \vdash P \mathbf{t}[\sigma] \)}
\end{prooftree}
\begin{minted}[firstnumber=58]{coq}
| IndR : forall Γ Δ pr σ,
    Φ pr ->
    (forall Q us,
        In (Q; us) (preds pr) ->
        LKID (Γ ⊢ (FPred Q (V.map (subst_term σ) us) :: Δ))) ->
    (forall P ts,
        In (P; ts) (indpreds pr) ->
        LKID (Γ ⊢ (FIndPred P (V.map (subst_term σ) ts) :: Δ))) ->
    LKID ( Γ ⊢ FIndPred
             (indcons pr)
             (V.map (subst_term σ) (indargs pr))
             :: Δ).
\end{minted}

\begin{example}
  Produkcijsko pravilo za produkciju \texttt{PA\_prod\_E\_succ} glasi:
  \begin{prooftree}
    \AxiomC{\( \Gamma \vdash \mathit{Odd}(x)[\sigma], \Delta \)}
    \UnaryInfC{\( \Gamma \vdash \mathit{Even}(s(x))[\sigma], \Delta \)}
  \end{prooftree}
  \noindent gdje je \(x\) proizvoljna varijabla, a \(\sigma\) proizvoljna supstitucija.
  Supstitucijom \(\sigma\) kažemo da u produkcijama imena varijabli nisu bitna,
  već samo njihov redoslijed.
\end{example}

Konstruktorima \texttt{IndL} te \texttt{IndR} smo dovršili definiciju tipa \texttt{LKID}.
Ističemo suptilnu razliku između pravila za induktivne predikate;
lijevo pravilo se definira za pojedini \textit{induktivni predikatni simbol},
dok se desno pravilo definira za pojedinu \textit{produkciju}.

\section{Dokaz i teorem}
Prikazali smo sva pravila izvoda sustava \(\mathit{LKID}\) te konačno možemo definirati dokaz.
Za proizvoljnu sekventu \(\Gamma \vdash \Delta\), njen \textbf{dokaz} je stanovnik tipa \texttt{LKID(\(\Gamma \vdash \Delta\))}.
Za formulu \(\varphi\) kažemo da je \textbf{teorem} sustava \(\mathit{LKID}\) te
pišemo \(\vdash \varphi\) ako postoji dokaz sekvente \( () \vdash (\varphi)\).
Neformalni pojam konačnog stabla formaliziran je induktivnim tipom \texttt{LKID},
dok njegovi konstruktori formaliziraju odgovarajuća pravila izvoda.

\begin{example}\label{ex:lkid-indr-proof}
  Formula \(\forall x, \mathit{Even}(x) \rightarrow \mathit{Even}(ssx)\) je dokaziva u sustavu \(\mathit{LKID}_{\mathit{PA}}\).
\begin{minted}{coq}
Definition Even_succ_succ_Even : formula Σ__PA :=
  FAll
    (FImp
       (FIndPred PA_Even [var_term 0])
       (FIndPred PA_Even [TFunc PA_succ
                            [TFunc PA_succ
                               [var_term 0]]])).
\end{minted}
  Neformalni dokaz prikazujemo idućim stablom, a radi preglednosti
  pišemo \(Et\) umjesto \(\mathit{Even}(t)\) te \(Ot\) umjesto \(\mathit{Odd}(t)\).
  \begin{prooftree}
    \AxiomC{}
    \RightLabel{\( \mathit{(Ax)} \)}
    \UnaryInfC{\( Ex \vdash Ex, Essx \)}
    \RightLabel{\( \mathit{(} \mathtt{PA\_prod\_O\_succ} \mathit{IndR)} \)}
    \UnaryInfC{\( Ex \vdash Osx, Essx \)}
    \AxiomC{}
    \RightLabel{\( \mathit{(Ax)} \)}
    \UnaryInfC{\( Ex, Osx \vdash Osx \)}
    \RightLabel{\( \mathit{(} \mathtt{PA\_prod\_E\_succ}\mathit{IndR)} \)}
    \UnaryInfC{\( Ex, Osx \vdash Essx \)}
    \RightLabel{\( \mathit{(Cut)} \)}
    \BinaryInfC{\( Ex \vdash Essx \)}
    \RightLabel{\( \mathit{(ImpR)} \)}
    \UnaryInfC{\(  \vdash Ex \rightarrow Essx \)}
    \RightLabel{\( \mathit{(AllR)} \)}
    \UnaryInfC{\( \vdash \forall x, Ex \rightarrow Essx \)}
  \end{prooftree}
\begin{minted}{coq}
Lemma provable_Even_succ_succ_Even :
    @LKID _ Φ__PA ([] ⊢ [Even_succ_succ_Even]).
\end{minted}
\end{example}

\begin{example}\label{ex:lkid-indl-proof}
  Formula \(\forall x, \mathit{Nat}(x) \rightarrow \mathit{Even}(x) \lor \mathit{Odd}(x)\) je dokaziva u sustavu \(\mathit{LKID}_{\mathit{PA}}\).
  Neformalni dokaz prikazujemo idućim stablom, a radi preglednosti
  pišemo \(Nt\), \(Et\) i \(Ot\) redom umjesto \(\mathit{Nat}(t)\), \(\mathit{Even}(t)\) i \(\mathit{Odd}(t)\).
  \begin{scriptsize}
    \begin{prooftree}
      \AxiomC{}
      \RightLabel{\(\mathit{(1)}\)}
      \UnaryInfC{\( \vdash Eo \)}
      \RightLabel{\(\mathit{(Wk)}\)}
      \UnaryInfC{\( \vdash Eo, Ex \lor Ox\)}
      \RightLabel{\(\mathit{(OrR_{1})}\)}
      \UnaryInfC{\(\vdash Eo \lor Oo, Ex \lor Ox\)}
      \AxiomC{}
      \RightLabel{\(\mathit{(Ax)}\)}
      \UnaryInfC{\( Ey \vdash Ey\)}
      \RightLabel{\(\mathit{(2)}\)}
      \UnaryInfC{\(Ey \vdash Osy\)}
      \RightLabel{\(\mathit{(OrR_{2})}\)}
      \UnaryInfC{\( Ey \vdash Esy \lor Osy \)}
      \AxiomC{}
      \RightLabel{\(\mathit{(Ax)}\)}
      \UnaryInfC{\(Oy \vdash Oy\)}
      \RightLabel{\(\mathit{(3)}\)}
      \UnaryInfC{\(Oy \vdash Esy\)}
      \RightLabel{\(\mathit{(OrR_{1})}\)}
      \UnaryInfC{\(Oy \vdash Esy \lor Osy\)}
      \RightLabel{\(\mathit{(OrL)}\)}
      \BinaryInfC{\( Ey \lor Oy \vdash Esy \lor Osy \)}
      \RightLabel{\(\mathit{(Wk)}\)}
      \UnaryInfC{\( Ey \lor Oy \vdash Esy \lor Osy, Ex \lor Ox \)}
      \AxiomC{}
      \RightLabel{\(\mathit{(Ax)}\)}
      \UnaryInfC{\( Ex \lor Ox \vdash Ex \lor Ox\)}
      \RightLabel{\(\mathit{(NatIndL)}\)}
      \TrinaryInfC{\( Nx \vdash Ex \lor Ox \)}
      \RightLabel{\( \mathit{(ImpR)}\)}
      \UnaryInfC{\( Nx \rightarrow Ex \lor Ox \)}
      \RightLabel{\(\mathit{(AllR)}\)}
      \UnaryInfC{\( \vdash \forall x, Nx \rightarrow Ex \lor Ox \)}
    \end{prooftree}
  \end{scriptsize}
  \noindent Ovdje je induktivna hipoteza formula \(Ex \lor Ox\).
  Valja primijetiti da smo u dokazu koristili i produkcijska pravila za produkcije:
  \begin{enumerate}[label={(\arabic*)}]
  \item \texttt{PA\_prod\_E\_zero},
  \item \texttt{PA\_prod\_O\_succ} i
  \item \texttt{PA\_prod\_E\_succ}.
  \end{enumerate}
\begin{minted}{coq}
Lemma provable_every_nat_is_even_or_odd :
  @LKID Σ__PA Φ__PA ([] ⊢ [every_nat_is_even_or_odd]).
\end{minted}
\end{example}

\section{Adekvatnost}\label{sec:adekvatnost}
Za dokazivače teorema nužno je da dokazivost formule povlači njenu istinitost ---
u~\ref{chap:coq} poglavlju objasnili smo zašto.
Ovo svojstvo dokaznosg sustava naziva se \textit{adekvatnost},
a ovdje ćemo ju prikazati na primjeru sustava \(\mathit{LKID}\).
Za istinitost ćemo promatrati samo strukture koje su standardni modeli
jer u takvima interpretacije induktivnih predikatnih simbola imaju smisla.

\begin{definition}
  Kažemo da je sekventa \(\Gamma \vdash \Delta\) \textit{istinita} i pišemo \(\Gamma \VDash \Delta\) ako i samo ako
  za svaki standardni model \(M\) i njegovu okolinu \(\rho\) vrijedi:
  \[
    (\forall \varphi, \varphi \in \Gamma \rightarrow \rho \vDash \varphi) \rightarrow \exists \psi, \psi \in \Delta \land \rho \vDash \psi,
  \]
  \noindent odnosno istinitost svake formule u \(\Gamma\) povlači istinitost neke formule u \(\Delta\),
  u skladu s intuitivnim shvaćanjem sekventi.
\begin{minted}{coq}
Definition Sat_sequent (s : sequent) : Prop :=
  let '(Γ ⊢ Δ) := s in            
  forall (M : structure Σ), standard_model Σ Φ M -> forall (ρ : env M),
      (forall φ, In φ Γ -> ρ ⊨ φ) -> exists ψ, In ψ Δ /\ ρ ⊨ ψ.
\end{minted}
\end{definition}

Kažemo da vrijedi \textit{lokalna adekvatnost} za pravilo izvoda
ako istinitost premisa (a posebno i sekvenata u premisama) povlači istinitost konkluzije (odnosno sekvente u konkluziji).
Lokalna adekvatnost kaže da se istinitost čuva kod primjena pravila zaključivanja.

\begin{example}
  Za pravilo \(\mathit{Ax}\) vrijedi lokalna adekvatnost,
  to jest ako u listama formula \(\Gamma\) i \(\Delta\) postoji neka zajednička formula \(\varphi\),
  onda vrijedi \(\Gamma \VDash \Delta\). Tada je \textit{svjedok istinitosti} u \(\Delta\) upravo formula \(\varphi\).
\begin{minted}{coq}
Lemma LS_Ax : forall Γ Δ φ, In φ Γ -> In φ Δ -> Γ ⊫ Δ.
\end{minted}
\end{example}

Ako za sva pravila izvoda vrijedi lokalna adekvatnost,
indukcijom po strukturi dokaza slijedi i (globalna) adekvatnost.
Glavni cilj ovog poglavlja bio je dokazati lokalne adekvatnosti svih pravila sustava \(\mathit{LKID}\).
Cilj nije postignut samo za pravilo indukcije.

\begin{theorem}
  Za pravila izvoda sustava \(\mathit{LKID}\) osim \(\mathit{IndL}\) vrijedi lokalna adekvatnost.
  Nadalje, ako je sekventa \(\Gamma \vdash \Delta\) dokaziva bez primjene pravila \(\mathit{IndL}\), onda je ona istinita.
\end{theorem}

\misao{Ako ne uspijem dokazati lokalnu adekvatnost za IndL, onda ću ostaviti ovaj tekst.}

%%% Local Variables:
%%% mode: LaTeX
%%% TeX-master: "master"
%%% End:
