\documentclass[diplomskirad, upload]{fer}
% Dodaj opciju upload za generiranje konačne verzije koja se učitava na FERWeb

\usepackage[T1]{fontenc}
\usepackage{amsmath}
\usepackage{amsthm}
\usepackage{amssymb}
\usepackage{mathtools}
\usepackage{enumitem}
\usepackage{hyperref}
\usepackage{booktabs}
\usepackage{minted}
\usepackage{xcolor}
\usepackage{tikz}
\usepackage{forest}
\usepackage{pmboxdraw}
\usepackage{textgreek}
\usepackage{bussproofs}
\usepackage{pdflscape}
\usepackage{float}
\usepackage[makeroom]{cancel}
\usepackage[autostyle,croatian=quotes]{csquotes}

% https://github.com/stipub/stix_type1/issues/3
\renewcommand{\mapsto}{\mathchar"39AD}

% smanjenje proreda u itemize i enumerate okolinama
% \setlist{topsep=-20pt}
\setlist{nosep, topsep=-10pt}

% smanjenje proreda u minted okolini
\BeforeBeginEnvironment{minted}{\begin{spacing}{1.0}}
\AfterEndEnvironment{minted}{\end{spacing}}
\AfterEndEnvironment{minted}{\vspace{0pt}}

\DeclareUnicodeCharacter{22A8}{\(\vDash\)}
\DeclareUnicodeCharacter{22A2}{\(\vdash\)}
\DeclareUnicodeCharacter{2286}{\(\subseteq\)}
\DeclareUnicodeCharacter{22AB}{\(\VDash\)}
\DeclareUnicodeCharacter{2291}{\(\sqsubseteq\)}

\renewcommand{\ldots}{\textellipsis}

% BEGIN okoline za definicije
\theoremstyle{definition}
\newtheorem{definition}{Definicija}
\newtheorem{example}{Primjer}
\newtheorem{lemma}{Lema}
\newtheorem{theorem}{Teorem}
\newtheorem{proposition}{Propozicija}
\newtheorem{corollary}{Korolar}

\theoremstyle{remark}
\newtheorem*{remark}{Napomena}
% END okoline za definicije

% postavke za coq unutar minted okoline
\setminted[coq]{frame=single, linenos, obeytabs, samepage}


\title{Applications of the Coq Proof Assistant in mathematics and computer science}
\naslov{Primjene Coq alata za dokazivanje u matematici i računarstvu}
\brojrada{578}
\author{Miho Hren}
\mentor{}

% makro za engleske prijevode
\renewcommand{\engl}[1]{(\textit{#1})}

% za pisati todo
\newcommand\misao[1]{ \textcolor{red}{\textbf{Misao:} \textit{#1}} }

% 
\newcommand\coqset{\texttt{Set}}
\newcommand\coqprop{\texttt{Prop}}
\newcommand\coqtype[1]{\texttt{Type\(_{#1}\)}}

\begin{document}
\maketitle
\zadatak{zadatak.pdf}

\begin{zahvale}
  Ovo je zahvala.
\end{zahvale}

\mainmatter{}

\tableofcontents

\chapter{Uvod}\label{cha:uvod}

\chapter{Coq}\label{pog:glavni_dio}
U ovom poglavlju dajemo pogled svisoka na programski sustav Coq.
Prvo ćemo objasniti što je uopće Coq, u kojem je kontekstu nastao, i od kojih komponenti se sastoji.
Zatim ćemo dati kratak pregled programiranja u Coqu, nakon čega ćemo se baviti naprednijim konceptima i spomenuti neka ograničenja.
Za širi opseg gradiva, čitatelja upućujemo na knjige \textit{Coq'Art}~\cite{coqart}, \textit{Software Foundations}~\cite{sf-lf, sf-plf, sf-vfa} i \textit{Certified Programming with Dependent Types}~\cite{cpdt} te na službenu dokumentaciju~\cite{coqrefman}.

\section{Što je Coq?}\label{sec:osnovno-o-coqu}
Alat za dokazivanje Coq\footnote{\url{https://coq.inria.fr/}}, punog naziva \textit{The Coq Proof Assistant},
programski je sustav pomoću kojeg korisnici mogu dokazivati matematičke tvrdnje. \misao{ne sluzi samo tome, moze biti i opceniti funkcijski programski jezik, moze sluziti za programiranje sa zavisnim tipovima}
Alat se temelji na \(\lambda\)-računu i teoriji tipova, a prva je inačica implementirana godine 1984.~\cite{coqrefman}
Ovaj rad koristi inačicu \(8.18\) iz rujna godine 2023.

Program Coq može se pokrenuti u interaktivnom ili u skupnom načinu rada.
Interaktivni način rada pokreće se naredbom \texttt{coqtop}, a korisniku omogućuje rad u ljusci sličnoj \texttt{bash} i \texttt{python} ljuskama.
Interaktivna ljuska (također poznata pod imenom \textit{toplevel}) služi unosu definicija i iskazivanju lema.
Skupni način rada pokreće se naredbom \texttt{coqc}, a korisniku omogućuje semantičku provjeru i prevođenje izvornih datoteka u jednostavnije formate.
Kod formaliziranja i dokazivanja, korisnik će najčešće koristiti interaktivni način rada, po mogućnosti kroz neku od dostupnih razvojnih okolina.
\footnote{Autor rada koristio je paket \textit{Proof General} za uređivač teksta \textit{Emacs}. Druge često korištene okoline su \textit{VsCoq} i \textit{CoqIDE}.}

\misao{spomenuti proof mode? spomenuti workflow dokazivanja? ima jedna zgodna slika u QED at Large}
\misao{treba naglasiti da je Coq \textit{interaktivni} dokazivač teorema}

Kao programski jezik, Coq se sastoji od više podjezika različitih namjena, od kojih spominjemo \textit{Vernacular}, \textit{Gallinu} i \textit{Ltac}.
\begin{description}
\item[Vernacular] \misao{\textit{vernacular} znači ``govorni jezik''} je jezik naredbi kojima korisnik komunicira sa sustavom (i u interaktivnom i u skupnom načinu rada);
  svaka Coq skripta (datoteka s nastavkom \texttt{.v}) je niz naredbi.
  Neke od najčešće korištenih naredbi su \texttt{Check}, \texttt{Definition}, \texttt{Inductive}, \texttt{Fixpoint} i \texttt{Lemma}.
  Pomoću naredbi za tvrdnje, kao što je \texttt{Lemma}, Coq prelazi iz \textit{toplevela}\misao{treba bolji prevod} u način dokazivanja~\engl{proof mode}.
\item[Gallina] je Coqov strogo statički tipiziran specifikacijski jezik.
  Kako se glavnina programiranja u Coqu svodi upravo na programiranje u Gallini, posvećujemo joj idući odjeljak.
\item[Ltac] je Coqov netipiziran jezik za definiciju i korištenje taktika.
  Taktike su pomoćne naredbe kojima se u načinu dokazivanja konstruira dokaz.
  Može se reći da je Ltac jezik za metaprogramiranje Galline.
  Primjeri taktika su \texttt{intros}, \texttt{destruct}, \texttt{apply} i \texttt{rewrite}.
\end{description} % https://coq.discourse.group/t/what-is-the-difference-between-gallina-and-ltac/1184 
\noindent Pogledajmo ilustrativan primjer.
\begin{minted}{coq}
  Lemma example_lemma : 1 + 1 = 2.
  Proof.
    cbn. reflexivity.
  Qed.
\end{minted}
\noindent Ključne riječi \texttt{Lemma}, \texttt{Proof} i \texttt{Qed} dio su Vernaculara,
izraz \texttt{example\_lemma : 1 + 1 = 2} dio je Galline,
a pomoćne naredbe \texttt{cbn} i \texttt{reflexivity} dio su Ltaca.

Jezgra programskog sustava Coq je algoritam za provjeru tipova \engl{type checking} implementiran u OCamlu ---
svaka tvrdnja koja se dokazuje izrečena je pomoću tipova.
Ostatak sustava u načelu služi za knjigovodstvo i poboljšanje korisničkog iskustva.
Nužno je da jezgra sustava bude relativno mala kako bismo se mogli uvjeriti u njenu točnost.
U suprotnom, možemo li biti sigurni da su naše dokazane tvrdnje doista istinite?
\misao{kažem istinite, ali u stvari mislim dokazive, no to je nespretno za napisati i diskusija oko toga je preopćenita}

Prve inačice Coqa implementirale su račun konstrukcija, no kasnije je dodana podrška za induktivno i koinduktivno definirane tipove~\cite{cic, coinductive}.
Danas se može reći da Coq implementira polimorfni kumulativni račun induktivnih konstrukcija~\cite{coqcoqcorrect}.
\misao{mogu spomenuti i preteče računa konstrukcija i jezike koji ih implementiraju, npr.~ Lisp je implementacija \(\lambda\)-računa}
Coq se, osim kao dokazivač teorema, može koristiti i za programiranje sa zavisnim tipovima.
U toj sferi konkuriraju jezici Agda\footnote{\url{https://wiki.portal.chalmers.se/agda/}}, Idris\footnote{\url{https://www.idris-lang.org/}} i Lean\footnote{\url{https://lean-lang.org/}}. Coq se između njih istiće po usmjerenosti prema dokazivanju, posebno po korištenju taktika (jezik Ltac) i nepredikativnoj sorti \texttt{Prop} (o kojoj će kasnije biti riječi).
Još jedna prednost Coqa je mehanizam \textit{ekstrakcije} pomoću kojeg korisnik može proizvoljnu funkciju\footnote{Za koju je dokazao točnost, štogod to značilo.} prevesti u jezik niže razine apstrakcije.\footnote{Trenutno su podržani Haskell, OCaml i Scheme.}
Mehanizam ekstrakcije nije dokazano točan, no poželjno je da funkcije zadržavaju točnost i nakon ekstrakcije pa se radi na verifikaciji ekstrakcije~\cite{coqcoqcorrect}.

\section{Programiranje u Coqu}\label{sec:programiranje-u-gallini}
Gallina je funkcijski programski jezik, što znači da su funkcije prvoklasni objekti ---
funkcije mogu biti argumenti i povratne vrijednosti drugih funkcija.
Dodatno, varijable su nepromjenjive \engl{immutable} te se iteracija ostvaruje rekurzijom.
Za uvod u funkcijsko programiranje, čitatelja upućujemo na knjigu \textit{Programming in Haskell}~\cite{Hutton_2016}.
Primjeri koje ćemo vidjeti u ostatku ovog odjeljka dijelom se osljanjanju na tipove i funkcije definirane u Coqovoj standardnoj knjižnici.\footnote{\url{https://coq.inria.fr/library/}}

Gallina je strogo statički tipiziran jezik, što znači da se svakom termu prilikom prevođenja dodjeljuje tip\footnote{Tipovi su kolekcije srodnih objekata.}.
Naredbom \texttt{Check} možemo provjeriti tip nekog terma ili doznati da se termu ne može dodjeliti tip.
Dalje u radu pod ``term'' mislimo na dobro formirane terme, odnosno na one terme kojima se može dodijeliti tip.
Kažemo da je term \textit{stanovnik} tipa koji mu je dodijeljen.
Za tip kažemo da je \textit{nastanjen}, odnosno \textit{nenastanjen}, ako postoji, odnosno ne postoji, stanovnik tog tipa.
Kako su u Coqu i tipovi termi, radi razumljivosti i zvučnosti umjesto ``tip tipa'' kažemo ``sorta tipa''.

Kao i u ostalim jezicima, kod programiranja u Coqu korisnik se oslanja na dostupne primitivne izraze,
od kojih su najvažniji:
\begin{itemize}
\item \texttt{forall}, pomoću kojeg se konstruiraju funkcijski tipovi i zavisni produkti;
\item \texttt{match}, pomoću kojeg se provodi \textit{pattern matching};
\item \texttt{fun}, pomoću kojeg se definiraju funkcije;
\item \texttt{fix}, pomoću kojeg se definiraju rekurzivne funkcije i
\item \texttt{cofix}, pomoću kojeg se definiraju korekurzivne funkcije.
\end{itemize}

Jedna od osnovnih naredbi za stvaranje novih terma je naredba \texttt{Definition}.
\begin{minted}{coq}
Definition negb (b : bool) : bool :=
match b with
| false => true
| true => false
end.
\end{minted}
\noindent U gornjem kodu definirana je funkcija \texttt{negb} čiji se argument \texttt{b} tipa \texttt{bool} destrukturira te se vraća njegova negacija, također tipa \texttt{bool}.
Važno je napomenuti da svaki \texttt{match} izraz mora imati po jednu granu za svaki konstruktor tipa.\footnote{U tandemu s uvjetom strukturalne rekurzije, ovime je osigurana totalnost svake funkcije.}
U ovom su primjeru konstante \texttt{false} i \texttt{true} jedini konstruktori tipa \textit{bool}.
Funkcija \texttt{negb} je tipa \texttt{bool\(\rightarrow\)bool}.
\begin{minted}{coq}
Definition mult_zero_r : Prop := forall (n : nat), n * 0 = 0.
\end{minted}
Ovdje je definirana propozicija (tip) imena \texttt{mult\_zero\_r} kao univerzalno kvantificirana tvrdnja po prirodnim brojevima.

Naredbom \texttt{\textbf{Inductive}} definira se \textit{induktivni} tip te se automatski za njega generiraju principi \textit{indukcije} i \textit{rekuzrije}.
\begin{minted}{coq}
Inductive nat : Set :=
| O : nat
| S : nat -> nat.
\end{minted}
Ovim kodom definirali smo tri terma:
\begin{itemize}
\item \texttt{nat} (tip prirodnih brojeva) je term sorte \texttt{Set},
\item \texttt{O} (broj nula) je term tipa \texttt{nat} i
\item \texttt{S} (funkcija sljedbenika) je term tipa \texttt{nat\(\rightarrow\)nat}.
\end{itemize}
\noindent Za term \texttt{nat} kažemo da je konstruktor tipa \engl{type constructor}, a za terme \texttt{O} i \texttt{S} kažemo da su konstruktori objekata \engl{object constructors}.

Rekurzija nad induktivnim tipovima može se ostvariti naredbom \texttt{Fixpoint}, koja u pozadini koristi izraz \texttt{fix}.
\begin{minted}{coq}
Fixpoint plus (n m : nat) {struct n} : nat :=
match n with
| O => m
| S n' => S (plus n' m)
end.
\end{minted}
U ovom primjeru definirana je funkcija \texttt{plus} koja prima dva argumenta tipa \texttt{nat}.
Funkcija je rekurzivna po prvom argumentu što je vidljivo oznakom \texttt{\{struct n\}}.
Napominjemo da su induktivni tipovi dobro utemeljeni, to jest svaki term induktivnog tipa je konačan.

Osim induktivnih, u Coqu postoje i koinduktivni tipovi, koji nisu dobro utemeljeni,
zbog čega za njih nije moguće definirati principe indukcije i rekurzije.
Umjesto rekurzije, koinduktivni tipovi koriste se u korekurzivnim funkcijama.
Standardan primjer koinduktivnog tipa je beskonačna lista.
\begin{minted}{coq}
Set Primitive Projections.
CoInductive Stream (A : Type) := {
      hd : A;
      tl : Stream A;
    }.
\end{minted}
Ovime smo definirali familiju tipova \texttt{Stream} indeksiranu tipskom varijablom \texttt{A}.
Svaki \texttt{Stream} ima glavu i rep koji je također \texttt{Stream}.

Stanovnici koinduktivnih tipova konstruiraju se korekurzivnim funkcijama naredbom \texttt{CoFixpoint}, koja u pozadini koristi \texttt{cofix}.
\begin{minted}{coq}
CoFixpoint from (n : nat) : Stream nat := Cons _ n (from (n + 1)).
\end{minted}
Ovime je definirana funkcija \texttt{from} koja za ulazni argument \texttt{n} vraća niz prirodnih brojeva od \(n\) na dalje.

Razlika induktivnih i koinduktivnih tipova može se izreći epigramom:\\
\[
  \textrm{``Induktivni tipovi su domene rekurzivnih funkcija, a koinduktivni tipovi su kodomene korekurzivnih funkcija''.}
\]
\noindent Time se želi reći da se termi induktivnih tipova destrukturiraju u rekurzivnim funkcijama,
dok se termi koinduktivnih tipova konstruiraju u korekurzivnim funkcijama.

\misao{Barendregtova kocka, term koji ovisi o termu, term koji ovisi o tipu, tip koji ovisi o termu, tip koji ovisi o tipu}

\section{Hijerarhija tipova}\label{sec:kumul-hijer-tipova}
U usporedbi s tradicionalnim programskim jezicima, Coqov tipski sustav je ekspresivniji jer dopušta tipove koji mogu ovisiti o termima.
Takvi tipovi se u Coqu konstruiraju \texttt{forall} izrazom.
Primjer jednog takvog tipa je ``lista duljine \(n\)'', gdje je \(n\) neki prirodan broj, a čiji su stanovnici \(n\)-torke.
Općeniti tip tada glasi ``za svaki \(n\), lista duljine \(n\)'' --- on ovisi o stanovniku drugog tipa (u ovom slučaju, tipa \texttt{nat}).

Spomenuli smo da su i tipovi termi te im se može dodijeliti sorta.
Postoji li najveći tip, odnosno postoji li tip \texttt{Type} čiji su stanovnici svi dobro formirani termi?
Prisjetimo se, svaki term ima svoj tip.
Kada bi takav \texttt{Type} postojao, tada bi vrijedilo \verb|Type:Type|, što može dovesti do paradoksa lašca\footnote{Ova rečenica je lažna. U teoriji skupova to je Russellov paradoks, u teoriji tipova to je Girardov paradoks}.
Takav dokazivač teorema bio bi inkonzistentan te bismo njime mogli dokazati kontradikciju, čime dokazivač efektivno gubi svoju svrhu.
Umjesto jedne ``velike'' sorte \texttt{Type}, u Coqu postoji niz monotono rastućih sorti \coqtype{n} za sve prirodne brojeve \(n\),
takav da vrijedi \(\coqtype{n} : \coqtype{m}\) kad god vrijedi \(n < m\).
Ilustracija ove \textbf{kumulativne hijerarhije tipova} prikazana je na slici~\ref{fig:kum-hijer-tip}.
Dvije najvažnije sorte u Coqu su sorta \texttt{Set} i sorta \texttt{Prop}.

\begin{figure}[htb]
  \centering
  \begin{forest}
    for tree={l sep+=5mm}
    [\vdots
    [\coqtype{m+1}
    [\ldots]
    [\texttt{bool\(\rightarrow\)\coqtype{m}}]
    [\coqtype{m} [\ldots][\vdots
    [\coqtype{1}
    [\coqset [\texttt{bool}\\ \texttt{nat}\\ \vdots, align=center, base=top] [\texttt{list nat}\\ \texttt{prod nat bool}\\ \texttt{nat\(\rightarrow\)nat}\\ \vdots{}, align=center, base=top] [\ldots]]
    [\texttt{bool}\(\rightarrow\)\coqset]
    [\ldots]
    [\coqprop [\texttt{True}\\ \texttt{1+1=2}\\ \vdots, align=center, base=top] [\texttt{False}\\ \texttt{\(\forall\)b,negb b=b}\\ \vdots, align=center, base=top] [\ldots]]
    [\texttt{nat}\(\rightarrow\)\coqprop]
    [\ldots]
    ]
    ] [\ldots]
    ] [\coqtype{m} \(\rightarrow\) \coqprop] [\ldots]]]
  \end{forest}
  \caption{Kumulativna hijerarhija tipova}\label{fig:kum-hijer-tip}
\end{figure}

Naziv \coqset{} sinonim je za sortu \coqtype{0}, a njeni stanovnici su \textbf{mali tipovi}.
\misao{NB: Tu je prije pisalo da su stanovnici od \texttt{Set} programi, no to nije tako. Stavovnici od \texttt{Set} su mali tipovi (ne znam kako ih drugačije nazvati), a tek onda su stanovnici malih tipova programi.} 
Primjerice, tipovi \texttt{nat} i \texttt{bool} su mali tipovi.
Dodatno, funkcije koje primaju i vraćaju male tipove su također mali tipovi.
Tada su i produkti, sume, liste i stabla malih tipova također mali tipovi.
Intuitivno se može reći da su mali oni tipovi s čijim se stanovnicima može efektivno računati.
Stanovnike malih tipova nazivamo \textbf{programima}.

Stanovnici sorte \texttt{Prop} su \textbf{propozicije}.
Za razliku od programa, sa propozicijama ne možemo efektivno računati, već ih moramo dokazivati.
Stanovnici propozicija su njihovi \textbf{dokazi}.

Za dokazivače teorema, poželjna je mogućnost definicije predikata (propozicija) nad proizvoljnim tipovima.
Zbog toga u Coqu prilikom definicije terma sorte \texttt{Prop} možemo raditi kvantifikaciju po proizvoljno velikim tipovima
(što uključuje i sortu \texttt{Prop}).
\begin{minted}{coq}
Inductive isNat : Set -> Prop :=
| IsNat : isNat nat.
\end{minted}
\noindent Tako je ovdje \texttt{isNat} predikat nad sortom \coqset{}, a u primjeru ispod \texttt{not} je predikat nad sortom \coqprop{}.
\begin{minted}{coq}
Definition not (P : Prop) := P -> False.
\end{minted}
\noindent Ovaj stil kvantifikacije omogućuje nam definiciju proizvoljnih propozicija i propozicijskih veznika.
Kažemo da je sorta \coqprop{} \textbf{nepredikativna}.
S druge strane, sorta \coqset{} je \textbf{predikativna}, to jest \textit{ne dopušta} kvantifikaciju po proizvoljno velikim tipovima.
Na praktičnoj strani, predikativnost ograničava korisnika da prilikom definicije programa smije koristiti samo druge programe.


\section{Propozicije i tipovi, dokazi i termi}\label{sec:propozicije-i-tipovi}
U decimalnom zapisu broja \(\pi\), barem jedna znamenka pojavljuje se beskonačno mnogo puta.
Doista, kada bi se svaka znamenka javljala samo konačno mnogo puta, broj \(\pi\) bio bi racionalan.
Međutim, nije jasno \textit{koja} znamenka ima to svojstvo.
Da bismo odgovorili na to pitanje, morali bismo prebrojiti \textit{sve} broja \(\pi\),
što nije moguće u konačno mnogo koraka.

Sličnim pitanjima bavili su se logičari dvadesetog stoljeća. \textit{Klasični} logičari bi gornju tvrdnju smjesta prihvatili,
dok bi \textit{konstruktivisti} tražili konkretnu znamenku.
Između ostalog, ovakva razmatranja rezultirala su fundamentalnim uvidom u povezanost programiranja i dokazivanja.
Naime, želimo li dokazati konjunkciju, dovoljno je zasebno dokazati njene konjunkte.
S druge strane, želimo li konstruirati par objekata, dovoljno je zapakirati prvi i drugi objekt u konstruktor para.
Na sličan način, želimo li dokazati implikaciju, dovoljno je pretpostaviti njen antecedent te pomoću njega dokazati konzekvens.
Ako pak želimo konstruirati funkciju, smijemo uzeti jedan argument i pomoću njega konstruirati povratnu vrijednost.
Dodatno, nemoguće je dokazati laž, a istina trivijalno vrijedi.
S druge strane, ako tip nema konstruktore, nije moguće definirati vrijednost tog tipa.
Ako pak tip ima barem jedan konstruktor, tada postoje i njegovi stanovnici.
Kroz ove primjere vidimo fenomen \textbf{Curry--Howardove korespondencije}, koju možemo sažeti epigramom:
\[
  \textrm{``Propozicije su tipovi, dokazi su programi.''}
\]
\\ Time se dokazivanje svodi na programiranje. Pogledi na dvije strane ovog novčića mogu se vidjeti u tablici~\ref{tab:curryhoward}.
\begin{table}[!hbt]
  \centering
  \begin{tabular}[!htb]{rl}
    \toprule \\
    Dokazivanje & Programiranje \\
    \midrule \\
    propozicija & tip \\
    dokaz & program \\
    laž & prazan tip \\
    istina & nastanjen tip \\    
    konjunkcija & produktni tip \\
    disjunkcija & zbrojni tip \\
    implikacija & funkcijski tip \\
    univerzalna kvantifikacija & zavisni produkt \\
    egzistencijalna kvantifikacija & zavisni koprodukt \\
    \bottomrule
  \end{tabular}
  \caption{Sličnosti dokazivanja i programiranja}\label{tab:curryhoward}
\end{table}

Za bolju ilustraciju, prikazujemo princip matematičke indukcije u Coqu.
Prisjetimo se, za proizvoljni predikat \(P\) nad prirodnim brojevima, princip matematičke indukcije glasi:
\[
  P(0) \land (\forall n, P(n) \rightarrow P(n + 1)) \rightarrow \forall n, P(n).
\]
\noindent Tvrdnju dokazujemo analizom broja \(n\).
Ako je \(n = 0\), tvrdnja slijedi iz \textit{baze} indukcije.
Ako pak je \(n = n^{\prime} + 1\) za neki \(n^{\prime}\), tada rekurzivno konstruiramo dokaz za \(P(n^{\prime})\), a konačna tvrdnja slijedi primjenom \textit{koraka} indukcije na rekurzivno konstruirani dokaz.
\begin{minted}{coq}
Definition nat_ind (P : nat -> Prop) (baza : P 0) (korak : forall n, P n -> P (S n))
  : forall n, P n :=
  fix F (n : nat) : P n :=
    match n with
    | O => baza
    | S n' => korak n' (F n')
    end.
\end{minted}
Za term \texttt{nat\_ind} kažemo da je dokazni objekt \engl{proof object} za tvrdnju matematičke indukcije.
Princip matematičke indukcije je samo poseban slučaj \textbf{principa indukcije}, kojeg Coq automatski generira za svaki induktivno definiran tip.

\section{Ograničenja tipskog sustava}\label{sec:ogranicenja}
Kao što smo već vidjeli, Coqov tipski sustav je ekspresivniji od tipskih sustava tradicionalnih programskih jezika.
Međutim, kako bi se sačuvala poželjna svojstva algoritma provjere tipova, ipak se tipski sustav mora ograničiti.

\textbf{Uvjet pozivitnosti} odnosi se na definiciju induktivnih i koinduktivnih tipova.
\misao{NB: uvjet pozitivnosti mora vrijediti i u pozitivinim i u negativnim koinduktivnim tipovima.}
Ovo ograničenje zabranjuje \textit{negativne} pojave tipa kojeg definiramo u argumentima njegovih konstruktora.
\begin{minted}{coq}
Inductive Lam :=
| LamVar (n : nat)
| LamApp (M N : Lam)
| LamAbs (M : Lam -> Lam).
\end{minted}
\noindent Pokretanje primjera iznad rezultira greškom \texttt{Non strictly positive occurrence of "Lam" in "(Lam -> Lam) -> Lam"} ---
drugim riječima, tip \texttt{Lam} se javlja negativno u konstruktoru \texttt{LamAbs}, odnosno kao argument funkcije koja je parametar konstruktora.
Ovaj uvjet štiti korisnika od inkonzistentnosti, a za točnu definiciju pozitivnosti čitatelja upućujemo na dokumentaciju.\footnote{\url{https://coq.inria.fr/doc/v8.18/refman/language/core/inductive.html\#well-formed-inductive-definitions}}
Uz uvjet pozitivnosti za induktivne tipove vezan je \textbf{uvjet strukturalne rekurzije}.
Ovim uvjetom osigurava se totalnost rekurzivno definirane funkcije
tako da se argument po kojem je funkcija rekurzivna strukturalno smanjuje u svakom koraku rekurzije.

\textbf{Uvjet produktivnosti} odnosi se na definiciju korekurzivnih funkcija, a dualan je uvjetu strukturalne rekurzije.
Ovaj uvjet također štiti korisnika od inkonzistentnosti, a glasi: svaki korekurzivni poziv smije se javljati točno kao izravan argument konstruktora koinduktivnog tipa čiji element definiramo.\footnote{\url{https://coq.inria.fr/doc/v8.18/refman/language/core/coinductive.html\#co-recursive-functions-cofix}}
Zbog tog uvjeta, iduća definicija nije moguća.
\begin{minted}{coq}
Set Primitive Projections.
CoInductive NatStream := {
    nat_hd : nat;
    nat_tl : NatStream;
  }.

CoFixpoint foo : NatStream := foo.
\end{minted}
Greška koju sustav javlja glasi \texttt{Unguarded recursive call in "foo"},
što znači da se korekurzivni poziv \texttt{foo} \textit{ne} javlja kao izravni argument konstruktora.
S druge strane, definicija
\begin{minted}{coq}
CoFixpoint bar : NatStream := {| nat_hd := 0; nat_tl := bar |}.
\end{minted}
\noindent je sasvim legalna.
\misao{Tu se baš i ne vidi zašto se \texttt{bar} javlja kao izravan argument konstruktora jer koristimo negativni koinduktivni tip. Ispod haube je to sve zamotano u neki \texttt{Build\_NatStream}.}

\misao{Bili smo spomenuli četiri ograničenja na sastanku.
  Uvjet strukturalne rekurzije i uvjet produktivnosti su na neki način dualni.
  Treće ograničenje bio je uvjet pozitivnosti.
  Čovjek očekuje da onda postoji nešto dualno uvjetu pozitivnosti za koinduktivne tipove.
  Nije li to opet uvjet pozitivnosti?
}

\misao{Ipak neću spomenuti razlike između pozitivinih i negativnih koinduktivnih tipova.
  Mislim da je to izvan dosega ovog rada.}

Posljednje ograničenje koje spominjemo vezano je uz irelevantnost dokaza \engl{proof irrelevance}.
Naime, mnogi teoremi mogu se dokazati na više načina, ali svaki pojedini dokaz (dakle, postupak kojim smo od pretpostavka došli do konkluzije) \textit{nije bitan}.
Matematičarima su bitni samo iskaz teorema i činjenica da se teorem \textit{može dokazati}.
Sam postupak dokazivanja smatra se ``implementacijskim detaljom''.
Upravo zato analiza dokaza ima smisla samo kada se dokazuje, ali ne i kada se programira.
U našoj terminologiji to znači da se \textit{pattern matching} nad dokazima smije provoditi samo kod definiranja terma sorte \texttt{Prop}.
U suprotnom, mogli bismo definirati programe koji ovise o \textit{konkretnom dokazu}, umjesto o iskazanom teoremu.
\textbf{Ograničenje eliminacije propozicije} nastalo je radi ekstrakcije ---
svi termi sorte \texttt{Prop} se ``brišu'' prilikom prevođenja iz Coqovog tipskog sustava u tipske sustave niže razine apstrakcije.
Kada ovog ograničenja nebi bilo, ekstrakcija u jednostavnije jezike naprosto nebi bila moguća.


%%% Local Variables:
%%% mode: LaTeX
%%% TeX-master: "master"
%%% End:

\chapter{Logika prvog reda s induktivnim definicijama}\label{cha:logika-prvog-reda}
U ovom poglavlju predstavljamo glavne rezultate diplomskog rada koji uključuju formalizaciju
logike prvog reda s induktivnim definicijama \(\mathit{FOL_{ID}}\)
te dokaznog sustava \(\mathit{LKID}\), koje je prvi uveo Brotherston~\cite{brotherston2005}.
Definicije, leme i dokazi u ovom poglavlju preuzete su iz Brotherstonove disertacije~\cite{brotherstonphd}.

Prvo ćemo definirati sintaksu i semantiku logike \(\mathit{FOL_{ID}}\),
nakon čega ćemo definirati njene standardne modele.
Zatim ćemo prikazati dokazni sustav \(\mathit{LKID}\) te konačno dokazati adekvatnost
sustava \(LKID\) s obzirom na standardnu semantiku,
što je ujedno i glavni rezultat ovog diplomskog rada.

Svaka definicija i lema u ovom poglavlju bit će popraćena svojim pandanom u Coqu.
Jedan je od ciljeva diplomskog rada prikazati primjene Coqa u matematici,
zbog čega leme nećemo dokazivati \enquote{na papiru},
već se dokaz svake leme može pronaći na GitHub repozitoriju rada.\footnote{TODO: REPO LINK}
Zainteresiranom čitatelju predlažemo interaktivni prolazak kroz dokaze lema.

Prije no što krenemo na formalizaciju, valja prokomentirati odnos matematičkog i Coqovog vokabulara.
U matematici pojam \enquote{skup} može imati dva značenja;
prvo se odnosi na skupove kao \textit{domene diskursa},
dok se drugo odnosi na skupove kao \textit{predikate}, odnosno podskupove.
Primjerice, skup prirodnih brojeva \(\mathbb{N}\) je domena diskursa kada je riječ o svim prirodnim brojevima
te zbog toga pišemo \(n \in \mathbb{N}\) umjesto \(\mathbb{N}(n)\).
S druge strane, skup svih parnih brojeva \(E\) je podskup skupa \(\mathbb{N}\), a može se interpretirati
kao predikat na prirodnim brojevima te pišemo \(E(n)\) umjesto \(n \in E\).
U Coqu se skupovi kao domene diskursa formaliziraju tipovima sorte \coqset\footnote{Ili općenito kao tipovi sorte \texttt{Type}.},
dok se skupovi kao predikati formaliziraju funkcijama iz domene diskursa u sortu \coqprop.
Na primjer, tip prirodnih brojeva \texttt{nat} je sorte \texttt{Set}, a predikat \texttt{Nat.Even} je tipa \texttt{nat\(\rightarrow\)Prop}.

\section{Sintaksa}\label{sec:sintaksa}
Kao i u svakom izlaganju logike, na početku je potrebno definirati sintaksu.
\begin{definition}[Signatura]\label{def:signatura}
  \textit{Jezik prvog reda s induktivnim predikatima} (kratko signatura), u oznaci \(\Sigma\),
  je skup simbola od kojih razlikujemo \textit{funkcijske}, \textit{obične predikatne}
  i \textit{induktivne predikatne} simbole.
  Mjesnost simbola reprezentiramo funkcijom iz odgovarajućeg skupa simbola u skup \(\mathbb{N}\).
  Funkcijski simboli mjesnosti nula nazivaju se \textit{konstante},
  a predikatni simboli mjesnosti nula nazivaju se \textit{propozicije}.
  \begin{minted}{coq}
Structure signature := {
  FuncS : Set;
  fun_ar : FuncS -> nat;
  PredS : Set;
  pred_ar : PredS -> nat;
  IndPredS : Set;
  indpred_ar : IndPredS -> nat
}.
\end{minted}
\end{definition}

\begin{example}
  \misao{\(\Sigma_{PA}\)}
\end{example}

\noindent U ostatku poglavlja promatramo jednu proizvoljnu, ali fiksiranu signaturu \(\Sigma\).
Fiksiranje nekog proizvoljnog objekta je česta pojava u matematici,
prvenstveno zato što finksirane argumente ne trebamo spominjati eksplicitno.
Coq omogućuje fiksiranje naredbom \texttt{Context},
pod uvjetom da se korisnik nalazi u \texttt{Section} okolini.\footnote{\url{https://coq.inria.fr/doc/v8.18/refman/language/core/sections.html}}
Većina definicija i lema u ovom radu su napisane upravo unutar \texttt{Section} okoline.

\begin{definition}[Term]\label{def:term}
  \textit{Varijabla} je prirodan broj. Skup terma konstruiramo rekurzivno na način:
  \begin{enumerate}
  \item svaka varijabla je term;
  \item ako je \(f\) funkcijski simbol mjesnosti \(n\) te su \(t_{1}\), \ldots, \(t_{n}\) termi\footnote{Primijetimo, broj terma ovisi o mjesnosti funkcijskog simbola. U Coq implementaciji ovog \enquote{konstruktora} možemo vidjeti da je on zavisnog tipa.},
    onda je \(f(t_{1}, \ldots, t_{n})\) također term.
  \end{enumerate}
\begin{minted}{coq}
Inductive term : Set :=
| var_term : var -> term 
| TFunc : forall (f : FuncS Σ), vec term (fun_ar f) -> term.
\end{minted}
\end{definition}

\noindent Princip indukcije za \texttt{term} potrebno je ručno definirati.
Naime, induktivni tip \texttt{term} je \textit{ugniježđen} po konstruktoru \texttt{TFunc} što znači da se
javlja omotan oko drugog induktivnog tipa\footnote{Ovdje \texttt{vec}.} kao argument.
Za ugniježđene induktivne tipove, Coq generira \textit{neprikladne} principe indukcije.
\begin{minted}{coq}
Lemma term_ind
     : forall P : term Σ -> Prop,
         (forall v, P (var_term v)) ->
         (forall f args, (forall st, V.In st args -> P st) ->
                         P (TFunc f args)) ->
       forall t : term Σ, P t.
\end{minted}

\begin{definition}
  Skup svih varijabli koje se javljaju u termu \(t\), u oznaci \(\mathit{TV}(t)\), konstruiramo rekurzivno na način:
  \begin{enumerate}
  \item za varijablu \(v\) vrijedi \( \mathit{TV}(v) = \{ v \} \),
  \item za \(n\)-mjesni funkcijski simbol \(f\) i terme \(t_{1}, \ldots, t_{n}\) vrijedi \( \mathit{TV}(f(t_{1}, \ldots, t_{n})) = \bigcup_{1 \le i \le n} \mathit{TV}(t_{i}) \).
  \end{enumerate}
\begin{minted}{coq}
Inductive TV : term -> var -> Prop :=
| TVVar : forall v, TV (var_term v) v
| TVFunc : forall f args v st, V.In st args ->
                               TV st v -> TV (TFunc f args) v.
\end{minted}
\end{definition}

\begin{definition}[Formula]\label{def:formula}
  Skup formula konstruiramo rekurzivno na način:
  \begin{enumerate}
  \item ako je \(Q\) običan ili induktivan predikatni simbol mjesnosti \(n\) te su \(t_{1}\), \ldots, \(t_{n}\) termi,
    onda je \(Q(t_{1}, \ldots, t_{n})\) \textit{atomarna} formula;
  \item ako je \(\varphi\) formula, onda su \(\neg\varphi\) i \(\forall\varphi\) također formule;
  \item ako su \(\varphi\) i \(\psi\) formule, onda je \(\varphi \rightarrow \psi\) također formula.
  \end{enumerate}
\begin{minted}{coq}
Inductive formula : Set :=
| FPred (P : PredS Σ) : vec (term Σ) (pred_ar P) -> formula 
| FIndPred (P : IndPredS Σ) : vec (term Σ) (indpred_ar P) -> formula 
| FNeg : formula -> formula 
| FImp : formula -> formula -> formula 
| FAll : formula -> formula.
\end{minted}
\end{definition}
\noindent Za univerzalnu kvantifikaciju odstupamo od tradicionalne defincije formule.
Umjesto kvantificiranja po eksplicitnoj varijabli,
mi ćemo implicitno kvantificirati po varijabli \(0\).
Ovaj pristup kvantifikaciji\footnote{Ili općenitije, vezivanju varijabli.}, imena \enquote{de Bruijnovo indeksiranje},
bitno olakšava rad sa supstitucijama, a uveden je u članku~\cite{debruijnamelessdummies}.
O samoj implementaciji de Bruijnovog indeksiranja više se može pročitati u knjizi
\textit{Types and Programming Languages}~\cite{pierce2002tapl}.
Za potrebe ovog rada koristili smo program \textit{Autosubst2}\footnote{\url{https://github.com/uds-psl/autosubst2}}~\cite{stark-phd,autosubst2}
za automatsko generiranje tipova terma i formula te pripadajućih funkcija supstitucija i pomoćnih lema.
\misao{Ovaj odlomak preseliti nekamo na početak odjeljka.}

\begin{definition}
  Skup slobodnih varijabli formule \(\varphi\), u oznaci \(\mathit{FV}(\varphi)\), konstruiramo rekurzivno na način:
  \begin{enumerate}
  \item \( \mathit{FV}(P(u_{1}, \ldots, u_{n})) = \bigcup_{1 \le i \le n} \mathit{TV}(u_{i}) \),
  \item \( \mathit{FV}(\neg\varphi) = \mathit{FV}(\varphi)\),
  \item \( \mathit{FV}(\varphi \rightarrow \psi) = \mathit{FV}(\varphi) \cup \mathit{FV}(\psi) \),
  \item \( \mathit{FV(\forall \varphi)} = \{ v \mid 1 + v \in \mathit{FV}(\varphi) \} \).
  \end{enumerate}
\begin{minted}{coq}
Inductive FV : formula -> var -> Prop :=
| FV_Pred : forall R args v st,
    V.In st args -> TV st v -> FV (FPred R args) v
| FV_IndPred : forall R args v st,
    V.In st args -> TV st v -> FV (FIndPred R args) v
| FV_Imp_l : forall F G v, FV F v -> FV (FImp F G) v
| FV_Imp_r : forall F G v, FV G v -> FV (FImp F G) v
| FV_Neg : forall F v, FV F v -> FV (FNeg F) v
| FV_All : forall F v, FV F (S v) -> FV (FAll F) v.
\end{minted}
\end{definition}

\begin{definition}[Supstitucija]
  Supstitucija je svaka funkcija iz skupa \(\mathbb{N}\) u skup terma, a domena joj se može rekurzivno proširiti na skup terma i skup formula.
\begin{minted}{coq}
Fixpoint subst_term  (σ : var -> term) (t : term) : term :=
  match t with
  | var_term v => σ v
  | TFunc f args => TFunc f (V.map (subst_term σ) args)
  end.
\end{minted}
\begin{minted}{coq}
Fixpoint subst_formula
  (σ : var -> term Σ) (φ : formula )
  : formula  :=
  match φ return formula  with
  | FPred P args => FPred P (V.map (subst_term σ) args)
  | FIndPred P args => FIndPred P (V.map (subst_term σ) args)
  | FNeg ψ => FNeg (subst_formula σ ψ)
  | FImp ψ ξ => FImp (subst_formula σ ψ) (subst_formula σ ξ)
  | FAll ψ => FAll (subst_formula (up_term_term σ) ψ)
  end.
\end{minted}
\end{definition}

Konačno, potrebno je definirati sintaksu za indukciju.
U Coqu su definicije induktivnih propozicija proizvoljne do na ograničenje pozitivnosti,
no radi jednostavnosti u \(\mathit{FOL_{ID}}\) su moguće samo
induktivne definicije s atomarnim formulama, a pišemo ih u stilu prirodne dedukcije:
\begin{prooftree}
  \AxiomC{\(  Q_{1}\mathbf{u}_{1}  \ldots   Q_{n}\mathbf{u}_{n}  \)}
  \AxiomC{\(  P_{1}\mathbf{v}_{1}  \ldots   P_{m}\mathbf{v}_{m}  \)}
  \BinaryInfC{\(P\mathbf{t}\)}
\end{prooftree}
\noindent Ovdje su \(Q_{1}, \ldots, Q_{n}\) obični predikatni simboli, \(P_{1}, \ldots, P_{m}\) i \(P\) su induktivni predikatni simboli,
a podebljani znakovi predstavljaju vektore terma odgovarajućih duljina.

\begin{definition}[Produkcija]
  Uređene četvorke
  \begin{enumerate}
  \item listi parova običnih predikatnih simbola i vektora terma odgovarajućih duljina,
  \item listi parova induktivnih predikatnih simbola i vektora terma odgovarajućih duljina,
  \item induktivnog predikatnog simbola \(P\) mjesnosti \(n\) i
  \item vektora terma duljine \(n\)
  \end{enumerate}
  nazivamo produkcijama.
\begin{minted}{coq}
Record production := mkProd {
    preds : list ({ P : PredS Σ & vec (term Σ) (pred_ar P) });
    indpreds : list ({ P : IndPredS Σ & vec (term Σ) (indpred_ar P) });
    indcons : IndPredS Σ;
    indargs : vec (term Σ) (indpred_ar indcons);
  }.
\end{minted}
  \textit{Skup induktivnih definicija} je skup produkcija.
\begin{minted}{coq}
Definition IndDefSet := production -> Prop.
\end{minted}
\end{definition}


\section{Semantika}\label{sec:semantika}
\begin{definition}[Struktura]
  Struktura prvog reda \(M\) je uređena četvorka skupa kojeg nazivamo \textit{nosačem}
  te interpretacija funkcijskih, običnih predikatnih i induktivnih predikatnih simbola.
  Funkcijski se simboli interpretiraju kao \(n\)-mjesne funkcije, a predikatni simboli kao \(n\)-mjesne relacije na nosaču.
  Koristit ćemo ime strukture kao sinonim za njen nosač, a interpretacije označavamo sa \(f^{M}\) odnosno \(P^{M}\).
\begin{minted}{coq}
Structure structure := {
    domain :> Set;
    interpF (f : FuncS Σ) : vec domain (fun_ar f) -> domain;
    interpP (P : PredS Σ) : vec domain (pred_ar P) -> Prop;
    interpIP (P : IndPredS Σ) : vec domain (indpred_ar P) -> Prop;
  }.
\end{minted}
\end{definition}

\begin{definition}
  Neka je \(M\) proizvoljna struktura. \textit{Okolina} \(\rho\) za \(M\) je proizvoljna funkcija iz skupa prirodnih brojeva u nosač strukture.
\begin{minted}{coq}
Definition env := var -> M.
\end{minted}
  \noindent Okolina se može interpretirati kao niz \(d_{0}, d_{1}, d_{2}, \ldots\). Tada je \textit{pomaknuta okolina}, u oznaci \(d \cdot \rho\),
  niz \(d, d_{0}, d_{1}, d_{2}, \ldots\) za neki \(d \in M\).
  \noindent Proširenje domene okoline \(\rho\) na skup terma zovemo \textit{evaluacijom}.
\begin{minted}{coq}
Fixpoint eval (ρ : env) (t : term Σ) : M :=
  match t with
  | var_term x => ρ x
  | TFunc f args => interpF f (V.map (eval ρ) args)
  end.
\end{minted}
\end{definition}

\begin{definition}[Relacija ispunjivosti]
  Neka je \(M\) proizvoljna struktura te \(\rho\) okolina za \(M\).
  Ispunjivost formule \(\varphi\) u okolini \(\rho\) pišemo \(\rho \vDash \varphi\), a definiramo rekurzivno na način:
  \begin{enumerate}
  \item ako je \(P\) običan ili induktivan predikatni simbol mjesnosti \(n\) te su \(u_{1}, \ldots, u_{n}\) termi,
    onda vrijedi \(\rho \vDash P(u_{1}, \ldots, u_{n})\) ako i samo ako vrijedi \(P^{M}(\rho(u_{1}), \ldots, \rho(u_{n}))\),
  \item vrijedi \(\rho \vDash \neg \varphi \) ako i samo ako ne vrijedi \(\rho \vDash \varphi\),
  \item vrijedi \(\rho \vDash \varphi \rightarrow \psi \) ako i samo ako \(\rho \vDash \varphi\) povlači \(\rho \vDash \psi\) i
  \item vrijedi \(\rho \vDash \forall \varphi\) ako i samo ako za sve \(d \in M\) vrijedi \(d \cdot \rho \vDash \varphi\)
  \end{enumerate}
\begin{minted}{coq}
Fixpoint Sat (ρ : env M) (F : formula Σ) : Prop :=
  match F with
    | FPred P args => interpP P (V.map (eval ρ) args)
    | FIndPred P args => interpIP P (V.map (eval ρ) args)
    | FNeg G => ~ Sat ρ G
    | FImp F G => Sat ρ F -> Sat ρ G
    | FAll G => forall d, Sat (d .: ρ) G
  end.
\end{minted}
\end{definition}

\begin{lemma}
  Sintaktička i semantička supstitucija komutiraju pod relacijom ispunjivosti.
\begin{minted}{coq}
Lemma strong_form_subst_sanity2 :
  forall (φ : formula Σ) (σ : var -> term Σ)
      (M : structure Σ) (ρ : env M),
      ρ ⊨ (subst_formula σ φ) <-> (σ >> eval ρ) ⊨ φ.
\end{minted}
\end{lemma}
\misao{Zvuči kul. Je li jasno što je to semantička supstitucija i zašto kažem da komutiraju?}

\section{Standardni modeli}\label{sec:standardni-modeli}
Operator \(\varphi_{\Phi}\). Aproksimanti. Standardni model.

\section{Sistem sekvenata s induktivnim definicijama}\label{sec:sistem-sekvenata}
LKID. Dopustiva pravila. Primjeri dokaza.

\section{Adekvatnost}\label{sec:adekvatnost}
Lokalne adekvatnosti za pravila izvoda. Glavni teorem.

%%% Local Variables:
%%% mode: LaTeX
%%% TeX-master: "master"
%%% End:

\chapter{Sistem sekvenata s induktivnim definicijama}\label{sec:sistem-sekvenata}
Cilj je ovog odjeljka definirati dokazni sustav \(\mathit{LKID}\) za logiku \(\mathit{FOL_{ID}}\).
Ovaj dokazni sustav temelji se na Gentzenovu računu sekvenata, a proširen je lijevim i desnim pravilima za induktivne predikate.
Prije no što definiramo \(\mathit{LKID}\), potrebno je definirati pojam sekvente i međusobne zavisnosti induktivnih predikata.

\begin{definition}
  Neka su \(\Gamma\) i \(\Delta\) proizvoljne liste formula.
  Tada je \textit{sekventa} uređeni par \((\Gamma,\Delta)\), a označavamo ju s \(\Gamma \vdash \Delta\).
\begin{minted}{coq}
Inductive sequent : Set :=
| mkSeq (Γ Δ : list (formula Σ)).
\end{minted}
\end{definition}
\noindent Sekventom \(\Gamma \vdash \Delta\) konceptualno tvrdimo da istinitost \textit{svake}
tvrdnje u \(\Gamma\) povlači istinitost \textit{neke} tvrdnje u \(\Delta\).
Kažemo da je sekventa \(\Gamma \vdash \Delta\) \textit{dokaziva}, ili da \(\Gamma\) \textit{izvodi} \(\Delta\), ako u sustavu \(\mathit{LKID}\) postoji \textit{dokaz} za \(\Gamma \vdash \Delta\).
Neformalno možemo reći da je dokaz sekvente \(\Gamma \vdash \Delta\) konačno stablo u kojem je sekventa \(\Gamma \vdash \Delta\) korijen
te svaki čvor stabla slijedi iz svoje djece u skladu s pravilima izvoda sustava \(\mathit{LKID}\).
Kasnije ćemo precizirati pojam dokaza.

\begin{definition}
  Neka su \(P_{i}\) i \(P_{j}\) proizvoljni induktivni predikatni simboli.
  Kažemo da je simbol \(P_{i}\) u relaciji \(\mathit{Prem}\) sa simbolom \(P_{j}\)
  ako postoji produkcija u skupu induktivnih definicija \(\Phi\) takva da
  se simbol \(P_{i}\) javlja u njenoj konkluziji te
  se simbol \(P_{j}\) javlja u njenim premisama.
\begin{minted}{coq}
Definition Prem (Pi Pj : IndPredS Σ) :=
  exists pr, Φ pr /\
             indcons pr = Pi /\
             exists ts, List.In (Pj; ts) (indpreds pr).
\end{minted}
\end{definition}

\begin{definition}
  Definiramo relaciju \(\mathit{Prem}^{\ast}\) kao refleksivno i tranzitivno zatvorenje relacije \(\mathit{Prem}\) na skupu induktivnih predikatnih simbola.
\begin{minted}{coq}
Definition Prem_star := clos_refl_trans (IndPredS Σ) Prem.
\end{minted}
  \noindent Za induktivne predikatne simbole \(P\) i \(Q\) kažemo da su
  \textit{međusobno zavisni} ako vrijedi \(\mathit{Prem}^{\ast}(P, Q)\) i \(\mathit{Prem}^{\ast}(Q, P)\).
\begin{minted}{coq}
Definition mutually_dependent (P Q : IndPredS Σ) :=
  Prem_star P Q /\ Prem_star Q P.
\end{minted}
  \noindent Za međusobno zavisne induktivne predikatne simbole kažemo još da su
  definirani simultanom rekurzijom.
  Simultana rekurzija se koristi za formalnu definiciju međusobno zavisnih induktivnih predikata
  i u mnogo složenijim sustavima od \(\mathit{FOL}_{\mathit{ID}}\); primjerice,
  Coqovi predikati \texttt{EVEN} i \texttt{ODD} iz primjera~\ref{ex:structure}
  definirani su simultanom rekurzijom.
\end{definition}

\begin{example}
  U skupu induktivnih definicija \(\Phi_{\mathit{PA}}\),
  induktivni predikatni simboli \(\mathit{Even}\) i \(\mathit{Odd}\)
  su međusobno zavisni.
\begin{minted}{coq}
Example mut_dep_E_O :
  mutually_dependent Φ__PA PA_Even PA_Odd.
\end{minted}
  \noindent S druge strane, induktivni predikatni simboli \(\mathit{Nat}\) i \(\mathit{Even}\)
  \textit{nisu} međusobno zavisni.
\begin{minted}{coq}
Example not_mut_dep_N_E :
  ~mutually_dependent Φ__PA PA_Nat PA_Even.
\end{minted}
\end{example}

\begin{lemma}
  Međusobna zavisnost je relacija ekvivalencije na skupu induktivnih predikatnih simbola.
\begin{minted}{coq}
Lemma mutually_dependent_equiv : Equivalence mutually_dependent.
\end{minted}
\end{lemma}

Sustav \(\mathit{LKID}\) sastoji se od četiri vrste pravila izvoda:
strukturalna, propozicijska, pravila za kvantifikatore i pravila za induktivne predikate.
Navest ćemo ih tim redom, a definirati kroz više blokova Coqova koda.
Napominjemo da se sustav \(\mathit{LKID}\) razlikuje od Gentzenova sustava \(\mathit{LK}\)
na nekoliko mjesta, a zbog implementacijskih detalja te zbog induktivnih definicija.
Za neko pravilo izvoda kažemo da je \textit{dopustivo} ako nije u definiciji sustava \(\mathit{LKID}\),
ali je iz originalnih pravila izvedivo.
\begin{minted}{coq}
Inductive LKID : sequent -> Prop := (* nastavlja se *)
\end{minted}
\section{Strukturalna pravila}
Strukturalna su pravila prikazana na slici~\ref{fig:lkid-structural}, a služe baratanju strukturom sekvente.
Iako je sekventa implementirana kao par listi formula, te liste se ponašaju kao skupovi u strukturalnim pravilima
jer za sadržanost jedne liste u drugoj koristimo Coqov predikat \texttt{incl} koji se oslanja na predikat \texttt{In}.
Posljedično, pravilo slabljenja \(\mathit{Wk}\) povlači pravila permutacije i kontrakcije.
\begin{figure}[!htb]
  \centering

  \begin{prooftree}
    \AxiomC{\(\Gamma \cap \Delta \not = \varnothing \)}
    \RightLabel{\( \mathit{ (Ax) } \)}
    \UnaryInfC{\( \Gamma \vdash \Delta \)}
  \end{prooftree}

  \begin{prooftree}
    \AxiomC{\(\Gamma^{\prime} \vdash \Delta^{\prime}\)}
    \AxiomC{\(\Gamma^{\prime} \subseteq \Gamma\)}
    \AxiomC{\(\Delta^{\prime} \subseteq \Delta\)}
    \RightLabel{\( \mathit{ (Wk) } \)}
    \TrinaryInfC{\(\Gamma \subseteq \Delta\)}
  \end{prooftree}
  
  \begin{prooftree}
    \AxiomC{\( \Gamma \vdash \varphi, \Delta\)}
    \AxiomC{\( \varphi, \Gamma \vdash \Delta \)}
    \RightLabel{\( \mathit{ (Cut) } \)}
    \BinaryInfC{\( \Gamma \vdash \Delta \)}
  \end{prooftree}

  \begin{prooftree}
    \AxiomC{\( \Gamma \vdash \Delta \)}
    \RightLabel{\( \mathit{ (Subst) } \)}
    \UnaryInfC{\( \Gamma[\sigma] \vdash \Delta[\sigma] \)}
  \end{prooftree}
  
  \caption{Strukturalna pravila sustava \(\mathit{LKID}\).}\label{fig:lkid-structural}
\end{figure}
\begin{minted}[firstnumber=last]{coq}
(* Strukturalna pravila. *)
| Ax : forall Γ Δ φ, In φ Γ -> In φ Δ -> LKID (Γ ⊢ Δ)
| Wk : forall Γ' Δ' Γ Δ,
    Γ' ⊆ Γ ->
    Δ' ⊆ Δ ->
    LKID (Γ' ⊢ Δ') ->
    LKID (Γ ⊢ Δ)
| Cut : forall Γ Δ φ,
    LKID (Γ ⊢ φ :: Δ) ->
    LKID (φ :: Γ ⊢ Δ) ->
    LKID (Γ ⊢ Δ)
| Subst : forall Γ Δ,
    LKID (Γ ⊢ Δ) ->
    forall σ, LKID (map (subst_formula σ) Γ ⊢ map (subst_formula σ) Δ)
\end{minted}

\begin{example}
  Prošireno pravilo aksioma je dopustivo.
  \begin{prooftree}
    \AxiomC{}
    \RightLabel{\( \mathit{ (AxExtended) } \)}
    \UnaryInfC{\( \varphi, \Gamma \vdash \varphi, \Delta \)}
  \end{prooftree}
\begin{minted}[linenos=false]{coq}
Lemma AxExtended : forall Γ Δ φ, LKID (φ :: Γ ⊢ φ :: Δ).
\end{minted}
  \noindent Nadalje, pravilo permutacije je dopustivo.
  \begin{prooftree}
    \AxiomC{\( \Gamma^{\prime} \vdash \Delta^{\prime} \)}
    \AxiomC{\( \Gamma^{\prime} \sim \Gamma \)}
    \AxiomC{\( \Delta^{\prime} \sim \Delta \)}
    \RightLabel{\(\mathit{ (Perm) }\)}
    \TrinaryInfC{\( \Gamma \vdash \Delta \)}
  \end{prooftree}
  \noindent Ovdje znakom \(\sim\) označujemo relaciju permutacije na listama.
\begin{minted}[linenos=false]{coq}
Lemma Perm : forall Γ' Δ' Γ Δ,
    Permutation Γ' Γ ->
    Permutation Δ' Δ ->
    LKID (Γ' ⊢ Δ') ->
    LKID (Γ ⊢ Δ).
\end{minted}
  \noindent Konačno, lijeva i desna pravila kontrakcije su dopustiva.
  \begin{prooftree}
    \AxiomC{\( \varphi, \varphi, \Gamma \vdash \Delta \)}
    \RightLabel{\(\mathit{(ContrL)}\)}
    \UnaryInfC{\( \varphi, \Gamma \vdash \Delta \)}
  \end{prooftree}
  \begin{prooftree}
    \AxiomC{\( \Gamma \vdash \varphi, \varphi, \Delta \)}
    \RightLabel{\(\mathit{(ContrR)}\)}
    \UnaryInfC{\( \Gamma \vdash \varphi, \Delta \)}
  \end{prooftree}
\begin{minted}[linenos=false]{coq}
Lemma ContrL : forall Γ Δ φ,
    LKID (φ :: φ :: Γ ⊢ Δ) -> LKID (φ :: Γ ⊢ Δ).
Lemma ContrR : forall Γ Δ φ,
    LKID (Γ ⊢ φ :: φ :: Δ) -> LKID (Γ ⊢ φ :: Δ).
\end{minted}
\end{example}

\section{Propozicijska pravila}
Propozicijska su pravila prikazana na slici~\ref{fig:lkid-propositional}, a služe rasuđivanju s implikacijama i negacijama.
Ova su pravila identična u sustavima \(\mathit{LK}\) i \(\mathit{LKID}\).
\begin{figure}[!htb]
  \centering
  \begin{prooftree}
    \AxiomC{\( \Gamma \vdash \varphi, \Delta \)}
    \RightLabel{\( \mathit{ (NegL) } \)}
    \UnaryInfC{\( \neg \varphi, \Gamma \vdash \Delta \)}
  \end{prooftree}

  \begin{prooftree}
    \AxiomC{\( \varphi, \Gamma \vdash \Delta \)}
    \RightLabel{\( \mathit{ (NegR) } \)}
    \UnaryInfC{\( \Gamma \vdash \neg \varphi, \Delta \)}
  \end{prooftree}

  \begin{prooftree}
    \AxiomC{\( \Gamma \vdash \varphi, \Delta \)}
    \AxiomC{\( \psi, \Gamma \vdash \Delta \)}
    \RightLabel{\( \mathit{ (ImpL) } \)}
    \BinaryInfC{\( \varphi \rightarrow \psi, \Gamma \vdash \Delta \)}
  \end{prooftree}

  \begin{prooftree}
    \AxiomC{\( \varphi, \Gamma \vdash \psi, \Delta \)}
    \RightLabel{\( \mathit{ (ImpR) } \)}
    \UnaryInfC{\( \Gamma \vdash \varphi \rightarrow \psi, \Delta \)}
  \end{prooftree}
  
  \caption{Propozicijska pravila sustava \(\mathit{LKID}\).}\label{fig:lkid-propositional}
\end{figure}
\begin{minted}[firstnumber=16]{coq}
(* Propozicijska pravila. *)
| NegL : forall Γ Δ φ, LKID (Γ ⊢ φ :: Δ) -> LKID (FNeg φ :: Γ ⊢ Δ)
| NegR : forall Γ Δ φ, LKID (φ :: Γ ⊢ Δ) -> LKID (Γ ⊢ FNeg φ :: Δ)
| ImpL : forall Γ Δ φ ψ,
    LKID (Γ ⊢ φ :: Δ) -> LKID (ψ :: Γ ⊢ Δ) ->
    LKID (FImp φ ψ :: Γ ⊢ Δ)
| ImpR : forall Γ Δ φ ψ,
    LKID (φ :: Γ ⊢ ψ :: Δ) -> LKID (Γ ⊢ (FImp φ ψ) :: Δ)
\end{minted}

\begin{example}
  Desno pravilo za konjunkciju je dopustivo.
  \begin{prooftree}
    \AxiomC{\( \Gamma \vdash \varphi, \Delta \)}
    \AxiomC{\( \Gamma \vdash \psi, \Delta \)}
    \RightLabel{\( \mathit{ (AndR) } \)}
    \BinaryInfC{\( \Gamma \vdash \varphi \land \psi, \Delta \)}
  \end{prooftree}
\begin{minted}[linenos=false]{coq}
Lemma AndR : forall Γ Δ φ ψ,
    LKID (Γ ⊢ φ :: Δ) -> LKID (Γ ⊢ ψ :: Δ) ->
    LKID (Γ ⊢ FAnd φ ψ :: Δ).
\end{minted}
  \noindent Lijevo pravilo za disjunkciju je dopustivo.
  \begin{prooftree}
    \AxiomC{\( \varphi, \Gamma \vdash \Delta \)}
    \AxiomC{\( \psi, \Gamma \vdash \Delta \)}
    \RightLabel{\( \mathit{ (OrL) } \)}
    \BinaryInfC{\( \varphi \lor \psi, \Gamma \vdash \Delta \)}
  \end{prooftree}
\begin{minted}[linenos=false]{coq}
Lemma OrL : forall Γ Δ φ ψ,
    LKID (φ :: Γ ⊢ Δ) -> LKID (ψ :: Γ ⊢ Δ) ->
    LKID (FOr φ ψ :: Γ ⊢ Δ).
\end{minted}
\end{example}

\section{Pravila za kvantifikatore}
Pravila za kvantifikatore prikazana su na slici~\ref{fig:lkid-quantificational}, a odnose se na univerzalnu kvantifikaciju.
Ova se pravila razlikuju od analognih u sustavu \(\mathit{LK}\) zbog korištenja de Bruijnovih indeksa kod kvantifikacije.
U lijevom pravilu koristimo supstituciju \(t \cdot \sigma_{\mathit{id}}\) u formuli \(\varphi\), gdje je \(\sigma_{\mathit{id}}\) supstitucija identiteta,
kako bismo iskazali da se term \(t\) javlja na mjestu neke varijable \(x\).
Za desno pravilo, primijetimo da se varijabla \(0\) ne javlja u listama formula \(\Gamma^{\uparrow}\) i \(\Delta^{\uparrow}\) zbog pomicanja.
Kako implicitno kvantificiramo po varijabli \(0\), a  ona se javlja
(eventualno) samo u formuli \(\varphi\), smijemo univerzalno kvantificirati formulu \(\varphi\).

\begin{figure}[!htb]
  \centering
  \begin{prooftree}
    \AxiomC{\( \varphi[t \cdot \sigma_{\mathit{id}}], \Gamma \vdash \Delta \)}
    \RightLabel{\( \mathit{ (AllL) } \)}
    \UnaryInfC{\( \forall\varphi, \Gamma \vdash \Delta \)}
  \end{prooftree}

  \begin{prooftree}
    \AxiomC{\( \Gamma^{\uparrow} \vdash \varphi, \Delta^{\uparrow}\)}
    \RightLabel{\( \mathit{ (AllR) } \)}
    \UnaryInfC{\( \Gamma \vdash \forall\varphi, \Delta \)}
  \end{prooftree}
  \caption{Pravila za kvantifikatore sustava \(\mathit{LKID}\).}\label{fig:lkid-quantificational}
\end{figure}
\begin{minted}[firstnumber=24]{coq}
(* Kvantifikatorska pravila. *)
| AllL : forall Γ Δ φ t,
    LKID (subst_formula (t .: ids) φ :: Γ ⊢ Δ) -> 
    LKID (FAll φ :: Γ ⊢ Δ)
| AllR : forall Γ Δ φ,
    LKID (shift_formulas Γ ⊢ φ :: shift_formulas Δ) ->
    LKID (Γ ⊢ (FAll φ) :: Δ)
\end{minted}

\begin{example}
  Lijevo pravilo za egzistencijalnu kvantifikaciju je dopustivo.
  \begin{prooftree}
    \AxiomC{\( \varphi, \Gamma^{\uparrow} \vdash \Delta^{\uparrow} \)}
    \RightLabel{\( \mathit{ (ExistL) } \)}
    \UnaryInfC{\( \exists\varphi, \Gamma \vdash \Delta\)}
  \end{prooftree}
\begin{minted}[linenos=false]{coq}
Lemma ExistL : forall Γ Δ φ,
    LKID (φ :: shift_formulas Γ ⊢ shift_formulas Δ) ->
    LKID (FExist φ :: Γ ⊢ Δ).
\end{minted}
\end{example}

Preostaje nam definicija pravila za induktivne predikatne simbole,
gdje ćemo prvo za svaki induktivni predikatni simbol
definirati lijevo pravilo koje nazivamo \textit{pravilo indukcije},
a zatim ćemo za svaku produkciju
definirati desno pravilo koje nazivamo \textit{produkcijsko pravilo}.

\section{Pravila indukcije}
Neka je \(P_{j}\) proizvoljni induktivan predikatni simbol za koji definiramo pravilo indukcije.
Svakom induktivnom predikatnom simbolu \(P_{i}\) dodjeljujemo
proizvoljnu \(|P_{i}|\)-torku \textit{induktivnih varijabli} \(\mathbf{z}_{i}\) te
proizvoljnu \textit{pretpostavku indukcije} \(G_{i}\) takvu da
vrijedi \(G_{i} = P_{i}(\mathbf{z}_{i})\) ako \(P_{i}\) i \(P_{j}\) nisu međusobno zavisni.
Sada pravilo indukcije za induktivan predikatni simbol \(P_{j}\) glasi:
\begin{prooftree}
  \AxiomC{\textit{male premise}}
  \AxiomC{\( G_{j} [\mathbf{u} / \mathbf{z}_{i}], \Gamma  \vdash \Delta\)}
  \RightLabel{\( \mathit{ (P_{j} \, Ind), } \)}
  \BinaryInfC{\( P_{j} \mathbf{u}, \Gamma  \vdash \Delta \)}
\end{prooftree}
\noindent gdje sekventu \( G_{j} [\mathbf{u} / \mathbf{z}_{i}] , \Gamma  \vdash \Delta\) zovemo \textit{velikom premisom},
a za svaku produkciju u skupu induktivnih definicija \(\Phi\) u čijoj se konkluziji javlja
induktivni predikatni simbol \(P_{i}\) međusobno zavisan sa simbolom \(P_{j}\), primjerice:
\begin{prooftree}
  \AxiomC{\(  Q_{1}\mathbf{u}_{1}  \ldots   Q_{n}\mathbf{u}_{n}  \)}
  \AxiomC{\(  P_{i_{1}}\mathbf{v}_{i_{1}}  \ldots   P_{i_{m}}\mathbf{v}_{i_{m}}  \)}
  \RightLabel{,}
  \BinaryInfC{\(P_{i}\mathbf{t}\)}
\end{prooftree}
\noindent postoji \textit{mala premisa} oblika:
\[
  Q_{1} \mathbf{u}_{1}^{\uparrow k}, \ldots, Q_{n} \mathbf{u}_{n}^{\uparrow k},
  G_{i_{1}}[ \mathbf{v}_{i_{1}}^{\uparrow k} / \mathbf{z}_{i_{1}} ], \ldots, G_{i_{m}}[ \mathbf{v}_{i_{m}}^{\uparrow k} / \mathbf{z}_{i_{m}} ],
  \Gamma \vdash G_{i}[\mathbf{t}^{\uparrow k} / \mathbf{z}_{i}], \Delta
\]
\noindent gdje je \(k\) neki prirodan broj koji se ne javlja u listama formula \(\Gamma\) i \(\Delta\) te formuli \(P_{j}\mathbf{u}\),
a s \(^{\uparrow k}\) označavamo supstituciju koja svaku varijablu \(n\) zamjenjuje varijablom \(n + k\).


Ilustrirajmo pravilo indukcije primjerima u Peanovoj aritmetici.
Kako se pravila za induktivne predikate sustava \(\mathit{LKID}\) odnose na proizvoljan skup induktivnih definicija \(\Phi\),
označimo s \(\mathit{LKID}_{\mathit{PA}}\)  sustav koji se odnosi na skup induktivnih defincija \(\Phi_{\mathit{PA}}\).
\begin{minted}{coq}
Definition LKID__PA := LKID Φ__PA.
\end{minted}
\begin{example}
  Pravilo indukcije simbola \(\mathit{Nat}\)
  u sustavu \(\mathit{LKID}_{\mathit{PA}}\) glasi:
  \begin{prooftree}
    \AxiomC{\( \Gamma \vdash G(o), \Delta \)}
    \AxiomC{\( G(x), \Gamma \vdash G(s(x)), \Delta \)}
    \AxiomC{\( G(t), \Gamma \vdash \Delta \)}
    \RightLabel{\( \mathit{(NatInd),} \)}
    \TrinaryInfC{\( \mathit{Nat}(t), \Gamma \vdash \Delta \)}
  \end{prooftree}
  \noindent gdje je G pretpostavka indukcije vezana uz induktivni predikatni simbol \(\mathit{Nat}\),
  a \(x\) je varijabla koja se ne javlja u \(\Gamma\), \(\Delta\) niti u \(\mathit{Nat}(t)\).
\end{example}

\begin{example}
  Pravilo indukcije simbola  \(\mathit{Even}\)
  u sustavu \(\mathit{LKID}_{\mathit{PA}}\) glasi:
  \begin{small}
    \begin{prooftree}
      \AxiomC{\( \Gamma \vdash G_{E}(o), \Delta \quad \Gamma, G_{E}(x) \vdash G_{O}(s(x)), \Delta \quad \Gamma, G_{O}(x) \vdash G_{E}(s(x)), \Delta \)}
      \AxiomC{\( \Gamma, G_{E}(t) \vdash \Delta \)}
      \RightLabel{\(\mathit{(EvenInd),}\)}
      \BinaryInfC{\( \Gamma, \mathit{Even}(t) \vdash \Delta \)}
    \end{prooftree}
  \end{small}
  \noindent gdje su \(G_{E}\) i \(G_{O}\) pretpostavke indukcije vezane uz
  induktivne predikatne simbole \(\mathit{Even}\) i \(\mathit{Odd}\),
  a \(x\) je varijabla koja se ne javlja u \(\Gamma\), \(\Delta\) ili \(\mathit{Even}(t)\).
\end{example}

\begin{minted}[firstnumber=31]{coq}
(* Pravila indukcije. *)
| IndL : forall Γ Δ
           (Pj : IndPredS Σ) (u : vec _ (indpred_ar Pj))
           (z_i : forall P, vec var (indpred_ar P))
           (z_i_nodup : forall P, VecNoDup (z_i P))
           (G_i : IndPredS Σ -> formula Σ)
           (HG2 : forall Pi, ~mutually_dependent Pi Pj ->
                        G_i Pi = FIndPred
                                   Pi
                                   (V.map var_term (z_i Pi))),
    let maxΓ := max_fold (map some_var_not_in_formula Γ) in
    let maxΔ := max_fold (map some_var_not_in_formula Δ) in
    let maxP := some_var_not_in_formula (FIndPred Pj u) in
    let shift_factor := max maxP (max maxΓ maxΔ) in
    let Fj := subst_formula
                (finite_subst (z_i Pj) u)
                (G_i Pj) in
    let minor_premises :=
      (forall pr (HΦ:Φ pr) (Hdep: mutually_dependent (indcons pr) Pj),
          let Qs := shift_formulas_by
                      shift_factor
                      (FPreds_from_preds (preds pr)) in
          let Gs := map (fun '(P; args) =>
                           let shifted_args :=
                             V.map
                               (shift_term_by shift_factor)
                               args in
                           let σ :=
                             finite_subst
                               (z_i P)
                               (shifted_args) in
                           let G := G_i P in
                           subst_formula σ G)
                      (indpreds pr) in
          let Pi := indcons pr in
          let ty := V.map
                      (shift_term_by shift_factor)
                      (indargs pr) in
          let Fi := subst_formula
                      (finite_subst (z_i Pi) ty)
                      (G_i Pi) in
          LKID (Qs ++ Gs ++ Γ ⊢ Fi :: Δ))
    in
    minor_premises ->
    LKID (Fj :: Γ ⊢ Δ) ->
    LKID (FIndPred Pj u :: Γ ⊢ Δ)
\end{minted}

\section{Produkcijska pravilo}
Neka je \(pr\) proizvoljna produkcija za koju definiramo produkcijsko pravilo.
Možemo pretpostaviti da je \(pr\) oblika:
\begin{prooftree}
  \AxiomC{\(  Q_{1}\mathbf{u}_{1}  \ldots   Q_{n}\mathbf{u}_{n}  \)}
  \AxiomC{\(  P_{1}\mathbf{v}_{1}  \ldots   P_{m}\mathbf{v}_{m}  \)}
  \RightLabel{.}
  \BinaryInfC{\(P\mathbf{t}\)}
\end{prooftree}
\noindent Tada produkcijsko pravilo za produkciju \(pr\) glasi:
\begin{prooftree}
  \AxiomC{\( \Gamma \vdash Q_{1} \mathbf{u}_{1}[\sigma], \Delta \quad \ldots \quad \Gamma \vdash Q_{n}\mathbf{u}_{n}[\sigma]\)}
  \AxiomC{\( \Gamma \vdash P_{1} \mathbf{v}_{1}[\sigma], \Delta \quad \ldots \quad \Gamma \vdash P_{m} \mathbf{v}_{m}[\sigma] \)}
  \RightLabel{\( \mathit{ (prProd) }\).}
  \BinaryInfC{\( \Gamma \vdash P \mathbf{t}[\sigma] \)}
\end{prooftree}
\begin{minted}[firstnumber=58]{coq}
(* Produkcijska pravila. *)
| IndR : forall Γ Δ pr σ,
    Φ pr ->
    (forall Q us,
        In (Q; us) (preds pr) ->
        LKID (Γ ⊢ (FPred Q (V.map (subst_term σ) us) :: Δ))) ->
    (forall P ts,
        In (P; ts) (indpreds pr) ->
        LKID (Γ ⊢ (FIndPred P (V.map (subst_term σ) ts) :: Δ))) ->
    LKID ( Γ ⊢ FIndPred
             (indcons pr)
             (V.map (subst_term σ) (indargs pr))
             :: Δ).
\end{minted}

\begin{example}
  Produkcijsko pravilo za produkciju \texttt{PA\_prod\_E\_succ} glasi:
  \begin{prooftree}
    \AxiomC{\( \Gamma \vdash \mathit{Odd}(x)[\sigma], \Delta \)}
    \RightLabel{(\texttt{PA\_prod\_E\_succ}\textit{Prod})}
    \UnaryInfC{\( \Gamma \vdash \mathit{Even}(s(x))[\sigma], \Delta \)}
  \end{prooftree}
  \noindent gdje je \(x\) proizvoljna varijabla, a \(\sigma\) proizvoljna supstitucija.
  Supstitucijom \(\sigma\) kažemo da u produkcijama imena varijabli nisu bitna,
  već samo njihov redoslijed.
\end{example}

\section{Dokaz i teorem}
Prikazali smo sva pravila izvoda sustava \(\mathit{LKID}\) te konačno možemo definirati dokaz.
Za proizvoljnu sekventu \(s \coloneq (\Gamma \vdash \Delta)\), njen \textbf{dokaz} je stanovnik tipa \texttt{LKID s}.
Za formulu \(\varphi\) kažemo da je \textbf{dokaziva} u sustavu \(\mathit{LKID}\) te
pišemo \(\vdash \varphi\) ako postoji dokaz sekvente \( [] \vdash [\varphi]\).
\begin{minted}{coq}
Definition provable (proofsystem : sequent -> Prop) (φ : formula Σ) :=
  proofsystem ([] ⊢ [φ]).
\end{minted}
\noindent Neformalni pojam konačnog stabla formaliziran je induktivnim tipom \texttt{LKID},
dok njegovi konstruktori formaliziraju odgovarajuća pravila izvoda.

\begin{example}
  Za proizvoljnu formulu \(\varphi\), formula \(\varphi \lor \neg \varphi\) je dokaziva u sustavu \(\mathit{LKID}\).
\begin{minted}{coq}
Lemma LKID_XM : forall φ, provable LKID (FOr φ (FNeg φ)).
\end{minted}
  \begin{prooftree}
    \AxiomC{}
    \RightLabel{\(\mathit{(Ax)}\)}
    \UnaryInfC{\( \varphi \vdash \varphi\)}
    \RightLabel{\(\mathit{(NegR)}\)}
    \UnaryInfC{\( \vdash \neg\varphi, \varphi\)}
    \RightLabel{\(\mathit{(Perm)}\)}
    \UnaryInfC{\( \vdash \varphi, \neg\varphi\)}
    \RightLabel{\(\mathit{(OrR)}\)}
    \UnaryInfC{\( \vdash \varphi \lor \neg\varphi\)}
  \end{prooftree}
\end{example}

\begin{example}
  Za sustav \(\mathit{LKID}\) vrijedi \textit{princip eksplozije}.
\begin{minted}{coq}
Lemma LKID_EXPLOSION : forall φ Δ, LKID ([FAnd φ (FNeg φ)] ⊢ Δ).
\end{minted}
  \begin{prooftree}
    \AxiomC{}
    \RightLabel{\(\mathit{(Ax)}\)}
    \UnaryInfC{\(\varphi \vdash \varphi, \Delta\)}
    \RightLabel{\(\mathit{(NegL)}\)}
    \UnaryInfC{\(\neg \varphi, \varphi \vdash \Delta\)}
    \RightLabel{\(\mathit{(Perm)}\)}
    \UnaryInfC{\(\varphi, \neg\varphi \vdash \Delta\)}
    \RightLabel{\(\mathit{(AndL)}\)}
    \UnaryInfC{\(\varphi \land \neg \varphi \vdash \Delta\)}
  \end{prooftree}
\end{example}

\begin{example}\label{ex:lkid-indr-proof}
  Formula \(\forall x, \mathit{Even}(x) \rightarrow \mathit{Even}(s(s(x)))\) je dokaziva u sustavu \(\mathit{LKID}_{\mathit{PA}}\).
\begin{minted}{coq}
Definition Even_succ_succ_Even : formula Σ__PA :=
  let x := var_term 0 in
  FAll
    (FImp
       (FIndPred PA_Even [x])
       (FIndPred PA_Even [TFunc PA_succ
                            [TFunc PA_succ
                               [x]]])).
\end{minted}
  \noindent Neformalni dokaz prikazujemo idućim stablom, a radi preglednosti
  pišemo \(Et\) umjesto \(\mathit{Even}(t)\) te \(Ot\) umjesto \(\mathit{Odd}(t)\)
  i ne pišemo zagrade u pozivu funkcijskog simbola \(s\).
  \begin{prooftree}
    \AxiomC{}
    \RightLabel{\( \mathit{(Ax)} \)}
    \UnaryInfC{\( Ex \vdash Ex, Essx \)}
    \RightLabel{\( \mathit{(} \mathtt{PA\_prod\_O\_succ} \mathit{Prod)} \)}
    \UnaryInfC{\( Ex \vdash Osx, Essx \)}
    \AxiomC{}
    \RightLabel{\( \mathit{(Ax)} \)}
    \UnaryInfC{\( Ex, Osx \vdash Osx \)}
    \RightLabel{\( \mathit{(} \mathtt{PA\_prod\_E\_succ}\mathit{Prod)} \)}
    \UnaryInfC{\( Ex, Osx \vdash Essx \)}
    \RightLabel{\( \mathit{(Cut)} \)}
    \BinaryInfC{\( Ex \vdash Essx \)}
    \RightLabel{\( \mathit{(ImpR)} \)}
    \UnaryInfC{\(  \vdash Ex \rightarrow Essx \)}
    \RightLabel{\( \mathit{(AllR)} \)}
    \UnaryInfC{\( \vdash \forall x, Ex \rightarrow Essx \)}
  \end{prooftree}
\begin{minted}{coq}
Lemma provable_every_succ_of_Even_is_Odd :
  provable LKID__PA every_succ_of_Even_is_Odd.
\end{minted}
\end{example}

\begin{example}\label{ex:lkid-indl-proof}
  Formula \(\forall x, \mathit{Nat}(x) \rightarrow \mathit{Even}(x) \lor \mathit{Odd}(x)\) je dokaziva u sustavu \(\mathit{LKID}_{\mathit{PA}}\).
  Neformalni dokaz prikazujemo idućim stablom koristeći iste pokrate kao u
  primjeru~\ref{ex:lkid-indr-proof}, te \(Nt\) za \(\mathit{Nat}(t)\).
  \begin{scriptsize}
    \begin{prooftree}
      \AxiomC{}
      \RightLabel{\(\mathit{(1)}\)}
      \UnaryInfC{\( \vdash Eo \)}
      \RightLabel{\(\mathit{(Wk)}\)}
      \UnaryInfC{\( \vdash Eo, Ex \lor Ox\)}
      \RightLabel{\(\mathit{(OrR_{1})}\)}
      \UnaryInfC{\(\vdash Eo \lor Oo, Ex \lor Ox\)}
      \AxiomC{}
      \RightLabel{\(\mathit{(Ax)}\)}
      \UnaryInfC{\( Ey \vdash Ey\)}
      \RightLabel{\(\mathit{(2)}\)}
      \UnaryInfC{\(Ey \vdash Osy\)}
      \RightLabel{\(\mathit{(OrR_{2})}\)}
      \UnaryInfC{\( Ey \vdash Esy \lor Osy \)}
      \AxiomC{}
      \RightLabel{\(\mathit{(Ax)}\)}
      \UnaryInfC{\(Oy \vdash Oy\)}
      \RightLabel{\(\mathit{(3)}\)}
      \UnaryInfC{\(Oy \vdash Esy\)}
      \RightLabel{\(\mathit{(OrR_{1})}\)}
      \UnaryInfC{\(Oy \vdash Esy \lor Osy\)}
      \RightLabel{\(\mathit{(OrL)}\)}
      \BinaryInfC{\( Ey \lor Oy \vdash Esy \lor Osy \)}
      \RightLabel{\(\mathit{(Wk)}\)}
      \UnaryInfC{\( Ey \lor Oy \vdash Esy \lor Osy, Ex \lor Ox \)}
      \AxiomC{}
      \RightLabel{\(\mathit{(Ax)}\)}
      \UnaryInfC{\( Ex \lor Ox \vdash Ex \lor Ox\)}
      \RightLabel{\(\mathit{(NatInd)}\)}
      \TrinaryInfC{\( Nx \vdash Ex \lor Ox \)}
      \RightLabel{\( \mathit{(ImpR)}\)}
      \UnaryInfC{\( Nx \rightarrow Ex \lor Ox \)}
      \RightLabel{\(\mathit{(AllR)}\)}
      \UnaryInfC{\( \vdash \forall x, Nx \rightarrow Ex \lor Ox \)}
    \end{prooftree}
  \end{scriptsize}
  \noindent Ovdje je pretpostavka indukcije formula \(Ex \lor Ox\).
  Valja primijetiti da smo u dokazu koristili (označene brojevima) i produkcijska pravila za produkcije:
  \begin{enumerate}[label={(\arabic*)}]
  \item \texttt{PA\_prod\_E\_zero},
  \item \texttt{PA\_prod\_O\_succ} i
  \item \texttt{PA\_prod\_E\_succ}.
  \end{enumerate}
\begin{minted}{coq}
Lemma provable_every_nat_is_even_or_odd :
  provable LKID__PA every_nat_is_even_or_odd.
\end{minted}
\end{example}

\section{Adekvatnost}\label{sec:adekvatnost}
Za dokazivače teorema nužno je da dokazivost formule povlači njenu istinitost ---
u~\ref{chap:coq} poglavlju objasnili smo zašto.
Ovo svojstvo dokaznog sustava naziva se \textit{adekvatnost},
a ovdje ćemo je prikazati na primjeru sustava \(\mathit{LKID}\).
Za istinitost ćemo promatrati samo strukture koje su standardni modeli,
jer u takvima interpretacije induktivnih predikatnih simbola imaju smisla.
Napomenimo da je \(\mathit{LKID}\) adekvatan i za nestandardne modele.

\begin{definition}
  Kažemo da je sekventa \(\Gamma \vdash \Delta\) \textit{istinita} i pišemo \(\Gamma \VDash \Delta\) ako i samo ako
  za svaki standardni model \(M\) i njegovu okolinu \(\rho\) vrijedi:
  \[
    (\forall \varphi, \varphi \in \Gamma \rightarrow \rho \vDash \varphi) \rightarrow \exists \psi, \psi \in \Delta \land \rho \vDash \psi,
  \]
  \noindent odnosno istinitost svake formule u \(\Gamma\) povlači istinitost neke formule u \(\Delta\),
  u skladu s intuitivnim shvaćanjem sekventi.
\begin{minted}{coq}
Definition Sat_sequent (s : sequent) : Prop :=
  let '(Γ ⊢ Δ) := s in            
  forall (M : structure Σ), standard_model Σ Φ M -> forall (ρ : env M),
      (forall φ, In φ Γ -> ρ ⊨ φ) -> exists ψ, In ψ Δ /\ ρ ⊨ ψ.
\end{minted}
\end{definition}

Kažemo da vrijedi \textit{lokalna adekvatnost} za pravilo izvoda
ako istinitost premisa (a posebno i sekvenata u premisama) povlači istinitost konkluzije (odnosno sekvente u konkluziji).
Lokalna adekvatnost kaže da se istinitost čuva kod primjena pravila zaključivanja.

\begin{example}
  Za pravilo \(\mathit{Ax}\) vrijedi lokalna adekvatnost,
  to jest ako u listama formula \(\Gamma\) i \(\Delta\) postoji neka zajednička formula \(\varphi\),
  onda vrijedi \(\Gamma \VDash \Delta\). Tada je \textit{svjedok istinitosti} u \(\Delta\) upravo formula \(\varphi\).
\begin{minted}{coq}
Lemma LS_Ax : forall Γ Δ φ, In φ Γ -> In φ Δ -> Γ ⊫ Δ.
\end{minted}
\end{example}

Ako za svako pravilo izvoda vrijedi lokalna adekvatnost,
indukcijom po strukturi dokaza slijedi i (globalna) adekvatnost.
Glavni cilj ovog poglavlja bio je dokazati lokalne adekvatnosti svih pravila sustava \(\mathit{LKID}\).
Cilj je postignut za sva pravila osim pravila indukcije.

\begin{theorem}
  Za pravila izvoda sustava \(\mathit{LKID}\) osim \(\mathit{IndL}\) vrijedi lokalna adekvatnost.
  Posljedično, ako je sekventa \(\Gamma \vdash \Delta\) dokaziva bez primjene pravila \(\mathit{IndL}\), onda je ona istinita.
\end{theorem}

\misao{Ako ne uspijem dokazati lokalnu adekvatnost za IndL, onda ću ostaviti ovaj tekst.}

%%% Local Variables:
%%% mode: LaTeX
%%% TeX-master: "master"
%%% End:

\chapter{Ciklički dokazi}\label{cha:cikliki-dokazi}
Svaki Coqov dokaz koji smo do sada prikazali je konačan term.
Svi formalni dokazi u sustavu \(\mathit{LKID}\) (uključujući primjere~\ref{ex:lkid-indr-proof} i~\ref{ex:lkid-indl-proof}) su konačni.
Ovakve dokaze jednim imenom nazivamo \textit{dobro utemeljenim dokazima} jer su sve \enquote{grane} njihovih dokaznih stabala konačne.
Međutim, možemo promatrati i \textit{neutemeljene dokaze} kod kojih postoje beskonačne grane.
Već smo u odjeljku~\ref{sec:programiranje-u-gallini} spomenuli neutemeljene, odnosno koinduktivne tipove.
Neutemeljeni dokaz je tada \enquote{beskonačan} dokaz neke koinduktivno definirane tvrdnje.
U ovom ćemo kratkom poglavlju ilustrirati neutemeljene dokaze te cikličke dokaze koji su posebna vrsta neutemeljenih.

\section{Primjer: lijene liste}
Definirajmo za početak \textit{lijene liste}, to jest liste koje mogu biti konačne, a mogu biti i beskonačne.
Zbog potencijalne beskonačnosti koristimo koinduktivnu definiciju tipa.
\begin{minted}{coq}
CoInductive LList (A : Type) :=
| LNil : LList A
| LCons : A -> LList A -> LList A.
\end{minted}
\noindent Kažemo da je lijena lista \textit{beskonačna} ako je njen rep opet beskonačna lijena lista.
\begin{minted}{coq}
CoInductive Infinite {A} : LList A -> Prop :=
| Infinite_LCons : forall a l, Infinite l -> Infinite (LCons a l).
\end{minted}
\noindent Ovdje konstruktor \texttt{Infinite\_LCons} možemo shvatiti kao pravilo izvoda.
Pokušajmo dokazati da je intuitivno beskonačna lista i formalno beskonačna.
Definiramo funkciju \texttt{from} analogno istoimenoj funkciji iz odjeljka~\ref{sec:programiranje-u-gallini}.
\begin{minted}{coq}
CoFixpoint from (n : nat) : LList nat := LCons n (from (S n)).
\end{minted}
\noindent Intuitivno je jasno da je za svaki \texttt{n} lijena lista \texttt{from n}
beskonačna pa želimo dokazati tvrdnju \texttt{forall n, Infinite (from n)}.

Prije glavnog dokaza, pogledajmo konkretan primjer na tvrdnji \texttt{Infinite (from 0)}.
Po definiciji funkcije \texttt{from}, dovoljno je dokazati
tvrdnju \texttt{Infinite (LCons 0 (from 1))}, a primjenom pravila \texttt{Infinite\_LCons}
dolazimo do tvrdnje \texttt{Infinite (from 1)}.
Opet je po definiciji funkcije \texttt{from} dovoljno dokazati tvrdnju
\texttt{Infinite (LCons 1 (from 2))}, a ponovnom primjenom pravila \texttt{Infinite\_LCons}
dolazimo do tvrdnje \texttt{Infinite (from 2)}.
Ilustrirajmo ovaj postupak dokaznim stablom.
\begin{prooftree}
  \AxiomC{\vdots}
  \UnaryInfC{\texttt{Infinite (from 2)}}
  \RightLabel{(\texttt{Infinite\_LCons})}
  \UnaryInfC{\texttt{Infinite (LCons 1 (from 2))}}
  \UnaryInfC{\texttt{Infinite (from 1)}}
  \RightLabel{(\texttt{Infinite\_LCons})}
  \UnaryInfC{\texttt{Infinite (LCons 0 (from 1))}}
  \UnaryInfC{\texttt{Infinite (from 0)}}
\end{prooftree}
\noindent Sada je jasno da ćemo dokazati originalnu tvrdnju tek nakon beskonačno
mnogo primjena pravila \texttt{Infinite\_LCons}. Ipak, lakše je dokazati
tvrdnju \texttt{forall n, Infinite (from n)}.
Nakon introdukcije varijable \texttt{n}, odmatanja konstante \texttt{from n} i
primjene pravila \texttt{Infinite\_LCons}
dolazimo do tvrdnje \texttt{Infinite (from (S n))}.

\begin{prooftree}
  \AxiomC{\texttt{*}}
  \UnaryInfC{\texttt{Infinite (from (S n))}}
  \RightLabel{(\texttt{Infinite\_LCons})}
  \UnaryInfC{\texttt{Infinite (LCons n (from (S n)))}}
  \UnaryInfC{\texttt{Infinite (from n)}}
  \UnaryInfC{\texttt{forall n, Infinite (from n)}}
\end{prooftree}

\noindent Između tvrdnji \texttt{Infinite (from n)} i \texttt{Infinite (from (S n))}
\enquote{skinut} je prvi član beskonačne liste te je napravljen \textit{napredak}
u dokazu.
Bez napretka nema smisla ponavljati prethodni postupak beskonačno mnogo puta
jer se tvrdnja ne mijenja.
Uz napredak u dokazivanju ima smisla pozivati se na \textit{tvrdnju koja se dokazuje} te
takve dokaze nazivamo \textit{cikličkima}. Dokaz tvrdnje \texttt{forall n, Infinite (from n)}
jedan je primjer cikličkog dokaza, gdje je simbolom \(\mathtt{*}\) označeno cikličko \enquote{pozivanje} dokaza.
U Coqu, napredak je osiguran uvjetom produktivnosti kojeg smo spomenuli
u odjeljku~\ref{sec:ogranicenja}.

Pažljiv čitatelj će primijetiti da ista argumentacija vrijedi
i za tvrdnju \texttt{Finite (from n)}, gdje je predikat \texttt{Finite} definiran na idući način.
\begin{minted}{coq}
Inductive Finite {A} : LList A -> Prop :=
| Finite_LNil : Finite LNil
| Finite_LCons : forall a l, Finite l -> Finite (LCons a l).
\end{minted}
\noindent Međutim, beskonačne liste po definiciji nisu konačne.
Zašto ipak gornji dokazi \textit{ne} prolaze za predikat \texttt{Finite}?
Riječ je o načinu definicije; predikat \texttt{Infinite} definiran je koinduktivno
te njegovi stanovnici \textit{nisu} dobro utemeljeni,
dok je predikat \texttt{Finite} definiran induktivno, a njegovi stanovnici \textit{jesu}
dobro utemeljeni. Štoviše, kada bismo predikat \texttt{Finite} definirali koinduktivno,
on bi postao trivijalan, to jest vrijedio bi za proizvoljnu lijenu listu.

\section{Primjer: \(\mathit{CLKID}^{\omega}\)}
Brotherston je formalizirao ideju cikličkih dokaza u dokaznom sustavu
\(\mathit{CLKID}^{\omega}\), podsustavu dokaznog sustava \(\mathit{LKID}^{\omega}\),
koji pak je proširenje sustava \(\mathit{LKID}\) potencijalno beskonačnim dokaznim stablima.
Dokazni sustav \(\mathit{CLKID}^{\omega}\) ograničen je u odnosu na sustav \(\mathit{LKID}^{\omega}\)
tako što svako (potencijalno beskonačno) dokazno stablo
smije imati \textit{najviše konačno mnogo} različitih podstabala te
svaka beskonačna grana tog stabla mora imati beskonačno mnogo \enquote{napredaka}.
Dokazni sustav \(\mathit{CLKID}^{\omega}\) prikazati ćemo neformalno.

\begin{example}
  Dokazujemo tvrdnju \(\forall x, \mathit{Nat}(x) \rightarrow \mathit{Even}(x) \lor \mathit{Odd}(x)\)
  u sustavu \(\mathit{CLKID}^{\omega}\).
  Koristimo pokrate kao u primjeru~\ref{ex:lkid-indl-proof}..
  \begin{prooftree}\label{ex:clkidw1}
    \AxiomC{}
    \RightLabel{(\texttt{PA\_prod\_E\_zero}\textit{Prod})}
    \UnaryInfC{\(\vdash Eo, Oo\)}
    \AxiomC{\(Nx \vdash Ex, Ox \, (\dagger)\)}
    \RightLabel{\( \mathit{(Subst)} \)}
    \UnaryInfC{\(Ny \vdash Ey, Oy\)}
    \RightLabel{\(\mathit{(Perm)}\)}
    \UnaryInfC{\( Ny \vdash Oy, Ey \)}
    \RightLabel{(\texttt{PA\_prod\_O\_succ}\textit{Prod})}
    \UnaryInfC{\( Ny \vdash Oy, Osy \)}
    \RightLabel{(\texttt{PA\_prod\_E\_succ}\textit{Prod})}
    \UnaryInfC{\(Ny \vdash Esy, Osy\)}
    \RightLabel{\(\mathit{(EqL)}\)}
    \UnaryInfC{\(x = sy, Ny \vdash Ex, Ox\)}
    \RightLabel{\(\mathit{(Case \, N)}\)}    
    \BinaryInfC{\( Nx \vdash Ex, Ox \, (\dagger) \)}
    \RightLabel{\(\mathit{(OrR)}\)}
    \UnaryInfC{\( Nx \vdash Ex \lor Ox \)}
    \RightLabel{\(\mathit{(ImpR)}\)}
    \UnaryInfC{\(\vdash Nx \rightarrow Ex \lor Ox\)}
    \RightLabel{\(\mathit{(AllR)}\)}
    \UnaryInfC{\(\vdash \forall x, Nx \rightarrow Ex \lor Ox\)}
  \end{prooftree}
  Iako zapisan u konačnom stablu, ovaj dokaz je beskonačan.
  Čitajući dokaz od dna do vrha, primjećujemo nove oznake te nova pravila.
  Za početak, vidimo da su znakom \(\dagger\) označene gornja i donja sekventa \(Nx \vdash Ex, Ox\).
  Gornju sekventu nazivamo \textit{pupoljkom} (\textit{bud}), a donju nazivamo \textit{pratilcem} (\textit{companion}).
  Može se reći da su pupoljak i njegov pratilac povezani bridom, tvoreći ciklus u dokaznom
  stablu --- odavde dolazi naziv \textit{ciklički} dokaz.
  U stvari se isječak stabla dokaza između pratilca i pupoljka ponavlja \textit{ad infinitum}
  pa je dokaz beskonačan.
  Nadalje, vidimo novo pravilo izvoda \(\mathit{Case}\) koje odgovara rastavu po slučajevima,
  a koristi se umjesto pravila indukcije.
  Ako se između pupoljka i njegovog pratilca javlja primjena pravila \(\mathit{Case}\),
  kažemo da je napravljen \textit{napredak}.
  Konačno, vidimo znak jednakosti uz kojeg je vezano lijevo pravilo za jednakost \(\mathit{EqL}\).
  Znak jednakosti je za potrebe sustava \(\mathit{LKID}^{\omega}\) i \(\mathit{CLKID}^{\omega}\)
  (odnosno, \(\mathit{Case}\) pravila) ugrađen u definiciju formule.\footnote{Jednakost se u matematičkoj logici najčešće tretira kao predikatni simbol signature, a ne kao primitivni veznik u formulama.}
\end{example}

U primjeru~\ref{ex:clkidw1}, pupoljak i njegov pratilac javljaju se u istoj grani dokaza,
no to nije nužno. Ovu pojavu možemo vidjeti na dokazu obrata sekvente \(Nx \vdash Ex \lor Ox\).


\begin{example}
  Dokazujemo sekventu \(Ex \lor Ox \vdash Nx \) u sustavu \(\mathit{CLKID}^{\omega}\).
  \begin{prooftree}
    \AxiomC{}
    \RightLabel{\((1)\)}
    % \RightLabel{(\texttt{PA\_prod\_N\_zero}\textit{Prod})}
    \UnaryInfC{\(\vdash No\)}
    \RightLabel{\(\mathit{(EqL)}\)}
    \UnaryInfC{\(x = o \vdash Nx\)}
    \AxiomC{\(Ox \vdash Nx \, (\dagger)\)}
    \RightLabel{\(\mathit{(Subst)}\)}
    \UnaryInfC{\(Oy \vdash Ny\)}
    % \RightLabel{(\texttt{PA\_prod\_N\_succ}\textit{Prod})}
    \RightLabel{\((2)\)}
    \UnaryInfC{\(Oy \vdash Nsy\)}
    \RightLabel{\(\mathit{(EqL)}\)}
    \UnaryInfC{\(x = sy, Oy \vdash Nx \)}
    \RightLabel{\(\mathit{(Case \, E)}\)}
    \BinaryInfC{\(Ex \vdash Nx \, (\ast)\)}
    \AxiomC{\(Ex \vdash Nx \, (\ast)\)}
    \RightLabel{\(\mathit{(Subst)}\)}
    \UnaryInfC{\(Ey \vdash Ny\)}
    % \RightLabel{(\texttt{PA\_prod\_N\_succ}\textit{Prod})}
    \RightLabel{\((3)\)}
    \UnaryInfC{\(Ey \vdash Nsy\)}
    \RightLabel{\(\mathit{(EqL)}\)}
    \UnaryInfC{\(x = sy, Ey \vdash Nx\)}
    \RightLabel{\((\mathit{Case \, O})\)}
    \UnaryInfC{\( Ox \vdash Nx \, (\dagger) \)}
    \RightLabel{\(\mathit{(OrL)}\)}
    \BinaryInfC{\( Ex \lor Ox \vdash Nx \)}
  \end{prooftree}
  Ovdje je znakom \(\dagger\) označen par pupoljka i pratilca sekvente \(Ox \vdash Nx\),
  a znakom \(\ast\) je označen par pupoljka i pratilca sekvente \(Ex \vdash Nx\).
  Iako se odgovarajući parovi pupoljaka i pratilaca naizgled javljaju u različitim granama,
  uz dva \enquote{odmatanja} stabla dokaza vidimo da to u stvari nije slučaj.
  \begin{scriptsize}
    \begin{prooftree}
      \AxiomC{}
      \RightLabel{\((1)\)}
      % \RightLabel{(\texttt{PA\_prod\_N\_zero}\textit{Prod})}
      \UnaryInfC{\(\vdash No\)}
      \RightLabel{\(\mathit{(EqL)}\)}
      \UnaryInfC{\(x = o \vdash Nx\)}

      \AxiomC{\(Ex \vdash Nx \, (\ast)\)}
      \RightLabel{\(\mathit{(Subst)}\)}
      \UnaryInfC{\(Ey \vdash Ny\)}
      % \RightLabel{(\texttt{PA\_prod\_N\_succ}\textit{Prod})}
      \RightLabel{\((2)\)}
      \UnaryInfC{\(Ey \vdash Nsy\)}
      \RightLabel{\(\mathit{(EqL)}\)}
      \UnaryInfC{\(x = sy, Ey \vdash Nx\)}
      \RightLabel{\((\mathit{Case \, O})\)}
      \UnaryInfC{\( Ox \vdash Nx \, \xcancel{(\dagger)}\)}
      
      \RightLabel{\(\mathit{(Subst)}\)}
      \UnaryInfC{\(Oy \vdash Ny\)}
      % \RightLabel{(\texttt{PA\_prod\_N\_succ}\textit{Prod})}
      \RightLabel{\((2)\)}
      \UnaryInfC{\(Oy \vdash Nsy\)}
      \RightLabel{\(\mathit{(EqL)}\)}
      \UnaryInfC{\(x = sy, Oy \vdash Nx \)}
      \RightLabel{\(\mathit{(Case \, E)}\)}
      \BinaryInfC{\(Ex \vdash Nx \, (\ast)\)}

      \AxiomC{}
      \RightLabel{\((1)\)}
      % \RightLabel{(\texttt{PA\_prod\_N\_zero}\textit{Prod})}
      \UnaryInfC{\(\vdash No\)}
      \RightLabel{\(\mathit{(EqL)}\)}
      \UnaryInfC{\(x = o \vdash Nx\)}
      \AxiomC{\(Ox \vdash Nx \, (\dagger)\)}
      \RightLabel{\(\mathit{(Subst)}\)}
      \UnaryInfC{\(Oy \vdash Ny\)}
      % \RightLabel{(\texttt{PA\_prod\_N\_succ}\textit{Prod})}
      \RightLabel{\((2)\)}
      \UnaryInfC{\(Oy \vdash Nsy\)}
      \RightLabel{\(\mathit{(EqL)}\)}
      \UnaryInfC{\(x = sy, Oy \vdash Nx \)}
      \RightLabel{\(\mathit{(Case \, E)}\)}
      
      \BinaryInfC{\(Ex \vdash Nx \, \xcancel{(\ast)}\)}
      \RightLabel{\(\mathit{(Subst)}\)}
      \UnaryInfC{\(Ey \vdash Ny\)}
      % \RightLabel{(\texttt{PA\_prod\_N\_succ}\textit{Prod})}
      \RightLabel{\((2)\)}
      \UnaryInfC{\(Ey \vdash Nsy\)}
      \RightLabel{\(\mathit{(EqL)}\)}
      \UnaryInfC{\(x = sy, Ey \vdash Nx\)}
      \RightLabel{\((\mathit{Case \, O})\)}
      \UnaryInfC{\( Ox \vdash Nx \, (\dagger) \)}
      \RightLabel{\(\mathit{(OrL)}\)}
      \BinaryInfC{\( Ex \lor Ox \vdash Nx \)}
    \end{prooftree}
  \end{scriptsize}
  Pupoljak sekvente \(Ex \vdash Nx\) je zamijenjen
  isječkom stabla od pratilca sekvente \(Ex \vdash Nx\) do pupoljka sekvente \(Ox \vdash Nx\),
  Nadalje, pupoljak sekvene \(Ox \vdash Nx\) zamijenjen je
  isječkom stabla od pratilca sekvente \(Ox \vdash Nx\) do pupoljka sekvente \(Ex \vdash Nx\).
  Dokaz ne bi bio moguć bez korištenja produkcijskih pravila za produkcije:
  \begin{enumerate}[label={(\arabic*)}]
  \item \texttt{PA\_prod\_N\_zero} i
  \item \texttt{PA\_prod\_N\_succ}.
  \end{enumerate}
\end{example}

Važno je spomenuti da je svojstvo \enquote{biti ispravan \(\mathit{CLKID}^{\omega}\) dokaz} odlučivo
te je na temelju toga napravljen ciklički dokazivač teorema \textsc{Cyclist}\footnote{\url{http://www.cyclist-prover.org/}}~\cite{cyclist}.

%%% Local Variables:
%%% mode: LaTeX
%%% TeX-master: "master"
%%% End:

\chapter{Zaključak}\label{cha:zakljucak}
Cilj ovog rada bio je ponuditi pregled visoke razine na alat za dokazivanje Coq te
ga primijeniti na formalizaciju logike prvog reda s induktivnim definicijama.
U tu svrhu smo u prvom poglavlju prikazali osnove Coqa i teorije na kojoj se on temelji,
dok smo ostala poglavlja posvetili logici.
Definirali smo osnovne pojmove poput formule, strukture, standardnog modela te dokaznog sustava
i dokazali smo neka njihova svojstva.
Kroz cijeli rad ilustrirali smo definicije i leme neformalnim i formaliziranim primjerima.
Iako je ponekad bilo teže razlučiti bitne dijelove neformalne definicije,
ipak se pokazalo da je Coq prikladan alat za formalizaciju logike te autor smatra
da je cilj rada uspješno postignut.

Ovim radom smo tek zagrebali površinu formalizacije logike prvog reda s induktivnim definicijama.
Prirodni nastavak rada jest dokaz potpunosti sustava \(\mathit{LKID}\) pomoću Henkinovih modela,
a nakon toga i formalizacija cikličkog dokaznog sustava \(\mathit{CLKID}^{\omega}\).
Nadalje, ima smisla formalizirati dokazivače teorema za sustave \(\mathit{LKID}\)
i \(\mathit{CLKID}^{\omega}\). Kako logika prvog reda nije odlučiva, takvi dokazivači teorema
trebali bi uvesti dodatne pretpostavke (kao što je pretpostavka zatvorenog svijeta u Prologu)
ili se usmjeriti prema interaktivnom dokazivanju.
Za našu formalizaciju sustava \(\mathit{LKID}\),
ima smisla iskoristiti dodatne Coqove mogućnosti,
posebno za automatizaciju dokaza jezikom taktika Ltac
te za prikladniju notaciju terma, formula i dokaza.

Po autorevu mišljenju, računalna formalizacija matematike je odličan način razmišljanja i razumijevanja,
posebno kada se koristi zajedno s tradicionalnim učenjem jer tada u potpunosti
postaje jasan pojam koji se proučava --- formalizacija nas tjera da budemo
posebice pažljivi. Nadalje, u današnje vrijeme umjetne inteligencije i automatski
generiranog koda, dokazivanje točnosti programa postaje bitnije no što je ikad bilo.
Za kraj ćemo reći da je formalizacija često naporan ali uvijek zadovoljavajuć proces.

%%% Local Variables:
%%% mode: LaTeX
%%% TeX-master: "master"
%%% End:


\bibliography{literatura}

\begin{sazetak}
  Coq je interaktivni dokazivač teorema temeljen na teoriji tipova.
  U ovom radu primijenili smo alat Coq na zadatak formalizacije
  logike prvog reda s induktivnim definicijama \(\mathit{FOL}_{\mathit{ID}}\). Nadalje,
  definirali smo standardne modele i dokazni sustav \(\mathit{LKID}\) za logiku \(\mathit{FOL}_{\mathit{ID}}\) te
  dokazali neka njihova svojstva. Također smo neformalno prikazali cikličke dokaze
  u kontekstu logike \(\mathit{FOL}_{\mathit{ID}}\).
  Sve definicije i leme u radu ilustrirali smo primjerima te formalizacijom u Coqu.  
\end{sazetak}

\begin{kljucnerijeci}
  Coq; logika prvog reda; induktivne definicije; dokaz
\end{kljucnerijeci}

\begin{abstract}
  Coq is an interactive theorem prover based on type theory.
  In this thesis we applied the Coq Proof Assistant to the task of formalizing
  first order logic with inductive definitions \(\mathit{FOL}_{\mathit{ID}}\). Furthermore,
  we defined standard models and the \(\mathit{LKID}\) proof system for \(\mathit{FOL}_{\mathit{ID}}\),
  and we proved some of their properties. We also informally presented cyclic proofs in the context of \(\mathit{FOL}_{\mathit{ID}}\).
  We illustrated all the definitions and lemmas in this thesis with examples and formalisations in Coq.
\end{abstract}

\begin{keywords}
  Coq; first order logic; inductive definitions; proof
\end{keywords}


\backmatter{}
% \chapter{Example appendix chapter}
\end{document}

%%% Local Variables:
%%% mode: LaTeX
%%% TeX-master: t
%%% End:
